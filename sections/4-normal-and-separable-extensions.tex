\section{Splitting Fields and Normal Extensions}

Given a field $K$, we are now mostly interested in field extensions $L/K$ that arise from adjoining some roots of a polynomial $f$ over $K$. As we have seen, such an extension is algebraic and finite, but further questions could be asked. In particular, does $L$ contain \textit{all} roots of $f$, instead of only a proper subset of them? If,  for any irreducible polynomial $f$ over $K$ that has at least one root in $L$, the answer is ``yes'', then we say $L/K$ is a \textit{normal} extension. 
A natural way to construct a normal extension would then be adjoining all roots of some chosen polynomial $f$ over $K$. Indeed, as we shall see, this gives rise to the idea of splitting fields and being the splitting field of some $f$ is in fact an equivalent condition for an extension to be normal. 

In this section, we first introduce splitting fields, with a particular focus on how they can help construct field automorphisms. Then we move on to algebraic closures, which can be considered as the splitting field of \textit{all} polynomials over a field. We then present normality and separability of field extensions. 


%Two important properties of field extensions, normality and separability, are the key ingredients for Galois extensions which we will discuss later. To pave the way for them, we need to first introduce splitting fields and algebraic closures, two closely related concepts. 

% \TODO maybe place the definitions somewhere else?
\subsection{Splitting Fields and Algebraic Closures}

The following discussions on splitting fields are adapted from Stewart \cite{Stewart}. 
\begin{definition}
    Let $K$ be a field. A polynomial $f$ over $K$ \textit{splits} in $K$ if it decomposes into linear factors $
    f(t) = k \prod _{i=1} ^n (t - \alpha_i),
    $
    where $k, \alpha_1, \ldots, \alpha_n \in K$. 
\end{definition}
If $K$ is a subfield of $\C$, Theorem \ref{thm:fundamental-algebra-2} indicates that $f$ splits over $K$ if and only if all of its roots are contained in $K$. 
Of all the fields in which $f$ splits, we are particularly interested in the smallest:

\begin{definition}
    % A \textbf{\textit{splitting field}} of a polynomial $f$ over a subfield $K$ of $\C$ is a subfield $\Sigma$ of $\C$ such that 
    % \begin{itemize}
    %     \item $K \subseteq \Sigma$,
    %     \item $f$ splits over $\Sigma$ and
    %     \item for any $\Sigma'$ such that $K \subseteq \Sigma' \subseteq \Sigma$ and $f$ splits over $\Sigma'$, we have $\Sigma' = \Sigma$. 
    % \end{itemize}
    The splitting field $F$ of a polynomial $f$ over a field $K$ is defined by $F = K(\alpha_1, \ldots, \alpha_n), $ where $\alpha_i$ are roots of $f$. 
\end{definition}

From this point onwards all of our splitting fields are referred to as $F$.

\begin{example}
	If we have a polynomial \(f(t) = t^4 - 12t^2 + 35\) the splitting field of \(f(t)\) is \(\mathbb{Q}(\sqrt{5},\sqrt{7})\) as it contains all of the roots of \(f(t)\) and if it was any smaller it would not contain all of the roots or would not be a field.
\end{example}

We see that $F$ is the smallest field containing $K$ and all roots of $f$. The uniqueness of $F$ follows directly from its definition. Also, each $\alpha_i$ is algebraic over $K$ and their number is finite, so $F / K$ is a finite extension by Theorem \ref{thm:finite-equi-def}. 
%Further, we now prove an upper bound for degree of a splitting field.
%\begin{theorem}\label{thm:upper-bound-splitting-field}
%	For every $f \in K[t]$ of degree $n>0$, there exists a field extension $L/K$ such that $L$ is a splitting field of $f$ over $K$ with $[L:K]\leq n!$. 
%\end{theorem}
%
%\begin{proof}
%	We construct a proof by induction on the degree of $f$, $n$. 
%	For the case where $n = 1$, this is trivial, where $L=K$.
%	Now, for $n>1$, let $h$ be an irreducible polynomial over $K$ which divides $f$ and let $a_1$ be a root of $h$ such that $a_1$ lies in $L_1$, where $L_1/K$ is a field extension with $[L_1 : K] = \partial h \le n$. Then $a_1$ is also a root of $f$, and then we can say that $f=(t-a_1)f_1(t)$, where $f_1\in L_1[t]$, and since $f$ has degree $n$, $f_1$ has degree $n-1$. By our induction hypothesis there must be a field extension $L/L_1$ such that $[L:L_1]\leq(n-1)!$. Then $f$ must split into linear factors in $L$, and so if we define the roots of $f_1$ as $a_2,...,a_n\in L$, we see that $L=L_1(a_2,...,a_n)$ and $L_1=K(a_1) \implies L=K(a_1,a_2,...,a_n)$. Thus by Theorem \ref{thm:tower-theorem}, we get $[L:K]=[L:L_1][L_1:K]\leq (n-1)!\cdot n = n!$.
%
%%	 let $f=gh$, with $g,h \in K[t]$ where $h$ is irreducible. Then, there is an extension $L_1/K$, such that $[L_1:K]=\partial h \leq n$, with a root of $f$ in $L_1$. Let us define the root as $a_1$, 
%\end{proof}
The splitting fields of a polynomial over two isomorphic fields are isomorphic, in the following sense:

% Theorem 9.6 Stewart
\begin{theorem} \label{thm:splitting-field-unique}
	Let $\iota: K \to K'$ be a field isomorphism. Let $f$ be a polynomial over $K$ with the splitting field $F$, and then let $F'$ be the splitting field of $\iota(f)$ over $K'$. Then there is an isomorphism $j : F \to F'$ such that $j | _K = \iota$. 
\end{theorem}
%\TODO add theorem 5.16 of Stewart to simple field extensions

\begin{proof}
	We construct $j$ by induction on the degree of $f$. We write $ f(t) = k (t - \alpha_1) \ldots (t - \alpha_n), $ where $\alpha_i \in F$ and $k \in K$. Let $m_1$ be the minimal polynomial of $\alpha_1$ over $K$. $m_1$ is an irreducible factor of $f$ and hence $\iota(m_1)$ is an irreducible factor of $\iota(f)$. Thus $\iota(m_1)$ splits in $F'$. Let $\iota(m_1)(t) = (t - \beta_1) \ldots (t - \beta_n)$, where $\beta_i \in F'$ and $\iota(m_1)$ is the minimal polynomial of $\beta_i$ over $K'$. Then by Theorem \ref{thm:minimal-polynomial-roots-isomorphic}, there is an isomorphism $j_1 : K(\alpha_1) \to K'(\beta_1)$ such that $j_1 | _K = \iota$ and $j_1(\alpha_1) = \beta_1$. Now $f / (t - \alpha_1)$ is a polynomial over $K(\alpha_1)$ with splitting field $F$, and $F'$ is the splitting field of $\iota(f / (t - \alpha_1))$ over $K'(\beta_1)$. By induction, there exists an isomorphism $j: F \to F'$ such that $j | _{K(\alpha_1)} = j_1$ and therefore $j | _K = \iota$. 
\end{proof}



Theorem \ref{thm:splitting-field-unique} provides a convenient way to ``extend'' a field isomorphism using splitting fields. In other words, given a ``smaller'' isomorphism $\iota: K \to K'$, we end up with a ``larger'' one $j: F \to F'$, where $K \subseteq F$ and $K' \subseteq F'$, and further $j$ preserves $\iota$ on $K$. Notably, when $F  = F'$, $j$ is an automorphism of $F$. The next theorem captures the ideas and enables us to ``extend" a $K$-monomorphism $M \to F$ to a $K$-automorphism of $F$, where $K \subseteq M \subseteq F$. 

\begin{theorem} \label{thm:monomorphism-extend-automorphism}
	Let $f$ be a polynomial over $K$ with splitting field $F$. Let $M$ be an intermediate field such that $K \subseteq M \subseteq F$. Let $\tau$ be a $K$-monomorphism $M \to F$. Then there exists a $K$-automorphism $\sigma$ of $F$ such that $\sigma | _M = \tau$.  
%	Let $L/K$ be a normal field extension and let $M$ be an intermediate field such that $K \subseteq M \subseteq L$. Let $\tau$ be a $K$-monomorphism $M \to L$. Then there exists a $K$-automorphism $\sigma$ of $L$ such that $\sigma | _M = \tau$.  
\end{theorem}

\begin{proof}
%	By (i) $\Rightarrow$ (ii) of Theorem \ref{thm:normal-equiv-def}, 
	We know that $F$ is the splitting field for $f$ over $K$ and thus over $M$. We consider $\tau$ as an isomorphism from $M$ to $\tau(M)$ which fixes every element in $K$ and hence $f$. Then $F$ is both the splitting field for $f$ over $M$ and the splitting field for $f$ over $\tau(M)$. Then by Theorem \ref{thm:splitting-field-unique}, there exists an isomorphism $\sigma: F \to F$ such that $\sigma | _M = \tau$. Since $\sigma | _K = \tau |_K = \Id$, $\sigma$ is a $K$-automorphism of $F$. 
\end{proof}

When $f$ is irreducible, we can then construct an $K$-automorphism of $F$ that sends a certain root of $f$ to another. This is a major result about splitting fields. 

\begin{theorem} \label{thm:automorphism-from-zeros}
	Let $f$ be an irreducible polynomial over $K$ with splitting field $F$. Let $\alpha, \beta \in F$ be roots of $f$. Then there exists a $K$-automorphism $\sigma$ of $F$ such that $\sigma(\alpha) = \beta$. 
\end{theorem}

\begin{proof}
	$K(\alpha)/K$ and $K(\beta)/K$ are isomorphic extensions by Corollary \ref{thm:minimal-polynomial-roots-isomorphic}. The isomorphism from $K(\alpha)$ to $K(\beta)$ is also a $K$-monomorphism from $K(\alpha)$ to $F$ and thus can be extended to a $K$-automorphism $\sigma$ of $F$ by Theorem \ref{thm:monomorphism-extend-automorphism}, where $\sigma(\alpha) = \beta$.
\end{proof}

The following discussions on algebraic closures combine ideas from \cite{galois-theory-lectures, rotman_galois_1998}. 

\begin{definition}
	A field $K$ is \textit{algebraically closed} if every non-constant polynomial over $L$ has a root in $K$. 
\end{definition}

Clearly $\C$ is algebraically closed by Theorem \ref{thm:fundamental-algebra}. It can be shown that we can extend any field to an algebraically closed one \cite{rotman_galois_1998}. 

\begin{definition}
	The smallest algebraically closed extension of a field $K$ is its \textit{algebraic closure}, denoted as $\overline K$.
\end{definition}

\begin{example}
	$\overline \R  =  \C$. That is, $\C$ is the smallest field that contains a root of any non-constant polynomial over $\R$ \cite{rotman_galois_1998}.
\end{example}


For a field $K$, $\overline K$ is the field of all $\alpha$ such that $\alpha$ is algebraic over $K$. Equivalently, if we define the splitting field of \textit{a set of polynomials} over a field $K$ as the smallest field containing $K$ and all roots of every polynomial in the set, then it can be shown that the algebraic closure of a field $K$ is the splitting field of the set of \textit{all} polynomials over $K$. In other words, any polynomial over $K$ splits in $\overline K$. In addition, any polynomial over $\overline K$ splits in $\overline K$.

Let us define $\mathbb A = \overline \Q$, the field of algebraic numbers. Clearly $\mathbb A$ is a proper subset of $\C$, as $\mathbb A$ is countable (due to the countability of polynomials over $\Q$) and does not contain any transcendental numbers. A natural question to ask is whether any element in $\mathbb A$ is expressible by radicals, and clearly this amounts to whether any polynomial over $\Q$ is solvable by radicals. We discuss this topic in detail in Section \ref{sec:galois-groups-and-polynomials}. 

%Isomorphic fields have isomorphic algebraic closures, in the following sense:
%
%\begin{theorem} 
%	Let $\iota: K \to K'$ be a field isomorphism. Then there is an isomorphism $j: \overline K \to \overline {K'}$ such that $j |_K = \iota$. 
%\end{theorem}
%
%The detailed proof is omitted, but resembles that of Theorem \ref{thm:splitting-field-unique}.

%\begin{definition}
%    A field $K$ is algebraically closed if every polynomial $f$ over $K$ has a root in $K$. 
%\end{definition}
%
%\begin{theorem}
%    For a field $K$, its smallest algebraically closed extension is the algebraic closure of $K$.
%\end{theorem}



\subsection{Normal and Separable Extensions}

We now state the definition of a normal extension $L/K$, which says that an irreducible polynomial over $K$ has ``none or all'' of its roots in $L$. 

\begin{definition}
    An algebraic field extension $L/ K$ is \textit{normal} if every irreducible polynomial over $K$ which has a root in $L$ splits in $L$. 
\end{definition}

\begin{example}
	For any field $K$, $\overline K / K$ is a normal extension, as every polynomial over $K$ splits in $\overline K$. In particular, $\C / \R$ is normal. 
\end{example}


\begin{example}
    Let $K = \Q(\sqrt[3]{2})$. Then $K / \Q$ is not normal. The polynomial $f(t) = t^3 - 2$ over $\Q$ is irreducible and has a root $\sqrt[3]{2}$ in $K$, but $K \subseteq \R$ does not contain the two non-real roots of $f$. Hence $f$ does not split in $K$.
\end{example}

We now establish some equivalent statements as a finite extension being normal. The next theorem formalizes ideas from \cite{galois-theory-lectures}. (We have limited our discussion on \textit{finite} extensions, while \cite{galois-theory-lectures} does not have this assumption.)

\begin{theorem} \label{thm:normal-equiv-def}
    Let $L/K$ be an finite field extension. Then the following three statements are equivalent: (i) $L/K$ is normal; 
    (ii) $L$ is the splitting field for a polynomial $f$ over $K$; 
    (iii) Any $K$-monomorphism $\sigma: L \to \overline K$ is a $K$-automorphism of $L$. 
    % \end{enumerate}
    % \begin{enumerate}[label=(\roman*)]
    %     \item $L/K$ is normal;
    %     \item $L$ is the splitting field for a polynomial $f$ over $K$;
    %     \item Any $K$-monomorphism $\sigma: L \to \overline K$ is a $K$-automorphism of $L$. 
    % \end{enumerate}
\end{theorem}

\begin{proof}
    (i) $\Rightarrow$ (ii). 
    Suppose $L/K$ is normal and finite. The finiteness of $L/K$ implies that $L = K(\alpha_1, \dots, \alpha_r)$ for $r$ finite and $\alpha_i$ algebraic over $K$ by Theorem \ref{thm:finite-equi-def}. For each $\alpha_i$, take the minimal polynomial $f_i$ of $\alpha_i$ over $K$. Each $f_i$ is irreducible over $K$, so the normality of $L/K$ implies that $f_i$ splits in $L$.  Then take $f = f_1 \dots f_r$ and $L$ is the splitting field of $f$.
	
%	    For each $\alpha \in L$, let $f_\alpha$ be its minimal polynomial over $K$. Each $f_\alpha$ is irreducible over $K$, so the normality of $L/K$ implies that $f_\alpha$ splits in $L$. Then clearly $L$ is the splitting field of $\{ f_\alpha \text{ over } K : \alpha \in L \}$.    
    

    (ii) $\Rightarrow$ (iii). Let $\sigma$ be a $K$-monomorphism from $L$ to $\overline K$. $\sigma$ can be considered as an isomorphism from $L$ to $\sigma(L)$ which fixes all elements in $K$. Since $L$ is the splitting field for a polynomial $f$ over $K$, $f$ have coefficients in $K$ and thus is fixed under $\sigma$. Hence $\sigma(L) = L$ and $\sigma$ is a $K$-automorphism of $L$. 
    
    (iii) $\Rightarrow$ (i). Let $f$ be an irreducible polynomial over $K$ with a root $\alpha$ and let $\alpha \in L$. Let $F$ be the splitting field of $f$ over $K$. For any root $\beta \in F$ of $f$, there exists a $K$-automorphism $\sigma$ of $F $ such that $\sigma(\alpha) = \beta$ by Theorem \ref{thm:automorphism-from-zeros}. Then $\sigma$ can also be considered as a $K$-monomorphism $L \to \overline K$. By assumption, $\sigma|_L(L) = L$. With $\alpha \in L$ we deduce that $\beta \in L$. Hence $L/K$ is normal.
\end{proof}

As discussed earlier, the normality of a field extension is naturally linked to splitting fields, and (ii) in the above theorem formalizes this idea. (iii) mainly builds on the idea of Theorem \ref{thm:automorphism-from-zeros}, in that any pair of roots of the irreducible $f$ correspond to a $K$-automorphism of the splitting field of $f$. 

\begin{example}
	Let $K = \Q (\sqrt 2)$. Then $K$ is the splitting field for $f(t) = t^2 - 2$ over $\Q$, and thus $K/ \Q$ is normal. We also know that any $\Q$-monomorphism $\sigma: K \to \mathbb A$ is a $\Q$-automorphism of $K$. That is, $\sigma$ can only send $\sqrt 2$ to $\pm \sqrt 2$. 
\end{example}

\begin{example}
	Let $K = \Q(\sqrt[3]{2})$. We have seen that $K / \Q$ is not normal. Indeed, define a $\Q$-monomorphism $\sigma : K \to \mathbb A$ by $\sigma(\sqrt[3]{2}) =  \sqrt[3]{2}\zeta $ where $\zeta = e^{2 \pi i / 3}$. Then $\sigma$ is not a $\Q$-automorphism of $K$. 
\end{example}



Galois did not specifically acknowledge the idea of separability, as he solely worked with subfields of $\C$, where separability is inherently present, as we will discover. Following the approach in \cite{Stewart}, we shall introduce the separability of a field extension in a ``recursive'' manner: we will introduce separable \textit{polynomials} over a field, separable \textit{elements} in a field and separable \textit{field extensions}, one after another. 

\begin{definition}
    Let $f$ be an irreducible polynomial over field $K$. Then $f$ is \textit{separable} over $K$ if $f$ takes the form 
    $
        f(t) = k \prod_{i = 1} ^ n(t - \sigma_i)
    $
    where $\sigma_i \in \overline K$ are distinct and $k \in K$.
    Let $L/K$ be a field extension. An element $\alpha \in L$ is \textit{separable} if the minimal polynomial of $\alpha$ over $K$ is separable over $K$.
    A field extension $L / K$ is \textit{separable} if every $\alpha \in L$ is separable.
\end{definition}

The following theorem of a computational nature is useful to determine whether a polynomial is separable and is given without proof \cite{Stewart}. 

\begin{theorem} \label{thm:separable-derivative}
    A polynomial $f$ over a subfield $K$ of  $\C$ is separable if and only if it is coprime with $Df$ over $K$, where $Df$ is its derivative. 
\end{theorem}

The next theorem implies that we can take for granted the separability of irreducible polynomials over subfields of $\C$, as well as the separability of any field extension within $\C$.


\begin{theorem} \label{thm:separable-poly-in-C}
    If $K$ is a subfield of $\C$, then every irreducible polynomial over $K$ is separable. 
\end{theorem}

\begin{proof}
    Let $f$ be an irreducible polynomial over $K$. Suppose that $f$ is not separable, then by Theorem \ref{thm:separable-derivative}, $f$ and $Df$ shares a common factor of degree $\ge 1$. Since $f$ is irreducible, this common factor must be $f$. Since $Df$ has a smaller degree than $f$, $Df$ must be $0$. This implies that $f$ is constant.
\end{proof}


\begin{corollary} \label{thm:separable-extension-in-C}
    A field extension $L/K$ such that $K \subseteq L \subseteq \mathbb C$ is always separable. 
\end{corollary}


