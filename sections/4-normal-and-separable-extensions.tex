\section{Normal and Separable Extensions}
In our focus on field extensions, it is crucial to understand the properties of normality and separability as they are the key ingredients for Galois extensions which we will discuss later. Before introducing normal and separable extensions, we first look at splitting fields and algebraic closures, two closely related concepts. 

\subsection{Splitting Fields}
\begin{definition}
    Let $K$ be a subfield of $\C$. A polynomial $f$ over $K$ \textit{splits} in $K$ if it factors into linear factors $$
    f(t) = k \prod _{i=1} ^n (t - \alpha_i),
    $$
    where $k, \alpha_1, \ldots, \alpha_n \in K$. 
\end{definition}
Theorem \ref{thm:fundamental-algebra} indicates that $f$ splits over $K$ if and only if all of its zeros are contained in $K$. 

\begin{definition}
    % A \textbf{\textit{splitting field}} of a polynomial $f$ over a subfield $K$ of $\C$ is a subfield $\Sigma$ of $\C$ such that 
    % \begin{itemize}
    %     \item $K \subseteq \Sigma$,
    %     \item $f$ splits over $\Sigma$ and
    %     \item for any $\Sigma'$ such that $K \subseteq \Sigma' \subseteq \Sigma$ and $f$ splits over $\Sigma'$, we have $\Sigma' = \Sigma$. 
    % \end{itemize}
    The splitting field $\Sigma$ of a polynomial $f$ over a subfield $K$ of $\C$ is the field generated by $K$ and all zeros of $f$. Equivalently, it is the smallest field containing $K$ and all zeros of $f$. 
\end{definition}

\begin{example}
If we have a polynomial \(p(x) = x^4 - 12x^2 + 35\) the splitting field of \(p(x)\) is \(\mathbb{Q}[\sqrt{5},\sqrt{7}]\) as it contains all of the roots of \(p(x)\) and if it was any smaller it would not contain all of the roots or would not be a field.
\end{example}

The splitting fields of a polynomial over two isomorphic fields are isomorphic, in the following sense:

% Theorem 9.6 Stewart
\begin{theorem} \label{thm:splitting-field-unique}
	Let $\iota: K \to K'$ be a field isomorphism. Let $f$ be a polynomial over $f$ with the splitting field $\Sigma$, and then let $\Sigma'$ be the splitting field of $\iota(f)$ over $K'$. Then there is an isomorphism $j : \Sigma \to \Sigma'$ such that $j | _K = \iota$. 
\end{theorem}

\begin{proof}
	\TODO
\end{proof}


\subsection{Algebraic Closures}
We might extend the definition of a splitting field of \textit{a set of polynomials} over a field $K$. The definition would then be the smallest field containing $K$ and all zeros of every polynomial in the set. 
\begin{definition}
    The algebraic closure of a field $K$ is the splitting field of the set of all polynomials over $K$, denoted as $\overline K$. 
\end{definition}

%The existence of $\overline K$ for any field $K$ can be proven but is omitted here. 

\begin{theorem} 
	Let $\iota: K \to K'$ be a field isomorphism. Then there is an isomorphism $j: \overline K \to \overline {K'}$ such that $j |_K = \iota$. 
\end{theorem}

%\begin{definition}
%    A field $K$ is algebraically closed if every polynomial $f$ over $K$ has a root in $K$. 
%\end{definition}
%
%\begin{theorem}
%    For a field $K$, its smallest algebraically closed extension is the algebraic closure of $K$.
%\end{theorem}



\subsection{Normal Extensions}


\begin{definition}
    An algebraic field extension $L/ K$ is \textit{normal} if every irreducible polynomial over $K$ which has a zero in $L$ splits in $L$. 
\end{definition}
\begin{example}
	For any field $K$, $\overline K / K$ is a normal extension, as every polynomial over $K$ splits in $\overline K$. 
\end{example}
\begin{example}
    $\C / \R$ is normal, since every polynomial over $\R$ splits in $\C$ by Theorem \ref{thm:fundamental-algebra}.
\end{example}
\begin{example}
    Let $K = \Q(\sqrt[3]{2})$. Then $K / \Q$ is not normal. The polynomial $f(t) = t^3 - 2$ over $\Q$ is irreducible and has a zero $\sqrt[3]{2}$ in $K$, but $K \subseteq \R$ does not contain the two non-real zeros of $f$. Hence $f$ does not split in $K$.
\end{example}

\begin{theorem} \label{thm:normal-equiv-def}
    Let $L/K$ be an finite field extension. Then the following three statements are equivalent:
    \begin{enumerate}[label=(\roman*)]
        \item $L/K$ is normal;
        \item $L$ is the splitting field for a polynomial $f$ over $K$;
        \item Any $K$-monomorphism $\sigma: L \to \overline K$ is a $K$-automorphism of $L$. 
    \end{enumerate}
\end{theorem}

\begin{proof}
    (i) $\Rightarrow$ (ii). 
    Suppose $L/K$ is normal and finite. The finiteness of $L/K$ implies that $L = K(\alpha_1, \dots, \alpha_r)$ for $r$ finite and $\alpha_i$ algebraic over $K$ by Theorem \ref{thm:finite-equi-def}. For each $\alpha_i$, take the minimal polynomial $f_i$ of $\alpha_i$ over $K$. Each $f_i$ is irreducible over $K$, so the normality of $L/K$ implies that $f_i$ splits in $L$.  Then take $f = f_1 \dots f_r$ and $L$ is the splitting field of $f$.
	
%	    For each $\alpha \in L$, let $f_\alpha$ be its minimal polynomial over $K$. Each $f_\alpha$ is irreducible over $K$, so the normality of $L/K$ implies that $f_\alpha$ splits in $L$. Then clearly $L$ is the splitting field of $\{ f_\alpha \text{ over } K : \alpha \in L \}$.    
    

    (ii) $\Rightarrow$ (iii). Let $\sigma$ be a $K$-monomophism from $L$ to $\overline K$. $\sigma$ can be considered as an isomorphism from $L$ to $\sigma(L)$ which fixes all elements in $K$. Since $L$ is the splitting field for a polynomial $f$ over $K$, $f$ have coefficients in $K$ and thus is fixed under $\sigma$. Hence $\sigma(L) = L$ and $\sigma$ is a $K$-automorphism of $L$. 
\end{proof}

Before we prove (iii) $\Rightarrow$ (i), we pulse and prove the following result, which enables us to ``extend" a $K$-monomorphism $M \to L$ to a $K$-automorphism of $L$ where $K \subseteq M \subseteq L$. 

\begin{theorem} \label{thm:monomorphism-extend-automorphism}
	Let $L/K$ be a normal field extension and let $M$ be an intermediate field such that $K \subseteq M \subseteq L$. Let $\tau$ be a $K$-monomorphism $M \to L$. Then there exists a $K$-automorphism $\sigma$ of $L$ such that $\sigma | _M = \tau$.  
\end{theorem}

\begin{proof}
	By (i) $\Rightarrow$ (ii) of Theorem \ref{thm:normal-equiv-def}, $L$ is the splitting field for a polynomial $f$ over $K$ and thus also over $M$. $\tau$ can be considered as an isomorphism from $M$ to $\tau(M)$ which fixes every element in $K$ and hence $f$. Then $L$ is both the splitting field for $f$ over $M$ and the splitting field for $f$ over $\tau(M)$. Then by Theorem \ref{thm:splitting-field-unique}, there exists an isomorphism $\sigma: L \to L$ such that $\sigma | _M = \tau$. Since $\sigma | _K = \tau |_K = \Id$, $\sigma$ is a $K$-automorphism of $L$. 
\end{proof}

\begin{theorem} \label{thm:automorphism-from-zeros}
    Let $L/K$ be a field extension and let $\alpha, \beta \in L$ be zeros of an irreducible polynomial $f$ over $K$. Then there exists a $K$-automorphism $\sigma$ of $L$ such that $\sigma(\alpha) = \beta$. 
\end{theorem}

\begin{proof}
     $K(\alpha)/K$ and $K(\beta)/K$ are isomorphic extensions by \TODO(Corollary 5.13). The isomorphism from $K(\alpha)$ to $K(\beta)$ is also a $K$-monomorphism from $K(\alpha)$ to $\overline K$ and thus can be extended to a $K$-automorphism $\sigma$ of $\overline K$ by Theorem \ref{thm:monomorphism-extend-automorphism}, where $\sigma(\alpha) = \beta$.
\end{proof}

\begin{proof}[Proof of Theorem \ref{thm:normal-equiv-def}, continued]
	    (iii) $\Rightarrow$ (i). Let $f$ be an irreducible polynomial over $K$ with zero $\alpha \in L$. For any zero $\beta \in \overline K$ of $f$, there exists a $K$-automorphism $\sigma$ of $\overline K $ such that $\sigma(\alpha) = \beta$ by Theorem \ref{thm:automorphism-from-zeros}. By assumption, $\sigma(L) = L$. With $\alpha \in L$ we deduce that $\beta \in L$. Hence $L/K$ is normal.
\end{proof}

%\begin{corollary}
%    $L/K$ is normal and finite if and only if $L$ is the splitting field for a polynomial $f$ over $K$. 
%\end{corollary}
%
%\begin{proof}
%    If $L/K$ is normal and finite, the finiteness of $L/K$ implies that $L = K(\alpha_1, \dots, \alpha_r)$ for $r$ finite and $\alpha_i$ algebraic over $K$ by Theorem \ref{thm:finite-equi-def}. For each $\alpha_i$, take the minimal polynomial $f_i$ of $\alpha$ over $K$. Then similarly as above, take $f = f_1 \dots f_r$ and $L$ is the splitting field of $f$.
%
%    The converse is trivial.
%\end{proof}


\subsection{Separable Extensions}

Galois did not specifically acknowledge the idea of separability, as he solely worked with subfields of $\C$, where separability is inherently present, as we will discover.

\begin{definition}
    Let $f$ be an irreducible polynomial over field $K$. Then $f$ is separable over $K$ if $f$ takes the form 
    $$
        f(t) = k(t - \sigma_1) \dots (t - \sigma_n)
    $$
    where $\sigma_i \in \overline K$ are distinct.
\end{definition}

The following theorem of a computational nature is useful to determine whether a polynomial is separable and is given without proof. 

\begin{theorem} \label{thm:separable-derivative}
    A polynomial $f$ over a subfield $K$ of $\C$ is separable if and only if it is coprime with $Df$ over $K$, where $Df$ is its derivative. 
\end{theorem}

The next theorem implies that we can take for granted the separability of irreducible polynomials over subfields of $\C$. 

\begin{theorem} \label{thm:separable-poly-in-C}
    If $K$ is a subfield of $\C$, then every irreducible polynomial over $K$ is separable. 
\end{theorem}

\begin{proof}
    Let $f$ be an irreducible polynomial over $K$. Suppose that $f$ is not separable, then by Theorem \ref{thm:separable-derivative}, $f$ and $Df$ shares a common factor of degree $\ge 1$. Since $f$ is irreducible, this common factor must be $f$. Since $Df$ has a smaller degree than $f$, $Df$ must be $0$. This implies that $f$ is constant.
\end{proof}

\begin{definition}
    Let $L/K$ be a field extension. An element $\alpha \in L$ is \textit{separable} if the minimal polynomial of $\alpha$ over $K$ is separable over $K$.
\end{definition}

\begin{definition} \label{def:separable-extension}
    A field extension $L / K$ is \textit{separable} if every $\alpha \in L$ is separable.
\end{definition}

\begin{theorem}
    A field extension $L/K$ such that $K \subseteq L \subseteq \mathbb C$ is separable. 
\end{theorem}

\begin{proof}
    This directly follows from Theorem \ref{thm:separable-poly-in-C} and Definition \ref{def:separable-extension}. 
\end{proof}
