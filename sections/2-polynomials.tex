
\section{Polynomials}
\subsection{Basics of Polynomials}

\begin{definition}
    A polynomial over $\C$ in the indeterminate $t$ to be an expression 

$$
    r_0 + r_1 t +...+ r_n t^n
$$

    where $r_0,r_1,...,r_n \in \C$, $0 \le n \in \Z$ and $t$ is undefined.
\end{definition}

\begin{definition}
    If $f$ is a polynomial over $\C$ and $f \neq 0$, then the degree of $f$ is the highest power of $t$ occurring in $f$ with a non-zero coefficient.
\end{definition}

\begin{theorem}
    Two polynomials $f,g$ over $\C$ define the same function if and only if they are equal polynomials, so they have the same coefficients.
\end{theorem}

\begin{proof}
    This is a basic proof to an obvious theorem. By taking the difference of the two polynomials, we must prove that if $f(t)$ is a polynomial over $\C$ and $f(t) = 0$ for all $t$, then the coefficients of $f$ are all 0. Let $P(n)$ be the statement: If a polynomial $f(t)$ over $\C$ has degree $n$, and $f(t) = 0$ for all $t \in \C$, then $f=0$. We prove $P(n)$ for all $n$ by induction on $n$. 
    
    Both $P(0)$ and $P(1)$ are obvious. Suppose that $P(n-1)$ is true. 

    With $f(t) = a_n t^n +...+ a_0$. In particular, $f(0) = 0$, so $a_0 = 0$ and,

    $$
    f(t) = a_n t^n +...+ a_1 t= t(a_n t^{n-1} +...+ a_1) = tg(t)
    $$

    where $g(t) = a_n t^{n-1} +...+ a_1$ has degree $n-1$. Now $g(t)$ vanishes for all $t \in \C$ except $t=0$. However is $g(0) = a_1 \neq 0$ then $g(t) \neq 0$ for sufficiently small $t$. Therefore $g(t)$ vanishes for all $t \in \C$. By induction, $g=0$. Therefore $f=0$ so $P(n)$ is true and the induction is complete
\end{proof}

Polynomials will often be written in descending order $r_n t^n + r_{n-1} t^{n+1} + ... + r_1 t + r_0$ for $n$

Two polynomials are defined to be equal if and only if all corresponding co-efficients are equal.

If $r = \sum (r_i t^i)$ and $s = \sum (s_i t^i)$ then $r+s = \sum (r_i + s_i)t^i$

and $rs = \sum (q_j t^j)$ where $q_j = \sum_{h+i=j} r_h s_i$

It is easy to check directly from the definitions above that the set of all polynomials over $\C$ in the $t$ obeys all the normal algebraic laws. We denoted the set by $\C[t]$and call it the ring of polynomials over $\C$ in the indeterminate $t$.

\subsection{Fundamental Theorem of Algebra} 

After discovering that equations can be solved over $\C$, the there is a question of why stop at $\C$. Why not try and find an equation with no solutions over $\C$? The answer is that no such polynomial equations exist as every polynomial over $\C$ has a solution over $\C$.

This theorem from the module \textit{Complex Functions} is stated without proof. 

\begin{theorem}[Fundamental Theorem of Algebra] \label{thm:fundamental-algebra}
    If $p(z)$ is a non-constant polynomial over $\mathbb{C}$, then there exists $z_0 \in \mathbb{C}$ such that $p\left(z_0\right)=0$.
\end{theorem}

Such a number $z$ is called a root of the equation $p(t)=0$, or a zero of the polynomial $p$. For example, $\mathrm{i}$ is a root of the equation $t^2+1=0$ and a zero of $t^2+1$. Polynomial equations may have more than one root; indeed, $t^2+1=0$ has at least one other root, $-i$ but others will have more roots. We do the proof above by starting out with a contradiction.

\begin{proposition}
    Let $p(t) \in \mathbb{C}[t]$ with $\partial p=n \geq 1$. Then there exist $\alpha_1, \ldots, \alpha_n \in \mathbb{C}$, and $0 \neq k \in \mathbb{C}$, such that
    $$
    p(t)=k\left(t-\alpha_1\right) \ldots\left(t-\alpha_n\right).
    $$
\end{proposition}

\begin{proof}
    Use induction on $n$. When $n = 1$ then it is obvious that this proposition follows. If $n > 1$ then we know by the Fundamental Theorem of Algebra, that $p(t)$ has at least one zero $\alpha_n$. By the Remainder Theorem above, there exists $q(t) \in \C[t]$ such that,

\begin{equation} \label{eq:2001}
    p(t) = (t-\alpha_n) q(t)
\end{equation}
    
    Then $\partial q = n - 1$, so by induction,

\begin{equation} \label{eq:2002}
    q(t) = k(t-\alpha_a)...(t-\alpha_{n-1})
\end{equation}     
    
    For suitable complex numbers $k,\alpha_1,...,\alpha_{n-1}$. Substitute \ref{eq:2002} into \ref{eq:2001} and the induction step is complete
\end{proof}

\subsection{Factorisation of Polynomials}
\begin{definition}
     A non-constant polynomial over a subring $R$ of $\mathbb{C}$ is \textit{reducible} if it is a product of two polynomials over $R$ of smaller degree. Otherwise it is \textit{irreducible}.
\end{definition}

\begin{theorem}
    Any non-zero polynomial over a subring $R$ of $\mathbb{C}$ is a product of irreducible polynomials over $R$.
\end{theorem}

\begin{proof}
    Let $g$ be any non-zero polynomial over $R$. We proceed by in introduction on the degree of $g$. If $\partial g = 0$ or 1 then $g$ is automatically irreducible. If $\partial g > 1$, then either $g$ is irreducible or $g = hk$ where $\partial h$, $\partial k < \partial g$. By induction, $h$ and $k$ are products of irreducible polynomials, where $g$ is such a product. The theorem then follows by induction.
\end{proof}

\begin{example}
    We can use the above theorem to prove irreducibilty for some cases, working very well for cubic polynomials over $\Z$. Let $R = \Z$ then the polynomial $f(t) = t^3 -5 t + 1$ is irreducible. If it was not irreducible then it must have a linear factor $t - \alpha$ over $\Z$ and then $\alpha \in \Z$ and $f(\alpha) = 0$. There also must exist $\beta, \lambda \in \Z$ such that,

    $$
    f(t) = (t-\alpha)(t^2 + \beta t + \lambda) = t^3 + (\beta - \alpha)t^2 + (\lambda - \alpha \beta)t - \alpha \lambda
    $$

    so $\alpha \lambda = -1$. Therefore, $\alpha = +/- 1$. But $f(1) = -3 \neq 0$ and $f(-1) = 5 \neq 0$. This shows that no factor exists.
\end{example}

\begin{theorem}
    Gauss Lemma - Let f be a polynomial over $\Z$ that is irreducible over $\Z$. Then $f$, considered as a polynomial over $\Q$, is also irreducible over $\Q$.
\end{theorem}

\begin{proof}
This lemma is useful because when we extend the subring of coefficients from $\Z$ to $\Q$ then there are new polynomials which may be factors of $f$. However, we now show that they are not. Starting with a contradiction, suppose that $f$ is irreducible over $\Z$ but reducible over $\Q$ so that $f = g h$ where $g$,$h$ are both polynomials over $\Q$, of a smaller degree. Multiplying through by the product of the denominators of the coefficients of $g$ and $h$ means we can rewrite this as $n f = g' h'$, where $n \in \Z$ and $g'$ and $h'$ are polynomials over $\Z$. We can now show that we can cancel out the prime factors of n one by one, without going outside $\Z[t]$

Suppose that $p$ is a prime factor of $n$. We now claim that if $g' = g_0 + g_1 t +...+ g_r t^r$, $h' = h_0 + h_1 t +...+ h_s t^s$ then either $p$ divides all the coefficients of $g_i$, or $p$ divides all the coefficients $h_j$.

If not then there must be smallest values $i$, $j$ such that $p \nmid g_i$ and $p \nmid h_j$. However, $p$ divides the coefficient of $t^{i+j}$ in $g' h'$, which is, 

$$
h_0 g_{i+j} + h_1 g_{i+j-1} +...+ h_j g_i +...+ h_{i+j} g_0
$$

and by the choice of $i$ and $j$, the prime $p$ divides every term of this expression except $h_j g_i$. $p$ divides the whole expression, so $p | h_j g_i$. However $p \nmid g_i$ and $p \nmid h_j$, a contradiction.
\end{proof}

\begin{theorem}
    Eisenstein's Criterion - Let
    
    $$f(x) = a_0 + a_1 t + ... + a_n t^n$$
    
    be a polynomial over $\Z$. Suppose there is a prime $q$ such that:

    (1) $q \nmid a_n$

    (2) $q | a_i$ for $i = {0, 1,..., n-1}$
    
    (3) $q^2 \nmid a_0$
    
    Then $f$ is irreducible over $\Q$
\end{theorem}

\begin{proof}
By Gauss's Lemma it is sufficient to show that $f$ is irreducible over $\Z$. Suppose for a contradiction that $f = gh$, where

$$
g=b_0+b_1 t+ ... +b_r t^r,
$$
$$
h=c_0+c_1 t+ ... +c_s t^s
$$

are polynomials of smaller degree over $\Z$. Then $r \ge 1, s \ge 1$ and $r+s = n$. Now $b_0 c_0 = a_0$ so by (2), $q | b_0$ or $q|c_0$. By (3), $q$ cannot divide both $b_0$ and $c_0$, so without
loss of generality we can assume $q | b_0$ and $q \nmid c_0$. If all $b_j$ are divisible by $q$, then $a_n$ is divisible by $q$, contrary to (1). Let $b_j$ be the first coefficient of $g$ not divisible by $q$. Then

$$
a_j = b_j c_0 + ...+ b_0 c_j
$$
where $j < n$.

 This implies that $q$ divides $c_0$, since $q$ divides $a_j, b_0,..., b_{j-1}$, but not $b_j$. This is a contradiction. Hence f is irreducible.
\end{proof}

\begin{example}
Consider

$$
f(t) = \frac{2}{9} t^5 + \frac{5}{3} t^4 + t^3 
$$

over $\Q$.

This is irreducible over $\Q$ if and only if

$$
9f(t) = 2t^5 + 15t^4 + 9t^3
$$

is irreducible over $\Q$. Eisenstein's criterion now applies with $q = 3$, showing that $f$ is irreducible.
\end{example}

\begin{theorem}
If $p$ is prime, the binomial coefficient $\binom{p}{r}$ is divisible by $p$ if $1 \le r \le p-1$
\end{theorem}

\begin{proof}
The binomial coefficient is an integer and $\binom{p}{r} = \frac{p!}{r!(p-r)!}$

The factor $p$ in the numerator cannot cancel with any factor in the denominator unless $r=0$ or $r=p$.
\end{proof}

Then we have,

\begin{theorem}
    If $p$ is a prime then the polynomial

    $$
    f(t) = 1 + t + ... = t^{p-1}
    $$

    is irreducible over $\Q$
\end{theorem}

\begin{proof}
First, $f(t) = \frac{t^p - 1}{t - 1}$. Put $t = 1 + u$ where $u$ is a new indeterminate. Then $f(t)$ is irreducible over $\Q$ if and only if $f(1+u)$ is irreducible. But

$$
f(1+u) = \frac{(1+u)^p - 1}{u}
$$

$$
f(1+u) = u^{p-1} + ph(u)
$$

where $h$ is a polynomial in $u$ over $\Z$ with constant term 1, by the above theorem. By Eisenstein's Criterion, $f(1+u)$ is irreducible over $\Q$.
\end{proof}

\begin{definition}
    A \textbf{symmetric polynomial} is a polynomial where if the variables are interchanged, the polynomial remains the same.

    Formally, for $f(x_1,x_2,...,x_n)$ and $\forall \sigma \in S_n$, we have $f(x_1,x_2,...,x_n = f(x_{\sigma(1)},x_{\sigma(2)},...,x_{\sigma(n)}$
\end{definition}

\begin{example}
    Consider the polynomial 
    $f(x,y)=x^2+y^2-4$, we can see that $f(y,x)=y^2+x^2-4$ and so $f(x,y)=f(y,x)$ and so $f(x,y)$ is a symmetric polynomial.
\end{example}

\begin{definition}
    The \textbf{elementary symmetric polynomials} are defined by 
    \begin{align*}
    e_k(x_1,x_2,...,x_n) = \sum_{1\leq j_1<j_2<...<j_k\leq n} X_{j_1}\cdot X_{j_2} \cdot ... \cdot X_{j_k}
    \end{align*}
\end{definition}

\begin{example}
    For example, in the case where $n=4$, we can see the following are elementary symmetric polynomials:
    \\For $k=1$: $e_1(x_1,x_2,x_3,x_4) = x_1 + x_2 + x_3 + x_4$
    \\For $k=3$: $e_3(x_1,x_2,x_3,x_4) = x_1x_2x_3+x_1x_2x_4+x_1x_3x_4+x_2x_3x_4$
\end{example}