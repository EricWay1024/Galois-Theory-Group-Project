\section{Conclusion}
To summarise, we have developed the basics behind field theory, and introduced key concepts like field extensions. We then categorised these field extensions, which then helped us to develop the idea behind Galois groups. We then introduce the Fundamental Theorem of Galois Theory in order to help us simplify complex problems in field theory, into simpler to solve problems in group theory, and then used this to solve problems such as the Abel-Ruffini Theorem.

In further study, we would (i) further develop Galois Theory on finite fields; (ii) look more in depth at the 'Inverse Galois Problem' which attempts to identify a field extension which could have a given Galois group. Though it has been shown that all finite groups occur as Galois groups, it is still an open question as to whether every finite Galois group is a possible Galois group of an extension of the rational numbers $\Q$; and (iii) look  at the real-world applications of Galois Theory, such as its use in cryptography and more specifically elliptic curve cryptography.

