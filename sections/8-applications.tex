\section{Applications} \label{sec:applications}

\subsection{Constructions}
When constructing shapes, the Greeks, specifically Plato, looked at 'perfect' geometric figures - those which can be constructed with only straight lines and a compass. This is to see if certain shapes and figures can be drawn without approximations. The following section is adapted from Stewart, Mathigon and...

Euclid set out five main axioms to drawing these figures:

\begin{enumerate}
    \item You can draw a straight line segment between any two points.
    \item You can extend any line segment infinitely.
    \item Given a point and a distance $r$, you can construct a circle centered at that point, with radius $r$.
    \item Any two right angles are congruent.
    \item Given a line $L$ and a point $P$, one can construct a line through $P$ that is parallel to $L$. 
\end{enumerate}

Now we can begin a formal definition, so now we construct the plane $\R^2$ which is isomorphic to $\C$. 
\begin{definition}
Let points be denoted by $z_1,z_2 \in \C$ and we let $0\leq r \in \R$.
    \begin{itemize}
        \item $L(z_1,z_2)$ is the line joining $z_1$ to $z_2$ given that $z_1 \neq z_2$. (In this report, we extend the line infinitely in both directions.)
        \item $C(z_1,r)$ is the circle around $z_1$ with radius $r$.
    \end{itemize}
    For the following we take $n \in \N$, and define $\mathcal{P}_n,\mathcal{L}_n,\mathcal{C}_n$ as the sets of n-constructable points, lines and circles respectively.
    \begin{itemize}
        \item $\mathcal{P}_0=\{0,1\}$; $\mathcal{P}_{n+1}=\{z\in \C $: $z$ lies on two distinct lines in $\mathcal{L}_{n+1}$ or it lies on one line in $\mathcal{L}_{n+1}$ and one circle in $\mathcal{C}_{n+1}$ or $z$ lies on two distinct circles in $\mathcal{C}_{n+1}$\}
        \item $\mathcal{L}_0=\varnothing$ ; $\mathcal{L}_{n+1}=\{L(z_1,z_2) : z_1,z_2\in \mathcal{P}_n\}$
        \item $\mathcal{C}_0=\varnothing$ ; $\mathcal{C}_{n+1}=\{C(z_1,|z_2-z_3|:z_1,z_2,z_3\in \mathcal{P}_n\}$
    \end{itemize}
\end{definition}

\begin{definition}\label{def:constructable-point}
    A point, $z \in \C$ is called constructable if there is a finite sequence of points:
    $z_0=0,z_1=1,z_2,z_3,...,z_k=z$, where $z_{j+1}$ lies in one of the following:
    \begin{itemize}
        \item On the intersection of two straight lines constructed between two sets of two points, i.e:
        $z_{j+1}\in L(z_{j_1},z_{j_2})\cap L(z_{j_3},z_{j_4})$.
        \item On the intersection of a circle and a line, i.e:
        $z_{j+1}\in L(z_{j_1},z_{j_2})\cap C(z_{j_3},|z_{j_4}-z_{j_5})|$.
        \item On the intersection of two circles, i.e:
        $z_{j+1}\in C(z_{j_1},|z_{j_2}-z_{j_3})|\cap C(z_{j_4},|z_{j_5}-z_{j_6})|$. 
    \end{itemize}
    Where $j_i\in \N$.
\end{definition}

\begin{lemma}\label{thm:point-subset}
    $\forall n \in \N$ we have that:
    $\mathcal{P}_n \subseteq \mathcal{P}_{n+1}$
\end{lemma}

\begin{proof}
    The proof is trivial, clearly all the points which lie in  $\mathcal{P}_n$ have to lie in $\mathcal{P}_{n+1}$.
\end{proof}

\begin{theorem}
    A point is called constructable if and only if $z\in \mathcal{P}_n$ for some $n\in \N$
\end{theorem}

\begin{proof}  
    $"\implies"$ Let $z \in \C$ be a constructable point, then by Definition \ref{def:constructable-point}, it is clear that $z=z_k\in \mathcal{P}_k$.
    $"\impliedby"$ On the other hand, if $z \in \mathcal{P}_k$, then we can find a sequence $z_j \in \mathcal{P}_j$ for $0\leq j \leq k$ which satisfies Definition \ref{def:constructable-point}.
\end{proof}

We now define the 'Pythagorean Closure' which will help us create a link between construction and Galois Theory.

\begin{definition}[Pythagorian Closure]
    $\Q^{py}$ is called the 'Pythagorean Closure' of $\Q$, which is the union of all subfields, denoted $K$ where:
    $\Q = K_0 \subseteq K_1 \subseteq ... \subseteq K_n=K$ such that $[K_{j+1}:K_j]\leq 2$
\end{definition}

Here we can see a tower of field extensions, which gives us a hint that we may be able to use fields and extensions to solve construction problems.

The next Lemma will help us to prove a Theorem which will show the link between the two.

\begin{lemma}\label{lemma:conjugate-in-Pn}
    Let $z \in \mathcal{P}_n$, then its complex conjugate $\overline{z} \in \mathcal{P}_{n+1}$.
\end{lemma}

\begin{proof}  
    A visual proof is also included in the appendix.
    
    Let us show that we can construct the complex conjugate using the points $z\in \mathcal{P}_n$, 0 and 1.
    We start with the trivial case where either $z=1$ or $\overline{z}=1$. Here the point has already been constructed, and since $\overline{z}=1\in \mathcal{P}_0$ this implies $\overline{z}=1\in \mathcal{P}_n$ by Lemma \ref{thm:point-subset}.
\newline
\noindent
    Now, let us say $z\in \mathcal{P}_n$ with $z\neq1$. Then we consider the circle $C(z,|z-1|)$, and the line $L(0,1)$. We have two cases here, either the circle meets the line at exactly one point, $1$, or the circle meets the line in exactly two points, $1$ and $s$.

    \textbf{Case One:} If $C(z,|z-1|)$ and $L(0,1)$ meet at exactly one point, then the line $L(1,z)$ must be perpendicular to $L(0,1)$. Then by drawing the circle $C(1,|z-1|)$, we see that it meets the line $L(1,z)$ at two points, one is $z$ and the other is $\overline{z}$.

    \textbf{Case Two:} If $C(z,|z-1|)$ and $L(0,1)$ meet at two points, then one of them must be $1$ and we shall call the other $z_1\neq1$. Then construct two more circles $C_1=C(1,|z-1|)$ and $C_2=C(z_1,|z-1|)$. Then $C_1$ and $C_2$ have two points of intersection, one being $z$ and the other $\overline{z}$.

    Thus we have all points lie in $\mathcal{P}_n$ and thus $\overline{z} \in \mathcal{P}_{n+1}$.
\end{proof}

Now we have the key results to help us prove the following theorem, where the proof here is adapated from Everitt \cite{constructions-and-galois}.

\begin{theorem}
    A point $z \in \C$ is constructable if and only if $z \in \Q^{py}$.
\end{theorem}

\begin{proof}
    "$\implies$"By Definition \ref{def:constructable-point}, we have a condition for when $z$ is constructable. Now, let $K_j$ be the field $\Q(z_1,..,z_j)$, where the $z_i$ come from the sequence as defined in the definition. Thus we get a tower of extensions $\Q \subseteq K_1 \subseteq ... \subseteq K_n$.
    As we showed above, if $z\in \mathcal{P}_n$, then so is $\overline{z}$. Thus if $z\in K_j$, then so is $\overline{z}$, and this means that $K_j$ is close under conjugation.

    \textcolor{red}{To Do:} Finish
\end{proof}

\begin{theorem}\label{thm:power-of-two-construction}
    If $z$ is constructable, then $[\Q(z):\Q]$ is a power of two.
\end{theorem}

\begin{proof}  
    Consider we already have a set of constructable points $X_k = \{0,1,z_1,...,z_k\}$, and we have another point, $z_{k+1}$ which can be constructed through the intersections of lines and circles of these points. Then, algebraically, we have a linear or quadratic equation, where $z_{k+1}$ is a solution. This is due to the fact that a line has the equation $y=mz+c$ and a circle has equation $(x-a)^2+(y-b)^2=r^2$, where $z=x+iy$ and $r=x^2+y^2$. Thus, the new point extends the field by order one (if linear) or two (if quadratic), so $[\Q(z_{k+1}):\Q(X_k)] \in \{1,2\} \forall k \in \N$. Then, since $X_0=\{0,1\}$, it is clear that this will also have order 2, thus every $[\Q(z):\Q]$ is a power of 2.

\end{proof}

\subsection{Squaring the Circle and Other Impossibility Proofs.}

There are many shapes and objects that are not possible to construct using just a ruler and compass. One of the most famous examples of this, is the construction of a square with the same area of a circle. In this section, we look at such impossible constructions, and show why this is impossible.

\begin{theorem}
    The regular heptagon cannot be constructed with ruler and compass.
\end{theorem}

\begin{proof}
    If the regular heptagon were to be constructable, then this means the point $z=e^{2i\pi/7} \in \Q^{py}$. This one one of our complex roots of unity, and this means that $z^7+z^6+z^5+...+z=0$. Since $z$ it is a root of unity, $z^7=1 \implies z^7 - 1 = 0$. We can factorise this as $z^7-1=(z-1)(z^6+z^5+...+z+1)$. Then by Theorem \ref{thm:irreducible-prime-polynomial}, we know that since 7 is prime, $z^6+z^5+...+z+1$ is irreducible over $\Q$ and it has degree 6, which is not a power of two, so a heptagon is not constructable.
\end{proof}

We now look at a proof of a theorem which looks at doubling the volume of a cube using only a compass and a straight edge.

\begin{theorem}
    The cube cannot be doubled by ruler-and-compass construction.
\end{theorem}

\begin{proof}  
    We can first see that to double the volume of a cube, we would need to increase the edge length by a scale factor of $\sqrt[3]{2}$. Thus we can formulate this problem to see if constructing $\sqrt[3]{2}$ is possible, that is to say $z = \sqrt[3]{2} \in \Q^{py}$.
\noindent
\newline
    So let us assume towards a contradiction that $\sqrt[3]{2} \in \Q^{py}$ with $\mu$ being its minimal polynomial over $\Q$. Then, by theorem \ref{thm:power-of-two-construction}, we have that the degree of $\mu = 2^k$ for some $k\in \N \cup \{0\}$.
\noindent
\newline
    Then clearly $z^3=2$, so $z^3-2=0$ we must have that $\mu$ divides $z^3-2$, which is irreducible over $\Q$, thus the field extension $\Q(\sqrt[3]{2}):\Q$ has order 3, which is not a power of two, which contradicts theorem \ref{thm:power-of-two-construction}, and thus $z = \sqrt[3]{2} \notin \Q^{py}$.

\end{proof}


'Squaring the circle' is a problem posed by the Greeks which asked if it was possible to construct a square with the same area as a circle, using only a compass and a straight edge. The following theorem shows the solution to the posed problem.

\begin{theorem}
    The circle cannot be squared using a ruler-and-compass construction.
\end{theorem}

\begin{proof}  
    
    It is sufficient to show that $\sqrt{\pi}\in \C$ is constructable from our initial set of points $\mathcal{P}_0$. From this we could construct $\pi$. Thus, we would have that $[\Q(\pi):\Q]$ is a power of 2, which would imply that $\pi$ is algebraic over $\Q$. However, as seen by Lindemann's proof (which we include in the Appendix), $\pi$ is not algebraic over $\Q$ and thus it is impossible to construct a square with the same area as a circle, using nothing but a compass and a straight edge.
\end{proof}
