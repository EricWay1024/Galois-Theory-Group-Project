\section{Applications} \label{sec:applications}

\subsection{Constructions}
When constructing shapes, the Greeks, specifically Plato, looked at 'perfect' geometric figures - those which can be constructed with only straight lines and a compass. This is to see if certain shapes and figures can be drawn without approximations. The following section is adapted from Stewart, Mathigon and...

Euclid set out five main axioms to drawing these figures:

\begin{enumerate}
    \item You can draw a straight line segment between any two points.
    \item You can extend any line segment infinitely.
    \item Given a point and a distance $r$, you can construct a circle centered at that point, with radius $r$.
    \item Any two right angles are congruent.
    \item Given a line $L$ and a point $P$, one can construct a line through $P$ that is parallel to $L$. 
\end{enumerate}

Now we can begin a formal definition, so now we construct the plane $\R^2$ which is isomorphic to $\C$. 
\begin{definition}
Let points be denoted by $z_1,z_2 \in \C$ and we let $0\leq r \in \R$.
    \begin{itemize}
        \item $L(z_1,z_2)$ is the line joining $z_1$ to $z_2$ given that $z_1 \neq z_2$.
        \item $C(z_1,r)$ is the circle around $z_1$ with radius $r$.
    \end{itemize}
    For the following we take $n \in \N$, and define $\mathcal{P}_n,\mathcal{L}_n,\mathcal{C}_n$ as the sets of n-constructable points, lines and circles respectively.
    \begin{itemize}
        \item $\mathcal{P}_0=\{0,1\}$
        \item $\mathcal{L}_0=\varnothing$
        \item $\mathcal{C}_0=\varnothing$
        \item $\mathcal{P}_{n+1}=\{z\in \C $: $z$ lies on two distinct lines in $\mathcal{L}_{n+1}$ or it lies on one line in $\mathcal{L}_{n+1}$ and one circle in $\mathcal{C}_{n+1}$ or $z$ lies on two distinct circles in $\mathcal{C}_{n+1}$\}
        \item $\mathcal{L}_{n+1}=\{L(z_1,z_2) : z_1,z_2\in \mathcal{P}_n\}$
        \item $\mathcal{C}_{n+1}=\{C(z_1,|z_2-z_3|:z_1,z_2,z_3\in \mathcal{P}_n\}$
    \end{itemize}
\end{definition}

\begin{definition}
    \textcolor{red}{To Do:} Definition 7.4
\end{definition}

\begin{theorem}
    \textcolor{red}{To Do:} Theorem 7.5
\end{theorem}

\begin{proof}  
    \textcolor{red}{To Do:} Prove
\end{proof}

\begin{definition}
    \textcolor{red}{To Do:} Definition 7.6
\end{definition}

\begin{lemma}
    \textcolor{red}{To Do:} Lemma 7.10
\end{lemma}

\begin{proof}  
    \textcolor{red}{To Do:} Prove
\end{proof}

\begin{lemma}
    \textcolor{red}{To Do:} Lemma 7.11
\end{lemma}

\begin{proof}  
    \textcolor{red}{To Do:} Prove
\end{proof}

\subsection{Squaring the Circle and Other Impossibility Proofs.}

There are many shapes and objects that are not possible to construct using just a ruler and compass. One of the most famous examples of this, is the construction of a square with the same area of a circle. In this section, we look at such impossible constructions, and show why this is impossible.

\begin{theorem}
    The regular heptagon cannot be constructed with ruler and compass.
\end{theorem}

\begin{proof}
    \textcolor{red}{To Do:} Prove
\end{proof}

\begin{theorem}
    The cube cannot be duplicated by ruler-and-compass construction.
\end{theorem}

\begin{proof}  
    \textcolor{red}{To Do:} Prove
\end{proof}

The following theorem requires knowledge that $\pi$ is not algebraic over $\Q$. The proof of this comes from Lindemann and is included in the appendix \textcolor{red}{To Do: Reference}

\begin{theorem}
    The circle cannot be squared using a ruler-and-compass construction.
\end{theorem}

\begin{proof}  
    \textcolor{red}{To Do:} Prove
\end{proof}

\subsection{p-adic Numbers}