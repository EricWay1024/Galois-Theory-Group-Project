\section{Galois Groups and Polynomials} \label{sec:galois-groups-and-polynomials}


\subsection{Soluble and Simple Groups}

\begin{definition} \label{def:soluble}
    A group $G$ is soluble (or solvable) if it has a finite series of subgroups 
    $$ \{ e \} = G_0 \triangleleft G_1 \triangleleft \dots \triangleleft G_n = G$$
    such that the quotient $G_{i+1} / G_{i}$ is abelian for each $i = 0, 1, ...,  n - 1$.
\end{definition}

\begin{observation}
    An abelian group is soluble. 
\end{observation}

The main properties of soluble groups are as follows. The proof can be seen in Theorem \ref{thm:soluble-main-appendix}.

\begin{theorem} \label{thm:soluble-main}
    Let $G$ be a group, $H \le G$ and $N \trianglelefteq G$. Then 
    \begin{enumerate}
        \item If $G$ is soluble, then $H$ is soluble;
        \item If $G$ is soluble, then $G / N$ is soluble; 
        \item If $N$ and $G / N$ are soluble, then $G$ is soluble. 
    \end{enumerate}
\end{theorem}

A particular result about soluble groups is that symmetric groups are in general not soluble. The proof can be seen in Theorem \ref{thm:symmetric-not-soluble-appendix}. 

\begin{theorem} \label{thm:symmetric-not-soluble}
	The symmetric group $S_n$ is not soluble when $n \ge 5$. 
\end{theorem}


\subsection{Roots of Unity}


\begin{definition}
	Let $n$ be a positive integer. An $n$-th root of unity $\omega$ in $\mathbb C$ is such that $\omega ^ n = 1$. An $n$-th root of unity $\omega$ is primitive if for any $m = 1, 2, \dots, n - 1$, $\omega ^ m \neq 1$.
\end{definition}

\begin{observation}
	Let $n$ be a positive integer. All $n$-th roots of unity form a cyclic group under multiplication, and the group is generated by any primitive $n$-th root of unity. For a prime number $p$, let $\omega$ be a $p$-th root of unity and $\omega \neq 1$. Then $\omega$ is a primitive $p$-th root of unity and generates the multiplicative group of all $p$-th roots of unity.
\end{observation}

\begin{theorem} \label{thm:unity-1}
	Let $n$ be a positive integer and let $\omega$ be a primitive $n$-th root of unity. Let $L$ be a subfield of $\mathbb C$. If $\omega \in L$, then the polynomial $t^n - 1$ splits in $L$.
\end{theorem}
\begin{proof}
	$\omega$ has order $n$ in the multiplicative group of $L$, so the elements $1, \omega, \omega^2, \ldots, \omega^{n-1}$ are distinct $n$-th roots of unity in $L$. Therefore $t^n-1$ splits in $L$.
\end{proof}

\begin{theorem} \label{thm:unity-2}
	Let $n$ be a positive integer and let $\omega$ be a primitive $n$-th root of unity. If $L$ is the splitting field for $t^n - 1$ over a subfield $K$ of $\mathbb C$, then $L = K(\omega)$.
\end{theorem}

\begin{proof}
	The derivative of $t^n-1$ is $n t^{n-1}$, which is prime to $t^n-1$, so the polynomial $t^n-1$ has no multiple zeros in $L$ by Theorem \ref{thm:separable-derivative}. The group of its zeros under multiplication thus has order $n$ and is cyclic. Let $\omega$ be a generator of this group, and thus $\omega$ is a primitive $n$-th root of unity. Then $L=K(\omega)$. 
\end{proof}




\begin{theorem} \label{thm:radical-1}
	Let $p$ be a prime and let $\omega$ be a primitive $p$-th root of unity. Let $K$ be a subfield of $\mathbb C$. Then $\Gal(K(\omega) / K)$ is abelian.
\end{theorem}
\begin{proof}
	Let $L = K(\omega)$.  Let $\alpha \in \Gal(L / K)$. Then $\alpha$ is uniquely determined by $\alpha(\omega)$ and $\alpha$ is a permutation of the multiplicative group generated by $\omega$. Thus $\alpha$ has the form
	$$
	\alpha_j: \omega \mapsto \omega^j,
	$$
	where $j=1,\dots,p-1$. Then $\alpha_i \alpha_j (\omega) = \alpha_j \alpha_i (\omega) = \omega^{i j}$, so $ \Gal(L / K)$ is abelian.
	
\end{proof}

\begin{theorem} \label{thm:radical-2}
	Let $K$ be a subfield of $\mathbb{C}$ which contains an $n$-th primitive root of unity. Let $\beta^n = c \in K $ where $n$ is a positive integer. Then $t^n - c$ splits in $K(\beta)$ and $\Gal(K(\beta) / K)$ is abelian.
\end{theorem}

\begin{proof}
	By Theorem \ref{thm:unity-1}, $K$ contains all $n$-th roots of unity. Let $L = K(\beta)$, and $\beta$ is a zero of $t^n-c$ in $L$. Any zero of $t^n-c$ in $L$ can be represented by $\omega \beta$, where $\omega$ is some $n$-th root of unity in $K$. Thus $t^n - c$ splits in $L$.  
	
	Let $\phi \in \Gal(L / K)$, then $\phi$ is uniquely determined by $\phi(\beta)$ and $\phi$ is a permutation of the zeros of $t^n - c$. Let $\phi_1, \phi_2 \in \Gal(L / K)$, and let $\phi_1(\beta) = \omega_1\beta$, $\phi_2(\beta) = \omega_2\beta$, where $\omega_1, \omega_2$ are $n$-th roots of unity in $K$. Then
	$$
	\phi_1 \phi_2(\beta)=\omega_1 \omega_2 \beta=\omega_2 \omega_1  \beta=\phi_2 \phi_1(\beta).
	$$
	Thus $\Gal(L / K)$ is abelian.
\end{proof}

\begin{theorem} \label{thm:unity-3}
	Let $K$ be a subfield of $\mathbb C$. Let $\alpha \notin K$ and $\alpha^p = c \in K$ where $p$ is a prime. Let $L / K$ be a normal extension such that $\alpha \in L$. Then $L$ contains a primitive $p$-th root of unity.
\end{theorem}

\begin{proof}
	Let $f$ be the minimal polynomial of $\alpha$ over $K$. Then $f$ splits in $L$ and has no repeated zeros. Since $\alpha \notin K$, we have $\partial f \ge 2$, and therefore there exists $\beta \in L$ such that $\beta$ is a zero of $f$ and $\beta \neq \alpha$. Since $\alpha ^ p - c = 0$, $f$ must divide $t ^ p - c$. Thus $\beta^p - c = 0$. Let $\omega=\alpha / \beta \in L $, then $\omega^p=1$ and $\omega \neq 1$.
\end{proof}





\subsection{Solubility by Radicals}
\begin{definition} \label{def:radical-extension}
    A field extension $L / K$ is \textit{radical} if $L=K\left(\alpha_1, \ldots, \alpha_m\right)$, where for each $i=1, \ldots, m$ there exists $n_i$ such that
$$
\alpha_i^{n_i} \in K\left(\alpha_1, \ldots, \alpha_{i-1}\right).
$$
The elements $\alpha_i$ form a radical sequence for $L / K$, and the radical degree of $\alpha_i$ is $n_i$.
\end{definition}

\begin{theorem} \label{thm:radical-single-prime}
    Let $K$ be a field. If $\alpha ^ n \in K$ for a positive integer $n$ and  $L = K (\alpha)$, then there exists a radical sequence $\beta_i$ for $L / K$ such that the radical degree for each $\beta_i$ is prime.
\end{theorem}

\begin{proof}
    Let $n = \prod_{k=1}^{s} p_{k}$ where $p_{k}$ are prime numbers (not necessarily distinct). Let $\beta_{i} = \alpha^ {p_{i + 1} \dots  p_{s}}$ for $i = 0, \dots s$. Write $L = K(\beta_1,  \dots, \beta_s = \alpha)$, and then $\beta_0 = \alpha^n \in K$ and  $\beta_i ^ {p_i} = \beta_{i-1} \in K(\beta_1, \dots, \beta_{i - 1})$ for each $i  = 1, \dots, s$. 
\end{proof}

\begin{example}
    Consider $K  = \mathbb Q$ and $\alpha = \sqrt[20]{2}$, where $\alpha ^ {20} \in \mathbb Q$. Write $n = 20 =  2 \times 2 \times 5 $. Then we can take the radical sequence as $\beta_1 = \sqrt 2, \beta_2 = \sqrt[4]{2}, \beta_3 = \sqrt[20]{2}$. Then $\mathbb Q(\sqrt[20]{2}) $ $= \mathbb Q(\sqrt{2}, \sqrt[4]{2}, \sqrt[20]{2})$ and 
    $$
    \beta_1 ^ 2 \in \mathbb Q, \quad \beta_2 ^ 2 \in \mathbb Q(\sqrt{2}), \quad \beta_3^5 \in \mathbb Q (\sqrt{2}, \sqrt[4]{2}),
    $$
    where each of the radical degrees $2, 2, 5$ is prime.
\end{example}

\begin{theorem} \label{thm:radical-all-prime}
    Let $L / K$ be a racial extension. Then there exists a radical sequence $\beta_j$ for $L / K$ where $j=1, \dots, r$ such that the radical degree of each $\beta_j$ is prime.
\end{theorem}

\begin{proof}
    For each $\alpha_i$ in any given radical sequence for $L / K$, replace it with a series $\beta_{ij}$ by Theorem \ref{thm:radical-single-prime}. Then combine all $\beta_{ij}$ and relabel the subscripts to make $\beta_{j}$. 
\end{proof}

\begin{theorem} \label{thm:radical-3}
If $K$ is a subfield of $\mathbb{C}$ and $L / K$ is normal and radical, then $\Gal(L / K)$ is soluble.
\end{theorem}

\begin{proof}
Let $L=K\left(\alpha_1, \ldots, \alpha_n\right)$ with $\alpha_j^{n_j} \in K\left(\alpha_1, \ldots, \alpha_{j-1}\right)$. By Theorem \ref{thm:radical-all-prime}, we may assume that $n_j$ is prime for all $j$. We prove the result by induction on $n$. 
 % In particular there is a prime $p$ such that $\alpha_1^p \in K$.

If $n = 0$, we have $L = K$, and the case is trivial.

If $n \ge 1$ and $\alpha_1 \in K$, then $L=K\left(\alpha_2, \ldots, \alpha_n\right)$ and $\Gal(L / K)$ is soluble by induction.

Let $n \ge 1$ and $\alpha_1 \notin K$. Let $p = n_1$, which is prime, and then $\alpha_1^p \in K$.  Then $L$ contains a primitive $p$-th root of unity $\omega$ by Theorem \ref{thm:unity-3}. Let $M = K(\omega)$ and let $N = M(\alpha_1)$. Consider the chain of subfields $K \subseteq M \subseteq N \subseteq L$. We now prove the solubility of $\Gal(L/N), \Gal(N/M), \Gal(L/M), \Gal(M/K)$, and $\Gal(L/K)$ step by step:

\begin{itemize}
    \item Since $L=N\left(\alpha_2, \ldots, \alpha_n\right)$, $L / N$ is a normal radical extension. By induction $\Gal\left(L / N\right)$ is soluble. 
    \item Since $ \omega \in M$ and $\alpha_1^p \in M$, Theorem \ref{thm:radical-2} implies that $N$ is a splitting field for $t^p-\alpha_1^p$ over $M$. Thus $N / M$ is normal. By Theorem \ref{thm:radical-2}, $\Gal\left(N / M\right)$ is abelian and hence soluble. 
    \item  $L/ K$ is finite and normal, and so is $L / M$. Apply Theorem \ref{thm:correspondence-quotient} to $L / M$ to deduce that
    $$
    \Gal\left( N / M \right) \cong \Gal(L / M) / \Gal\left(L / N\right).
    $$
    Hence by Theorem \ref{thm:soluble-main},  $ \Gal(L / M)$ is soluble.
    \item By Theorem \ref{thm:unity-1}, $M$ is the splitting field for $t^p-1$ over $K$. Then the extension $M / K$ is normal. By Theorem \ref{thm:radical-1}, $\Gal(M / K)$ is abelian and hence soluble.
    \item  Apply Theorem \ref{thm:correspondence-quotient} to $L / K$ to obtain
    $$
    \Gal(M / K) \cong \Gal(L / K) / \Gal(L / M). 
    $$
    Theorem \ref{thm:soluble-main} shows that $\Gal(L / K)$ is soluble, completing the induction step.
\end{itemize}
\end{proof}


\begin{definition}
	A \textit{normal closure} of a field extension $L / K$ is an extension $N$ of $L$ such that 
	\begin{enumerate}
		\item $N / K$ is normal;
		\item If $L \subseteq M \subseteq N$ and $M / K$ is normal, then $M = N$.
	\end{enumerate}
\end{definition}

\begin{theorem} \label{thm:radical-closure}
    If $L / K$ is a radical extension in $\mathbb{C}$ and $M$ is the normal closure of $L / K$, then $M / K$ is radical.
\end{theorem}

\begin{proof}
\TODO paraphrase

Let $L=K\left(\alpha_1, \ldots, \alpha_m\right)$ with $\alpha_i^{n_i} \in K\left(\alpha_1, \ldots, \alpha_{i-1}\right)$. Let $f_i$ be the minimal polynomial of $\alpha_i$ over $K$. Then $M \supseteq L$ is clearly the splitting field of $\prod_{i=1}^m f_i$. For every zero $\beta_{i j}$, by Theorem \ref{thm:automorphism-from-zeros}, there exists a $K$-automorphism $\tau$ of $M$ such that $\tau(\alpha_i) = \beta_{ij}$. Since $\alpha_i$ is a member of a radical sequence for a subfield of $M$, so is $\beta_{i j}$. Combining the sequences yields a radical sequence for $M$.
\end{proof}


\begin{definition}
    Let $f$ be a polynomial over a subfield $K$ of $\mathbb{C}$, with splitting field $\Sigma$ over $K$. Then $f$ is \textit{soluble by radicals} if there exists a field $M \supseteq \Sigma$ such that $M / K$ is a radical extension.  
\end{definition}


\begin{theorem} \label{thm:radical-galois-soluble}
    Let $f$ be a polynomial over a subfield $K$ of $\mathbb{C}$. If $f$ is soluble by radicals, then $\Gal(f)$ is soluble.
\end{theorem}

\begin{proof}
\TODO paraphrase

Since $f$ is soluble by radicals, its splitting field $\Sigma$ over $K$ satisfies $K \subseteq \Sigma \subseteq M$ where $M / K$ is a radical extension. Let $K_0$ be the fixed field of $\Gal(\Sigma / K)$, and let $N / M$ be the normal closure of $M / K_0$. Then
$$
K \subseteq K_0 \subseteq \Sigma \subseteq M \subseteq N.
$$
Since $M / K_0$ is radical, Theorem \ref{thm:radical-closure} implies that $N / K_0$ is a normal radical extension. By Theorem \ref{thm:radical-3}, $\Gal\left(N / K_0\right)$ is soluble.
By Theorem \ref{thm:fix-extension-normal}, the extension $\Sigma / K_0$ is normal. By Theorem \ref{thm:correspondence-quotient}
$$
\Gal\left(\Sigma / K_0\right) \cong \Gal\left(N / K_0\right) / \Gal(N / \Sigma).
$$
Theorem \ref{thm:soluble-main} implies that $\Gal\left(\Sigma / K_0\right)$ is soluble. But $\Gal(\Sigma / K)=\Gal\left(\Sigma / K_0\right)$, so $\Gal(\Sigma / K)$ is soluble.

\end{proof}

The converse also holds, but we do not prove it here.

\subsection{An Insoluble Quintic}

Now we establish a case where the Galois group of a polynomial over $\mathbb Q$ is isomorphic to a symmetric group.

\begin{theorem} \label{thm:galois-iso-symmetric}
    Let $p$ be a prime number and $f$ be a irreducible polynomial of degree $p$ over $\mathbb Q$. If $f$ has exactly two non-real zeros in $\mathbb C$, then the Galois group of $f$ over $\mathbb Q$ is isomorphic to the symmetric group $S_p$.
\end{theorem}

\begin{proof}
%    Theorem \ref{thm:fundamental-algebra} implies that there exists a splitting field $\Sigma $ of $f$ contained in $\mathbb C$. The Galois group $\Gal(\Sigma / \mathbb Q)$ of $f$ can be considered as a permutation group of the zeros of $f$ \textbf{(Group action??)}. By Theorem ??, $f$ has distinct zeros, so $\Gal(\Sigma / \mathbb Q)$ is isomorphic to a subgroup of $S_p$. We denote this subgroup of $S_p$ as $G$. 
	By Theorem \ref{thm:galois-group-isomorphic-symmetric-subgroup}, $\Gal(f)$ is isomorphic to a subgroup of $S_p$. We denote this subgroup as $G$. We now claim that $G = S_p$ by showing that $G$ contains a transposition and a $p$-cycle.

    Complex conjugation restricted to $\Sigma$ is a $\mathbb Q$-automorphism of $\Sigma$. It fixes all $p - 2$ real zeros of $f$ and transposes the two non-real zeros. Therefore $G$ contains a transposition. 

    Take any zero $\alpha$ of $f$. Since $f$ is irreducible, it is the minimal polynomial of $\alpha$. Then by the \textbf{Degree Theorem}, $$[\mathbb Q(\alpha) : \mathbb Q] = \partial f = p. $$ Theorem \ref{thm:tower-theorem} implies that $$[\Sigma : \mathbb Q] = [\Sigma : \mathbb Q(\alpha)] [ \mathbb Q(\alpha) : \mathbb Q]. $$ By Theorem \ref{thm:fixed}, $|G| = [\Sigma : \mathbb Q]$, and thus $p$ divides $|G|$. Theorem \ref{thm:cauchy} gives the existence of an element of order $p$ in $G$, but the only elements of order $p$ in $S_p$ are $p$-cycles. Therefore $G$ contains a $p$-cycle.

    Hence by Theorem \ref{thm:symmetric-prime}, $ G = S_p$ and therefore $\Gal(\Sigma / \mathbb Q) \cong S_p$.
\end{proof}

We now give a quintic polynomial not soluble by radicals. 

\begin{example}
    The polynomial $x^5 + 10 x^4 - 2$ over $\mathbb Q$ is not soluble by radicals.
\end{example}

\begin{proof}
    $f$ is irreducible by \textbf{Eisenstein's criterion} with $p = 2$. The degree of $f$ is $5$, which is prime. Calculus shows that $f$ has three real zeros. Thus the Galois group of $f$ is isomorphic to $S_5$ by Theorem \ref{thm:galois-iso-symmetric}. But $S_5$ is not soluble by Theorem \ref{thm:symmetric-not-soluble}. Thus by Theorem \ref{thm:radical-galois-soluble}, $f$ is not soluble by radicals.
\end{proof}

\subsection{Abel-Ruffini Theorem}

If we think of a general quadratic polynomial, i.e one of the form $aX^2+bX+c$, and try to find its roots, we know that there is a formula to find the roots of the general quadratic, which is $x = \frac{-b \pm \sqrt{b^2 - 4ac}}{2a}$. A similar formula can be found to find the roots of a general cubic and quartic polynomial. However, there is no such general formula for polynomial of degree 5 an higher.

The Abel-Ruffini theorem, was posed and proved (incompletely) by Paolo Ruffini and completed by Niels Henrik Abel, shows that the general polynomial of order five and above is not solvable by radicals. For this section, we assume that all fields that we mention have characteristic 0.

First, we need to understand a lemma in order to be able to prove the Abel-Ruffini Theorem. The proof of this lemma is adapted from a lecture by DeVille \cite{galois-lecture-polynomials} and the lecture notes by Hofscheier \cite{commutative-algebra-uon}

\newpage
For the following lemma, let $L :=$Frac$(\C[X_1,X_2,...,X_n])$, that is to say $L$ is the field of fractions of $\C[X_1,X_2,...,X_n]$. Now, recall that $e_i(X_1,...,X_n)$ is defined as the $i^{th}$ elementary symmetric polynomial. From this we define $K:=\C[e_1,...,e_n]$. 
We then see that $L$ is the splitting field of the general degree $n$ polynomial, over $K$

We define the generic polynomial as such:
$P_n(x) := x^n-s_1x^{n-1}+...+(-1)^ns_n \in K[x]$
    
\begin{lemma}\label{lemma:galois-symmetric}
    The Galois Group $\Gal(L/K)$ is isomorphic to $S_n$.
\end{lemma}

\begin{proof}

    $S_n$ acts on L by permutation of the variables $X_1,...,X_n$. Thus if we have a polynomial $f(X_1,X_2,...,X_n)$, we have that for $\sigma \in S_n$:
    $$   \sigma( f(X_1, \dots, X_n) )=f( X_{\sigma(1)},\dots,X_{\sigma(n)}) $$

    This action fixes the base field $K$, which tells us $S_n \leq \Gal(L/K) \implies |\Gal(L/K)|\geq n!$.

    Then, we know that $L$ is a splitting field of a polynomial $P_n$ over $K$, thus by Theorem \ref{thm:upper-bound-splitting-field}, $[L:K]\leq n!$. Then by Theorem \ref{thm:galois-less-than-extension}, we see that $n!\leq |\Gal(L/K)|\leq [L:K]\leq n!$ and so $|\Gal(L/K)| = n! = [L:K]$ and so $Gal(L/K)=S_n$

\end{proof}

Next we need to show that a simple radical extension is solvable, in order to be able to fully prove Galois' Theorem.

\begin{lemma}\label{lemma:solvable-radical-extension}
    A simple radical extension is solvable.
\end{lemma}

\begin{proof}

    \textcolor{blue}{Let $a$ be a root of the polynomial $f(x)=x^n-c$ and let $\omega$ be an n-th primitive root of unity. Then we have that all the roots of $f(x)$ are of the form $a \cdot \omega^i$ where $0\leq i \leq n-1$ $\Gal(K/k)$}
	    
	(\TODO Why is it of the form $K(\sqrt{a})$? It can be $K(e^{2\pi i / n}\sqrt[n]{a})$ in general.)
	
    Let $K(\sqrt{a})$ be a simple radical extension, where $K$ is a field and $a$ is not a square in $K$.

    Since $K$ is a field, $K(a)$ is an algebraic extension of $K$ and therefore $K(a)/K$ is a simple extension.
    
    \noindent
    Now let $b=\sqrt{a}$, so therefore $b$ is a root of the polynomial $f(x)=x^2-a$, and this is irreducible over $K$, and $K(b)/K = K(\sqrt{a})/K$ is also a simple extension. Thus we have $[K(b):K]=2$, and since the extension is simple $\Gal(K(b)/K)$ is isomorphic to $\Z/2\Z$. Since $\Z/2\Z$ is abelian, it follows that $K(\sqrt{a})$ is solvable.
\end{proof}

\noindent
Now, we propose a theorem, which is Galois' Theorem, in order to set up the main proof of the Abel-Ruffini Theorem. The following proof has been adapted from Wilson \cite{cambridge-galois-lecture-polynomials} and Mrinial \cite{Abel-Ruffini}.

\begin{theorem}[Galois' Theorem]\label{thm:galois-theorem}
     A polynomial $f(x)$ is solvable by radicals if and only if its Galois group is solvable
\end{theorem}

\begin{proof}
    "$\implies $"Given a general polynomial of the form $f(x) = a_0X^n - a_1X^{n-1}+...(-1)^na_n$, let us assume that it can be solved by radicals. Then we have a sequence of intermediate field extensions such that $K \subset L_1 \subset L_2 \subset ... \subset L_n :=F$, where we have that each extension $L_i/L_{i-1}$ is generated by a root of the polynomial, say $x=\zeta_i$, in $L_{i-1}[x].$
    By the Fundamental Theorem of Galois Theory \ref{thm:fundamental-theorem}, each subfield $L_i$ corresponds to a subgroup of $\Gal(F/K)$, and let $G_i$ be the subgroup which $L_i$ corresponds to. Then we can say that $\Gal(F/K)$ is a composition of the subgroups $G_i$:
    $1=G_1\subseteq G_2 \subseteq ... \subseteq G_n = \Gal(F/K)$. 
    Then we have that each quotient $G_i/G_{i-1}$ is either abelian or solvable $\implies Gal(F/K)$ is solvable, because it is the composition of subgroups that have solvable or abelian Galois groups.
    \newline Thus we have that if $f(x)$ is solvable by polynomials, then its Galois group $\Gal(F/K)$ is solvable.
    

    "$\impliedby$" Instead, supposed the polynomial $f(x)$ has a Galois group, $G$, which is solvable. Then we can consider the following chain of subfields: $F=K_0\subset K_1 \subset ... \subset K_n = K$. Here we define $K_i = \Fix(G_i)$ where $G_i$ is a subgroup of $G$.
    Since $G$ is solvable, and by the Galois Correspondence, we have that $K$ is cyclic which implies that $K_{i+1}/K_i$ is also cyclic.

    Then if we attach the $k_i^{th}$ roots of unity, we can compose this with our field $F$ as well as to the chain of subfields above to give us another chain, which we will define as $F_1 = F_1K_0 \subset F_1K_1 \subset ... \subset F_1K_n = F_1K$. We therefore have a chain of extensions such that $k_i \in F_1$ $\forall i$ \hspace{0.1cm} $\in I $ where $I$ is the index for number of roots of unity. Since $K_{i+1}/K_i$ is cyclic, we also have that $F_1K_{i+1}/F_1K_i$ is cyclic, and our base field contains all the roots of unity. Thus, the extensions are simple radical extensions, which by Lemma \ref{lemma:solvable-radical-extension} implies that they are solvable, which then implies that $f(x)$ is solvable.
\end{proof}

\begin{theorem}[Abel-Ruffini Theorem]\label{thm:abel-ruffini-thm}
    The general polynomial of degree n is not solvable by radicals for $n \ge 5$
\end{theorem}

\begin{proof}
    By Lemma \ref{lemma:galois-symmetric}, we know that the general polynomial, written in the form $f(x)=a_0X^n - a_1X^{n-1} + ... + (-1)^na_n$ has Galois group isomorphic to $S_n$. Then by Theorem \ref{thm:symmetric-not-soluble} we can see that for $n \geq 5$, $S_n$ is not soluble, which by Theorem \ref{thm:galois-theorem} implies that the general polynomial of degree $n \geq 5$ is not solvable.
\end{proof}
