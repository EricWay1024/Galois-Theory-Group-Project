\section{Galois Groups and Polynomials} \label{sec:galois-groups-and-polynomials}
If we think of a general quadratic polynomial of the form $at^2+bt+c$, and try to find its roots, we know that there is a formula to find the roots $\alpha$ of the general quadratic, which is $\alpha = \frac{-b \pm \sqrt{b^2 - 4ac}}{2a}$. A similar formula can be found to find the roots of a general cubic and quartic polynomial. However, there is no such general formula for polynomial of degree $5$ an higher.

In this section, we first briefly introduce the concept of solvable and simple groups from group theory. Then we concentrate on proving the Galois' Theorem, which establishes that a polynomial being solvable by radicals is equivalent to its Galois group being solvable. We also exemplify the theorem with a non-solvable quintic polynomial. Finally, we look at the Abel-Ruffini Theorem, which states the non-solvability of generic polynomials of degree $n \ge 5$.  

From now on, we assume that all fields has characteristic $0$.  
\subsection{Solvable Groups and Simple Groups}
This subsection, based on \cite[Chapter~14]{Stewart}, gives a brief overview of solvable groups and simple groups. 
\begin{definition} \label{def:soluble}
    A group $G$ is \textit{solvable} if it has a finite series of subgroups 
    $ \{ e \} = G_0 \triangleleft G_1 \triangleleft \dots \triangleleft G_n = G$
    such that the quotient $G_{i+1} / G_{i}$ is abelian for each $i = 0, 1, ...,  n - 1$.
\end{definition}

Clearly, any abelian group is solvable. The main properties of solvable groups are listed in the following theorem, which is of a group theory nature and is not closely related to Galois theory, so we postpone its proof to the appendix (see Theorem \ref{thm:soluble-main-appendix}). 

\begin{theorem} \label{thm:soluble-main}
    Let $G$ be a group, $H \le G$ and $N \trianglelefteq G$. Then 
    (i) if $G$ is solvable, then $H$ is solvable;
    (ii) if $G$ is solvable, then $G / N$ is solvable; 
    (iii) if $N$ and $G / N$ are solvable, then $G$ is solvable. 
    % \begin{enumerate}[label=(\roman*)]
    %     \item If $G$ is solvable, then $H$ is solvable;
    %     \item If $G$ is solvable, then $G / N$ is solvable; 
    %     \item If $N$ and $G / N$ are solvable, then $G$ is solvable. 
    % \end{enumerate}
\end{theorem}


An important result is that symmetric groups are in general not solvable.
For that we introduce simple groups and show that the alternating groups are simple in general. 

\begin{definition}
	A group $G$ is \textit{simple} if it is nontrivial and its only normal subgroups are $\{ e \}$ and $G$. 
\end{definition}

\begin{theorem} \label{thm:soluble-and-simple}
	A solvable group is simple if and only if it is a cyclic group of prime order.
\end{theorem}

\begin{proof}
	Let $G$ be a solvable group and suppose $G$ is simple. Consider the series of subgroups of $G$: 
	$
	\{ e \}=G_0 \triangleleft G_1 \triangleleft \ldots \triangleleft G_n=G,
	$
	with abelian quotients, and without loss of generality we assume $G_{i+1} \neq G_i$. Since $G$ is simple, $G_{n-1}$, which is a proper normal subgroup of $G_n = G$, must be $\{ e \}$. Solvability of $G$ implies that $G_n / G_{n -1 }$ is abelian, but $G_n / G_{n - 1} = G$, and thus $G$ is abelian. Thus for any element $g \in G$, the cyclic group $\langle g\rangle$ is a normal subgroup in $G$. Hence  $G = \langle g\rangle$ for any $g \neq e$. Hence $G$ is cyclic of prime order.
	The converse is trivial.
\end{proof}


\begin{theorem} \label{thm:simple-alternating}
	The alternating group $A_n$ is simple when $n \ge 5$. 
\end{theorem}

\begin{proof}
	Due to the length and technical nature of the complete proof, only a concise summary is presented here. 
	Suppose that $\{ e \} \neq N \triangleleft A_n$. We can prove that $N$ must contain a $3$-cycle using case-by-case analysis. Next, we can show that if $N$ contains a $3$-cycle, then it contains all $3$-cycles. Since $A_n$ is generated by the $3$-cycles when $n \ge 3$, this means $N = A_n$ \cite[p.~164]{Stewart}.
\end{proof}

\begin{theorem} \label{thm:symmetric-not-soluble-appendix}
	The symmetric group $S_n$ is not solvable when $n \ge 5$. 
\end{theorem}

\begin{proof}
	Suppose $S_n$ is solvable for $n \ge 5$. Then since $A_n \trianglelefteq S_n$, by Theorem \ref{thm:soluble-main}, $A_n$ is solvable. By Theorem \ref{thm:simple-alternating}, $A_n$ is also simple, so $A_n$ is cyclic of prime order by Theorem \ref{thm:soluble-and-simple}. But $|A_n| = n! / 2$ which obviously is not prime. This is a contradiction.
\end{proof}




%\begin{theorem} \label{thm:symmetric-not-soluble}
%	The symmetric group $S_n$ is not solvable when $n \ge 5$. 
%\end{theorem}
%\begin{proof}
%	See Theorem \ref{thm:symmetric-not-soluble-appendix}. 
%\end{proof}




\subsection{Solvability by Radicals}
This subsection, based on \cite[Chapter~15]{Stewart}, \cite[p.~71-75]{  rotman_galois_1998}, \cite[Section~9]{introduction-to-galois-theory} and \cite{cambridge-galois-lecture-polynomials}, focuses on establishing the Galois' Theorem. We first formalise the idea of a number in $\C$ being expressible by radicals; such a number can be constructed within finitely many steps of addition, subtraction, multiplication, division and taking the $n$-th root on rational numbers. We give a general definition as below. 
\begin{definition} \label{def:radical-extension}
	A field extension $L / K$ in $\C$ is \textit{radical} if $L=K\left(\alpha_1, \ldots, \alpha_m\right)$, where for each $i=1, \ldots, m$ there exists $n_i$ such that
	$
	\alpha_i^{n_i} \in K\left(\alpha_1, \ldots, \alpha_{i-1}\right).
	$
	The elements $\alpha_i$ form a radical sequence for $L / K$, and the radical degree of $\alpha_i$ is $n_i$. A number $\beta$ in $\C$ is \textit{expressible by radicals} if there exists a radical extension $L / \Q$ such that $\beta \in L$. 
\end{definition}

\begin{definition}
	Let $f$ be a polynomial over a subfield $K$ of $\C$. Then $f$ is \textit{solvable by radicals} if there is a radical field extension $M/K$ such that $f$ splits in $M$. 
\end{definition}

\begin{theorem} \label{thm:radical-single-prime}
	Let $L / K$ be a radical extension in $\C$. Then there exists a radical sequence $\beta_j$ for $L / K$ where $j=1, \dots, r$ such that the radical degree of each $\beta_j$ is prime.
\end{theorem}

\begin{proof}
	Without loss of generality, assume that $L/K$ is simple, i.e.  $L = K (\alpha)$ and $\alpha ^ n \in K$ for a positive integer $n$ (for if $L/K$ is not simple, we can always write it as a tower of simple extensions). Let $n = \prod_{k=1}^{s} p_{k}$ where $p_{k}$ are prime numbers (not necessarily distinct). Let $\beta_{i} = \alpha^ {p_{i + 1} \dots  p_{s}}$ for $i = 0, \dots s$. Write $L = K(\beta_1,  \dots, \beta_s = \alpha)$, and then $\beta_0 = \alpha^n \in K$ and  $\beta_i ^ {p_i} = \beta_{i-1} \in K(\beta_1, \dots, \beta_{i - 1})$ for each $i  = 1, \dots, s$. 
\end{proof}

\begin{example} \label{exm:radical-sequence-prime}
	Consider $K  = \mathbb Q$ and $\alpha = \sqrt[20]{2}$, where $\alpha ^ {20} \in \mathbb Q$. Write $n = 20 =  2 \times 2 \times 5 $. Then we can take the radical sequence as $\beta_1 = \sqrt 2, \beta_2 = \sqrt[4]{2}, \beta_3 = \sqrt[20]{2}$. Then $\mathbb Q(\sqrt[20]{2}) $ $= \mathbb Q(\sqrt{2}, \sqrt[4]{2}, \sqrt[20]{2})$ and 
	$
	\beta_1 ^ 2 \in \mathbb Q, \, \beta_2 ^ 2 \in \mathbb Q(\sqrt{2}), \, \beta_3^5 \in \mathbb Q (\sqrt{2}, \sqrt[4]{2}),
	$
	where each of the radical degrees $2, 2, 5$ is prime.
\end{example}
%\begin{proof}
%	For each $\alpha_i$ in any given radical sequence for $L / K$, replace it with a series $\beta_{ij}$ by Theorem \ref{thm:radical-single-prime}. Then combine all $\beta_{ij}$ and relabel the subscripts to make $\beta_{j}$. 
%\end{proof}

%See also Example \ref{exm:radical-sequence-prime}. 
Hence from now on, we can always assume without loss of generality that a radical extension has all prime radical degrees. We now introduce the roots of unity in $\C$, useful for further theorems. 


\begin{definition}
	Let $n$ be a positive integer. An $n$-th root of unity $\omega$ in $\mathbb C$ is such that $\omega ^ n = 1$. An $n$-th root of unity $\omega$ is \textit{primitive} if for any $m = 1, 2, \dots, n - 1$, $\omega ^ m \neq 1$.
\end{definition}


Let $n$ be a positive integer. All $n$-th roots of unity form a cyclic group under multiplication, the group is generated by any primitive $n$-th root of unity. For a prime number $p$, let $\omega$ be a $p$-th root of unity and let $\omega \neq 1$. Then $\omega$ is a primitive $p$-th root of unity and generates the multiplicative group of all $p$-th roots of unity. We also see that since the polynomial $t^n - 1$ over a subfield $K$ of $\C$ has no repeated roots by Theorem \ref{thm:separable-derivative}, if $\omega$ is a primitive $n$-th root of unity, $K(\omega)$ is the splitting field of $t^n - 1$ over $K$. The following two lemmas indicate how the roots of unity can give rise to abelian Galois groups of field extensions. 


%\begin{theorem} \label{thm:unity-1}
%	Let $n$ be a positive integer and let $\omega$ be a primitive $n$-th root of unity. Let $L$ be a subfield of $\mathbb C$. If $\omega \in L$, then the polynomial $t^n - 1$ splits in $L$.
%\end{theorem}
%\begin{proof}
%	$\omega$ has order $n$ in the multiplicative group of $L$, so the elements $1, \omega, \omega^2, \ldots, \omega^{n-1}$ are distinct $n$-th roots of unity in $L$. Therefore $t^n-1$ splits in $L$.
%\end{proof}

%\begin{theorem} \label{thm:unity-2}
%	Let $n$ be a positive integer and let $\omega$ be a primitive $n$-th root of unity. If $L$ is the splitting field for $t^n - 1$ over a subfield $K$ of $\mathbb C$, then $L = K(\omega)$.
%\end{theorem}
%
%\begin{proof}
%	The derivative of $t^n-1$ is $n t^{n-1}$, which is prime to $t^n-1$, so the polynomial $t^n-1$ has no multiple zeros in $L$ by . The group of its zeros under multiplication thus has order $n$ and is cyclic. Let $\omega$ be a generator of this group, and thus $\omega$ is a primitive $n$-th root of unity. Then $L=K(\omega)$. 
%\end{proof}




\begin{lemma} \label{thm:radical-1}
	Let $p$ be a prime and let $\omega$ be a primitive $p$-th root of unity. Let $K$ be a subfield of $\mathbb C$. Then $\Gal(K(\omega) / K)$ is abelian.
\end{lemma}
\begin{proof}
	Let $L = K(\omega)$.  Let $\alpha \in \Gal(L / K)$. Then $\alpha$ is uniquely determined by $\alpha(\omega)$ and $\alpha$ is a permutation of the multiplicative group generated by $\omega$. Thus $\alpha$ has the form
	$
	\alpha_j: \omega \mapsto \omega^j,
	$
	where $j=1,\dots,p-1$. Then $\alpha_i \alpha_j (\omega) = \alpha_j \alpha_i (\omega) = \omega^{i j}$, so $ \Gal(L / K)$ is abelian.	
\end{proof}

\begin{lemma} \label{thm:radical-2}
	Let $K$ be a subfield of $\mathbb{C}$ which contains an $n$-th primitive root of unity. Let $\beta^n = c \in K $ where $n$ is a positive integer. Then $\Gal(K(\beta) / K)$ is abelian.
\end{lemma}

\begin{proof}
	Clearly $K$ contains all $n$-th roots of unity. Let $L = K(\beta)$, and $\beta$ is a root of $t^n-c$ in $L$. Any root of $t^n-c$ in $L$ can be represented by $\omega \beta$, where $\omega$ is some $n$-th root of unity in $K$. Thus $t^n - c$ splits in $L$.  Let $\phi \in \Gal(L / K)$, then $\phi$ is uniquely determined by $\phi(\beta)$ and $\phi$ is a permutation of the roots of $t^n - c$. Let $\phi_1, \phi_2 \in \Gal(L / K)$, and let $\phi_1(\beta) = \omega_1\beta$, $\phi_2(\beta) = \omega_2\beta$, where $\omega_1, \omega_2$ are $n$-th roots of unity in $K$. Then
	$
	\phi_1 \phi_2(\beta)=\omega_1 \omega_2 \beta=\omega_2 \omega_1  \beta=\phi_2 \phi_1(\beta).
	$
	Thus $\Gal(L / K)$ is abelian.
\end{proof}
Using the fact that abelian groups are solvable, the following theorem implies that a simple radical extension with a prime degree is solvable. 
\begin{theorem} \label{thm:radical-simple-solvable}
	Let $f(t) = t^p - c$ be a polynomial over a subfield $K$ of $\C$ where $p$ is a prime. Then $\Gal(f)$ is solvable. 
\end{theorem}
\begin{proof}
	Let $M$ be the splitting field of $f$ over $K$, then $\Gal(f) = \Gal(M / K)$. 
	When $c = 0$, the case is trivial with $M = K$. Suppose $c \neq 0$. By Theorem \ref{thm:separable-derivative}, $f$ is separable over $K$, and thus its roots are all distinct. Take any two different roots $\alpha, \beta$ of $f$, we see that $\alpha / \beta \neq 1$ is a (primitive) $p$-th root of unity. Therefore, $M = K(\alpha, \omega)$, where $\alpha$ is a root of $f$ and $\omega$ is a primitive $p$-th root of unity. $K(\omega)$ is the splitting field for $t^p - 1$ over $K$ and thus $K(\omega) / K$ is Galois by Theorems \ref{thm:normal-equiv-def} and \ref{thm:separable-extension-in-C}.  Since $M / K$ is also a Galois extension, applying Theorem \ref{thm:correspondence-quotient} to obtain $\Gal(M / K) / \Gal(M / K(\omega)) \cong \Gal(K(\omega) / K). $
	Since $\Gal(K(\omega)/K)$ is abelian by Lemma \ref{thm:radical-1} and $\Gal(M / K(\omega))$ is abelian by Lemma \ref{thm:radical-2}, by Theorem \ref{thm:soluble-main} $\Gal(M/K)$ is solvable as desired. 
\end{proof}

The next theorem allows us to pass the solvability of the Galois group along a tower of simple radical extensions. 

%\begin{wrapfigure}{l}{0.25\textwidth}
%	\centering
%	\begin{tikzpicture}
%		\node(K) {$K$};
%		\node(L) [below=1cm of K] {$L$}; 
%		\node(M) [right=1.3cm of K] {$M$};
%		\node(N) [below=1cm of M] {$N$}; 
%		
%		\draw[->] (K) --  node[above] {$t^p - c$} (M);
%		\draw[->] (L) --  node[below] {$t^p - c$} (N);
%		\draw[->] (K) --  node[left] {$f(t)$} (L);
%		\draw[->] (M) --  node[right] {$f(t)$} (N);
%	\end{tikzpicture}
%\end{wrapfigure}

\begin{theorem} \label{thm:solvable:chain}
	Let $f$ be a polynomial over a subfield $K$ of $\C$. Let $K' = K(\alpha)$, where $\alpha^p = c \in K$ for some prime $p$. Let $L$ be the splitting field of $f$ over $K$, then $\Gal(L/K)$ is solvable if and only if $\Gal(L / K')$ is solvable. 
\end{theorem}



\begin{proof}
	Let $N$ be the spitting field of the polynomial $f(t)(t^p - c)$ over $K$ and let $M$ be the splitting field of $t^p -c $ over $K$ (see Figure \ref{fig:solvable-chain}). Then $M$ and $L$ are both contained in $N$, and $N/K$, $L/K$ and $M/K$ are all Galois extensions. By Theorem \ref{thm:correspondence-quotient}, we have that 
	$$ \Gal(L / K) \cong \frac{\Gal(N / K)}{\Gal(N / L)} \text{ and } \Gal(M / K) \cong \frac{\Gal(N / K)}{\Gal(N / M)}.$$
	Now $M$ and $N$ are the splitting fields for $t^p - c$ over $K$ and $L$ respectively, so by Theorem \ref{thm:radical-simple-solvable}, $\Gal(M/K)$ and $\Gal(N / L)$ are solvable. Then by Theorem \ref{thm:soluble-main}, $\Gal(L/K)$ is solvable iff $\Gal(N/K)$ is solvable, and $\Gal(N / K)$ is solvable iff $\Gal(N/M)$ is solvable. Thus we have that $\Gal(L/K)$ is solvable iff $\Gal(N/M)$ is solvable. 
	
	Meanwhile, we see that $M$ is also the splitting field of $t^p - c$ over $K' = K(\alpha)$, since $\alpha$ is a root of $t^p - c$. Also, $N$ and $L$ are also the splitting fields of $f(t) (t^p - c)$ and $f$ over $K'$. Then the above argument can be applied with $K$ replaced with $K'$ and we have that $\Gal(L/K')$ is solvable iff $\Gal(N/M)$ is solvable. We then deduce that $\Gal(L/K)$ is solvable iff $\Gal(L/K')$ is solvable. 
\end{proof}

Now, we propose our main theorem of this section, which is (one direction of) Galois' Theorem. 

\begin{theorem}[Galois' Theorem]\label{thm:galois-theorem}
	A polynomial $f$ over a subfield $K$ of $\C$ is solvable by radicals if and only if its Galois group is solvable.
\end{theorem}

\begin{proof}
%	``$\implies $''. 
	Assume that $f$ over $K$ can be solved by radicals. Then we have a sequence of intermediate field extensions such that $K = L_0 \subset L_1 \subset L_2 \subset ... \subset L_n :=F$, where $f$ splits in $F$ and for each $i = 1, \dots, m$ we have that  $L_i = L_{i-1} (\alpha_i)$ for $\alpha_i ^ {p_i} \in L_{i-1}$ with $p_i$ prime. We can assume $F$ to be the splitting field of $f$ over $K$ \footnote[2]{This assumption is made by \cite[p.~75]{introduction-to-galois-theory} without explicit justification. In the absence of a proof or counterexample, we present an alternative proof in Appendix \ref{sec:radical-alter}, which only assumes that for a polynomial $f$ over $K$ solvable by radicals, the splitting field of $f$ over $K$ is a \textit{subfield} of a radical extension.}.
%	Here we assume that $F$ is the splitting field of $f$ over $K$ \footnote{This assumption is made by \cite{introduction-to-galois-theory} but unjustified.  
%%		but might not be true, as in general the splitting field $F'$ of $f$ over $K$ is \textit{contained} in $F$. The reader is referred to \cite{Stewart} (\TODO or appendix?) for the complete proof. 
%	}. 
Now $\Gal(F / F)$ is trivial and solvable, and by Theorem \ref{thm:solvable:chain}, $\Gal(F / L_i)$ is solvable iff $\Gal(F / L_{i-1})$ is solvable for all $i = 1, \dots, m$. It thus follows that $\Gal(F/ K) = \Gal(f)$ is solvable. 
%	, say $x=\zeta_i$, in $L_{i-1}[x].$
%	By the Fundamental Theorem of Galois Theory \ref{thm:fundamental-theorem}, each subfield $L_i$ corresponds to a subgroup of $\Gal(F/K)$, and let $G_i$ be the subgroup which $L_i$ corresponds to. Then we can say that $\Gal(F/K)$ is a composition of the subgroups $G_i$:
%	$1=G_1\subseteq G_2 \subseteq ... \subseteq G_n = \Gal(F/K)$. 
%	Then we have that each quotient $G_i/G_{i-1}$ is either abelian or solvable $\implies Gal(F/K)$ is solvable, because it is the composition of subgroups that have solvable or abelian Galois groups.
%	\newline Thus we have that if $f(x)$ is solvable by polynomials, then its Galois group $\Gal(F/K)$ is solvable.
%	\TODO the $\impliedby$ direction is not needed by Abel-Ruffini Theorem so it can be a sketch of proof
%	
%	``$\impliedby$''. Suppose $G :=  \Gal(f)$ is solvable. Then we can consider the following chain of subfields: $F=K_0\subset K_1 \subset ... \subset K_n = K$. Here we define $K_i = \Fix(G_i)$ where $G_i$ is a subgroup of $G$.
%	Since $G$ is solvable, and by the Galois Correspondence, we have that $K$ is cyclic which implies that $K_{i+1}/K_i$ is also cyclic.
%	
%	Then if we attach the $k_i^{th}$ roots of unity, we can compose this with our field $F$ as well as to the chain of subfields above to give us another chain, which we will define as $F_1 = F_1K_0 \subset F_1K_1 \subset ... \subset F_1K_n = F_1K$. We therefore have a chain of extensions such that $k_i \in F_1$ $\forall i$ \hspace{0.1cm} $\in I $ where $I$ is the index for number of roots of unity. Since $K_{i+1}/K_i$ is cyclic, we also have that $F_1K_{i+1}/F_1K_i$ is cyclic, and our base field contains all the roots of unity. Thus, the extensions are simple radical extensions, which by Lemma \ref{lemma:solvable-radical-extension} implies that they are solvable, which then implies that $f(x)$ is solvable.
The converse is also true, but we omit the proof here \cite[p.~76]{introduction-to-galois-theory}. 
\end{proof}


Now we demonstrate an example where the Galois group of a polynomial over $\mathbb Q$ of degree five is not solvable by radicals. For that we need this following theorem, which gives a symmetric group as the Galois group of a polynomial $f$ over $\Q$ under certain constraints of $f$. 

\begin{theorem} \label{thm:galois-iso-symmetric}
	Let $p$ be a prime number and $f$ be a irreducible polynomial of degree $p$ over $\mathbb Q$. If $f$ has exactly two non-real roots in $\mathbb C$, then the Galois group of $f$ over $\mathbb Q$ is isomorphic to the symmetric group $S_p$.
\end{theorem}

\begin{proof}
	%    Theorem \ref{thm:fundamental-algebra} implies that there exists a splitting field $\Sigma $ of $f$ contained in $\mathbb C$. The Galois group $\Gal(\Sigma / \mathbb Q)$ of $f$ can be considered as a permutation group of the roots of $f$ \textbf{(Group action??)}. By Theorem ??, $f$ has distinct roots, so $\Gal(\Sigma / \mathbb Q)$ is isomorphic to a subgroup of $S_p$. We denote this subgroup of $S_p$ as $G$. 
	By Theorem \ref{thm:galois-group-isomorphic-symmetric-subgroup}, $\Gal(f)$ is isomorphic to a subgroup of $S_p$. We denote this subgroup as $G$. We show that $G$ contains a transposition and a $p$-cycle. Complex conjugation restricted to splitting field $F$ is a $\mathbb Q$-automorphism of $F$. It fixes all $p - 2$ real roots of $f$ and transposes the two non-real roots. Therefore $G$ contains a transposition. Now
	take any root $\alpha$ of $f$. Since $f$ is irreducible, it is the minimal polynomial of $\alpha$. Then by Theorem \ref{thm:degree-theorem}, $[\mathbb Q(\alpha) : \mathbb Q] = \partial f = p. $ Theorem \ref{thm:tower-theorem} implies that $[F : \mathbb Q] = [F : \mathbb Q(\alpha)] [ \mathbb Q(\alpha) : \mathbb Q]. $ By Theorem \ref{thm:fixed}, $|G| = [F : \mathbb Q]$, and thus $p$ divides $|G|$. Theorem \ref{thm:cauchy} gives the existence of an element of order $p$ in $G$, but the only elements of order $p$ in $S_p$ are $p$-cycles. Therefore $G$ contains a $p$-cycle. Hence by Theorem \ref{thm:symmetric-prime}, $ G = S_p$ and therefore $\Gal(F / \mathbb Q) \cong S_p$.
\end{proof}

\begin{example}
	The polynomial $t^5 + 10 t^4 - 2$ over $\mathbb Q$ is not solvable by radicals.
\end{example}

\begin{proof}
	$f(t) = t^5 + 10 t^4 - 2$ is irreducible by Theorem \ref{thm:eisenstein} with $q = 2$. The degree of $f$ is $5$, which is prime. Calculus shows that $f$ has three real roots. Thus the Galois group of $f$ is isomorphic to $S_5$ by Theorem \ref{thm:galois-iso-symmetric}. But $S_5$ is not solvable by Theorem \ref{thm:symmetric-not-soluble-appendix}. Thus by Theorem \ref{thm:galois-theorem}, $f$ is not solvable by radicals.
\end{proof}

We now can see that \textit{not} all elements in $\mathbb A = \overline \Q$ are expressible by radicals (see Figure \ref{fig:hasse}). 

The Abel-Ruffini theorem, posed and proved (incompletely) by Paolo Ruffini and completed by Niels Henrik Abel, shows that the generic polynomial of order five and above is not solvable by radicals. Here we give a brief introduction to this theorem. Our method is based on what we have obtained from Galois theory and thus is different from the original proof. 
%For this section, we assume that all fields that we mention have characteristic 0.

\begin{definition}[Generic polynomials]
	Let $F$ be a subfield of $\C$. Consider the polynomial ring $F[t_1, \dots, t_n]$ in $n$ indeterminates. Let $L = F(t_1, \dots, t_n)$ be its field of fractions. For a positive integer $n$, define the \textit{generic polynomial} $P_n$ of degree $n$ `over $F$' (though its coefficients are really in $L$) of the form
	$
	P_n(t) = \prod_{k=1}^{n} (t - t_k),
	$
	where $t_1, \dots, t_n $ are $n$ distinct roots of $P_n$. Expanding $P_n$ gives that 
	$
	P_n(t) = \sum_{i=0}^{n} (-1)^{n - i} s_{n-i} t^i,
	$
	where recall that $s_{i} = e_i(t_1,\dots,t_n)$ is defined as the $i^{th}$ elementary symmetric polynomial.
\end{definition}

It can be shown that $s_{i}$ are \textit{algebraically independent}, i.e. they satisfy no non-trivial polynomial relation \cite[p.~112]{Stewart}. From this we define $K := F(s_1, \dots, s_n)$, then it can be shown that $P_n$ is a polynomial over $K$ with splitting field $L$, and thus $\Gal(P_n) = \Gal(L/K)$. 

If $P_n$ were solvable by radicals for any given $n$, we would be able to solve any specific complex polynomial equation of degree $n$ by simply replacing the coefficients $s_i$ with specific numbers. However, this is not the case, as we shall see by the Abel-Ruffini Theorem \cite{galois-lecture-polynomials, commutative-algebra-uon, Abel-Ruffini}.




%We can see that the splitting field of $f$ is $L := F(t_1, \dots, t_n)$. 
%
%
%For the following lemma, let $L :=$Frac$(\C[X_1,X_2,...,X_n])$, that is to say $L$ is the field of fractions of $\C[X_1,X_2,...,X_n]$. Now,  From this we define $K:=\C[e_1,...,e_n]$. 
%We then see that $L$ is the splitting field of the general degree $n$ polynomial, over $K$

%We define the generic polynomial as such:
%$P_n(x) := x^n-s_1x^{n-1}+...+(-1)^ns_n \in K[x]$
%The proof of this following lemma is adapted from a lecture by DeVille  and the lecture notes by Hofscheier . 

\begin{lemma}\label{lemma:galois-symmetric}
	With $P_n$ defined as above, the Galois group $\Gal(P_n)$ is isomorphic to $S_n$ .
\end{lemma}

\begin{proof}
	
	Let $S_n$ act on $L$ by permutation of the variables $t_1,...,t_n$. Thus if we have $\sigma \in S_n$, then for any $f(t_1,\dots,t_n) \in L $, we have 
	$   \sigma( f(t_1, \dots, t_n) )=f( t_{\sigma(1)},\dots,t_{\sigma(n)}) \in L. $
	Thus $\sigma \in \Aut(L)$.  Also notice that $\sigma$ fixes any element in $K$, and therefore
%	 $K \subseteq \Fix(S_n)$ and 
	 $\sigma \in \Gal(L/K)$, and thus $S_n \le \Gal(L/K)$.  Now by Theorem \ref{thm:galois-group-isomorphic-symmetric-subgroup}, since $P_n$ has degree $n$, we also have that $\Gal(L/K) \le S_n$. Thus $\Gal(L/K) \cong S_n$ . 
%	This action fixes the base field $K$, which tells us $S_n \leq \Gal(L/K) \implies |\Gal(L/K)|\geq n!$.
%	
%	Then, we know that $L$ is a splitting field of a polynomial $P_n$ over $K$, thus by Theorem \ref{thm:upper-bound-splitting-field}, $[L:K]\leq n!$. Then by Theorem \ref{thm:galois-group-order-upper-bound}, we see that $n!\leq |\Gal(L/K)|\leq [L:K]\leq n!$ and so $|\Gal(L/K)| = n! = [L:K]$. Then since we have that $|\Gal(L/K)| = n!$ and $S_n \leq \Gal(L/K)$ which implies that $\Gal(L/K) = S_n$
\end{proof}


\begin{theorem}[Abel-Ruffini Theorem]\label{thm:abel-ruffini-thm}
	The generic polynomial $P_n$ of degree $n$ is not solvable by radicals for $n \geq 5$. 
\end{theorem}

\begin{proof}
	By Lemma \ref{lemma:galois-symmetric}, we know that $P_n$ has a Galois group isomorphic to $S_n$. Then by Theorem \ref{thm:symmetric-not-soluble-appendix} we can see that for $n \geq 5$, $S_n$ is not solvable, which by Theorem \ref{thm:galois-theorem} implies that $P_n$ with $n \ge 5$ is not solvable.
\end{proof}


%
%\subsection{Solubility by Radicals}
%
%
%
%
%
%\begin{theorem} \label{thm:radical-3}
%If $K$ is a subfield of $\mathbb{C}$ and $L / K$ is normal and radical, then $\Gal(L / K)$ is soluble.
%\end{theorem}
%
%\begin{proof}
%Let $L=K\left(\alpha_1, \ldots, \alpha_n\right)$ with $\alpha_j^{n_j} \in K\left(\alpha_1, \ldots, \alpha_{j-1}\right)$. By Theorem \ref{thm:radical-all-prime}, we may assume that $n_j$ is prime for all $j$. We prove the result by induction on $n$. 
% % In particular there is a prime $p$ such that $\alpha_1^p \in K$.
%
%If $n = 0$, we have $L = K$, and the case is trivial.
%
%If $n \ge 1$ and $\alpha_1 \in K$, then $L=K\left(\alpha_2, \ldots, \alpha_n\right)$ and $\Gal(L / K)$ is soluble by induction.
%
%Let $n \ge 1$ and $\alpha_1 \notin K$. Let $p = n_1$, which is prime, and then $\alpha_1^p \in K$.  Then $L$ contains a primitive $p$-th root of unity $\omega$ by Theorem \ref{thm:unity-3}. Let $M = K(\omega)$ and let $N = M(\alpha_1)$. Consider the chain of subfields $K \subseteq M \subseteq N \subseteq L$. We now prove the solubility of $\Gal(L/N), \Gal(N/M), \Gal(L/M), \Gal(M/K)$, and $\Gal(L/K)$ step by step:
%
%\begin{itemize}
%    \item Since $L=N\left(\alpha_2, \ldots, \alpha_n\right)$, $L / N$ is a normal radical extension. By induction $\Gal\left(L / N\right)$ is soluble. 
%    \item Since $ \omega \in M$ and $\alpha_1^p \in M$, Theorem \ref{thm:radical-2} implies that $N$ is a splitting field for $t^p-\alpha_1^p$ over $M$. Thus $N / M$ is normal. By Theorem \ref{thm:radical-2}, $\Gal\left(N / M\right)$ is abelian and hence soluble. 
%    \item  $L/ K$ is finite and normal, and so is $L / M$. Apply Theorem \ref{thm:correspondence-quotient} to $L / M$ to deduce that
%    $$
%    \Gal\left( N / M \right) \cong \Gal(L / M) / \Gal\left(L / N\right).
%    $$
%    Hence by Theorem \ref{thm:soluble-main},  $ \Gal(L / M)$ is soluble.
%    \item By Theorem \ref{thm:unity-1}, $M$ is the splitting field for $t^p-1$ over $K$. Then the extension $M / K$ is normal. By Theorem \ref{thm:radical-1}, $\Gal(M / K)$ is abelian and hence soluble.
%    \item  Apply Theorem \ref{thm:correspondence-quotient} to $L / K$ to obtain
%    $$
%    \Gal(M / K) \cong \Gal(L / K) / \Gal(L / M). 
%    $$
%    Theorem \ref{thm:soluble-main} shows that $\Gal(L / K)$ is soluble, completing the induction step.
%\end{itemize}
%\end{proof}
%
%
%\begin{definition}
%	A \textit{normal closure} of a field extension $L / K$ is an extension $N$ of $L$ such that 
%	\begin{enumerate}
%		\item $N / K$ is normal;
%		\item If $L \subseteq M \subseteq N$ and $M / K$ is normal, then $M = N$.
%	\end{enumerate}
%\end{definition}
%
%\begin{theorem} \label{thm:radical-closure}
%    If $L / K$ is a radical extension in $\mathbb{C}$ and $M$ is the normal closure of $L / K$, then $M / K$ is radical.
%\end{theorem}
%
%\begin{proof}
%\TODO paraphrase
%
%Let $L=K\left(\alpha_1, \ldots, \alpha_m\right)$ with $\alpha_i^{n_i} \in K\left(\alpha_1, \ldots, \alpha_{i-1}\right)$. Let $f_i$ be the minimal polynomial of $\alpha_i$ over $K$. Then $M \supseteq L$ is clearly the splitting field of $\prod_{i=1}^m f_i$. For every root $\beta_{i j}$, by Theorem \ref{thm:automorphism-from-zeros}, there exists a $K$-automorphism $\tau$ of $M$ such that $\tau(\alpha_i) = \beta_{ij}$. Since $\alpha_i$ is a member of a radical sequence for a subfield of $M$, so is $\beta_{i j}$. Combining the sequences yields a radical sequence for $M$.
%\end{proof}
%
%
%\begin{definition}
%    Let $f$ be a polynomial over a subfield $K$ of $\mathbb{C}$, with splitting field $\Sigma$ over $K$. Then $f$ is \textit{soluble by radicals} if there exists a field $M \supseteq \Sigma$ such that $M / K$ is a radical extension.  
%\end{definition}
%
%
%\begin{theorem} \label{thm:radical-galois-soluble}
%    Let $f$ be a polynomial over a subfield $K$ of $\mathbb{C}$. If $f$ is soluble by radicals, then $\Gal(f)$ is soluble.
%\end{theorem}
%
%\begin{proof}
%\TODO paraphrase
%
%Since $f$ is soluble by radicals, its splitting field $\Sigma$ over $K$ satisfies $K \subseteq \Sigma \subseteq M$ where $M / K$ is a radical extension. Let $K_0$ be the fixed field of $\Gal(\Sigma / K)$, and let $N / M$ be the normal closure of $M / K_0$. Then
%$$
%K \subseteq K_0 \subseteq \Sigma \subseteq M \subseteq N.
%$$
%Since $M / K_0$ is radical, Theorem \ref{thm:radical-closure} implies that $N / K_0$ is a normal radical extension. By Theorem \ref{thm:radical-3}, $\Gal\left(N / K_0\right)$ is soluble.
%By Theorem \ref{thm:fix-extension-normal}, the extension $\Sigma / K_0$ is normal. By Theorem \ref{thm:correspondence-quotient}
%$$
%\Gal\left(\Sigma / K_0\right) \cong \Gal\left(N / K_0\right) / \Gal(N / \Sigma).
%$$
%Theorem \ref{thm:soluble-main} implies that $\Gal\left(\Sigma / K_0\right)$ is soluble. But $\Gal(\Sigma / K)=\Gal\left(\Sigma / K_0\right)$, so $\Gal(\Sigma / K)$ is soluble.
%
%\end{proof}
%
%The converse also holds, but we do not prove it here.

%
%\subsection{Abel-Ruffini Theorem}
%
%
%
%To begin we define the characterisitic of a field.
%
%\begin{definition}
%    The characteristic of a field $K$, denoted by Char$(K)$, is the smallest number $n \in \N$ such that $1+1+...+1 (n$ times)$=0$. If such an $n$ does not exist, we say that Char$(K)=0$
%\end{definition}
%
%The Abel-Ruffini theorem, posed and proved (incompletely) by Paolo Ruffini and completed by Niels Henrik Abel, shows that the general polynomial of order five and above is not solvable by radicals. For this section, we assume that all fields that we mention have characteristic 0.
%
%First, we need to understand a lemma in order to be able to prove the Abel-Ruffini Theorem. 
%
%
%Next we need to show that a simple radical extension is solvable, in order to be able to fully prove Galois' Theorem.
%
%\begin{lemma}\label{lemma:solvable-radical-extension}
%    A simple radical extension is solvable.
%\end{lemma}
%
%\begin{proof}
%
%    \textcolor{blue}{Let $a$ be a root of the polynomial $f(x)=x^n-c$ and let $\omega$ be an n-th primitive root of unity. Then we have that all the roots of $f(x)$ are of the form $a \cdot \omega^i$ where $0\leq i \leq n-1$ $\Gal(K/k)$}
%	    
%	(\TODO Why is it of the form $K(\sqrt{a})$? It can be $K(e^{2\pi i / n}\sqrt[n]{a})$ in general.)
%	
%    Let $K(\sqrt{a})$ be a simple radical extension, where $K$ is a field and $a$ is not a square in $K$.
%
%    Since $K$ is a field, $K(a)$ is an algebraic extension of $K$ and therefore $K(a)/K$ is a simple extension.
%    
%    \noindent
%    Now let $b=\sqrt{a}$, so therefore $b$ is a root of the polynomial $f(x)=x^2-a$, and this is irreducible over $K$, and $K(b)/K = K(\sqrt{a})/K$ is also a simple extension. Thus we have $[K(b):K]=2$, and since the extension is simple $\Gal(K(b)/K)$ is isomorphic to $\Z/2\Z$. Since $\Z/2\Z$ is abelian, it follows that $K(\sqrt{a})$ is solvable.
%\end{proof}
%
%\noindent
%
%
