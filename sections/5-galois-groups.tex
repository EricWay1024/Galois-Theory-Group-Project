\section{Galois Groups and Galois Extensions} \label{sec:galois-extensions}

Given a field $L$, we now look at subgroups of the automorphism group $\Aut(L)$ and the elements in $L$ such a subgroup fixes. We investigate the matter with two approaches. On the one hand, we might choose a subfield $K$ of $L$ and define $G$ to be the group of all $K$-automorphisms of $L$, so that $G$ is guaranteed to fix any element in $K$, and we say $G$ is the \textit{Galois group} of the field extension $L/K$, denoted as $G = \Gal(L/K)$. We also define the Galois group of a polynomial $f$ over $K$ with splitting field $F$ as the Galois group of $F / K$. This group acts on the roots of $f$ in a natural way, as we shall see. On the other, we might choose any $G$, and define $K$ to be all elements fixed by $G$, denoted as $K = \Fix(G)$, in which case $K$ is always a field and is called the \textit{fixed field} of $G$. In general, the two approaches are not inverses of each other, i.e. $\Fix(\Gal(L/K)) \neq K$. In fact, the condition for the Galois correspondence $\Fix(\Gal(L/K)) = K$ is when $L/K$ is a finite ``Galois'' extension (defined as both normal and separable), as we will prove. 


\subsection{Galois Groups and Fixed Fields}
%For a field extension $L/K$, we now focus on all $K$-automorphisms of $L$ (see Definition \ref{def:automorphism}), which form the Galois group of the extension:

We now define the Galois group of a field extension and a polynomial, respectively.
\begin{definition}
    Let $L/K$ be a field extension. Then all $K$-automorphisms of $L$ form a group under map composition, and this group is called the \textit{Galois group} of $L$, written as \(\Gal(L/K)\).
	Let $f$ be a polynomial over $K$ and let $F$ be a splitting field of $f$, then $\Gal(F/K)$ is the \textit{Galois group} of $f$, written as \(\Gal(f)\).
	
%		 In the case that \(F\) is a splitting field of a polynomial \(f\) over $K$, \(G\) is called the Galois group of \(f\),
\end{definition}

The next theorem shows that an element in the Galois group of a finitely generated field extension is completely determined by how it maps each element in the generating set \cite{morandi_field_1996}. 

\begin{theorem} \label{thm:galois-group-determined-by-generator}
	Let $L/K$ be a field extension and $L = K(\alpha_1, \ldots, \alpha_n)$. Then if $\sigma, \tau \in \Gal(L/K)$ and $\sigma(\alpha_i) = \tau(\alpha_i)$ for all $i = 1, \dots, n$, then $\sigma = \tau$. 
\end{theorem}

\begin{proof}
	Each element $a \in L$ can be represented as $ f(\alpha_1, \dots, \alpha_n) / g(\alpha_1, \dots, \alpha_n)$, where $f, g \in K[x_1, \ldots, x_n]$. Since $\sigma$ and $\tau$ preserve addition and multiplication, fix all elements in $K$, and act in the same way on $\alpha_i$, it is clear that
	\begin{equation*}
		\begin{split}
		\sigma(a) 
			&= \sigma(f(\alpha_1, \dots, \alpha_n) / g(\alpha_1, \dots, \alpha_n)) 
			= f(\sigma(\alpha_1), \dots, \sigma(\alpha_n)) / g(\sigma(\alpha_1), \dots, \sigma(\alpha_n)) \\
			&= f(\tau(\alpha_1), \dots, \tau(\alpha_n)) / g(\tau(\alpha_1), \dots, \tau(\alpha_n))  
			=  \tau(f(\alpha_1, \dots, \alpha_n) / g(\alpha_1, \dots, \alpha_n)) \\
			&= \tau(a).\\ 
		\end{split}
	\end{equation*}
Thus $\sigma = \tau$.
\end{proof}





%The Galois group becomes increasingly useful, especially when dealing with polynomials, as we will see with the following theorem.

We now show that a $K$-monomorphism always maps a root of a polynomial over $K$ to another, in a more general setting than that of a Galois group which consists of $K$-automorphisms of a larger field $L$ \cite{morandi_field_1996}. 

\begin{theorem} \label{thm:galois-group-permutes-zeros}
Given fields $K \subseteq M \subseteq L$. Let $f$ be a polynomial over $K$ and let $\alpha \in M$ such that $f(\alpha) = 0$. Let $\sigma : M \to L$ be a $K$-monomorphism. Then $f(\sigma(\alpha)) = 0$. Further, $\sigma(\alpha)$ has the same minimal polynomial as $\alpha$ over $K$. 

%Let $\phi \in \Gal(L/K)$ where $L/K$ is a field extension. Then let $f$ be a polynomial over $K$ with a zero $a \in L$, then $\phi(a)$ is also a zero of $f$.
\end{theorem}

\begin{proof}
	Let $f(t) = a_0 + a_1 t + \dots a_n t^n$, where $a_i \in K$. Then $$f(\sigma(\alpha)) = \sum_{i=0}^n a_i \left( \sigma (\alpha) \right)^i
	= \sum_{i=0}^n \sigma(a_i)  \sigma (\alpha ^ i) 
	= \sigma(f(\alpha)) = \sigma(0) = 0. $$
	In particular, if $f$ is the minimal polynomial of $\alpha$ over $K$, then $\sigma(\alpha)$ is another root of the irreducible polynomial $f$, and hence $f$ is also the minimal polynomial of $\sigma(\alpha)$ over $K$. 
\end{proof}



For a polynomial $f$ over a field $K$, Theorem \ref{thm:galois-group-determined-by-generator} indicates that $\sigma \in \Gal(f)$ is completely determined by how it maps the roots of $f$, and Theorem \ref{thm:galois-group-permutes-zeros} indicates $\sigma \in \Gal(f)$ always maps a root of $f$ to another root.

\begin{example}
	Let $\sigma \in \Gal(\C/\R)$. Since $\C= \R(i)$, by Theorem \ref{thm:galois-group-determined-by-generator}, $\sigma$ is determined by $\sigma(i)$. But $i$ has minimal polynomial $t^2 + 1$ over $\R$, and $\sigma(i)$ has the same minimal polynomial over $\R$ by Theorem \ref{thm:galois-group-permutes-zeros}. Hence $\sigma(i) = \pm i$. When $\sigma(i) = i$, we have $\sigma = \Id$; when $\sigma(i) = -i$, $\sigma$ is the complex conjugation. Thus $\Gal(\C / \R) = \{\Id, z \mapsto \bar z \} \cong \Z_2$. 
\end{example}

It is easy to see that the Galois group of a polynomial naturally leads to a group action on its roots. The next theorem is inspired by \cite{visual-algebra}. 


\begin{theorem} \label{thm:galois-group-acts-on-zeros}
	Let $F$ be a splitting field of a polynomial $f$ over $K$ and let $G = \Gal(f)$. Let $X$ be the set of roots of $f$. Then $G$ acts on $X$ by restriction on $X$. 
\end{theorem}

\begin{proof}
	Take any $\sigma \in G$. Then clearly $\alpha \in X$ if and only if $\sigma(\alpha) \in X$. Restricting $\sigma$ on $X$ gives a bijection $\sigma | _X : X \to X$, which is a permutation of $X$ by Definition \ref{def:permutation}. Then $T_\sigma  = \sigma | _X$ forms a $G$-action on $X$ by Definition \ref{def:action}.  
\end{proof}


\begin{corollary} \label{thm:galois-group-isomorphic-symmetric-subgroup}
	$\Gal(f)$ is isomorphic to a subgroup of $S_n$, where $n$ is the degree of $f$. 
\end{corollary}

\begin{proof}
	Let $X$ be the set of roots of $f$. Then $|X| \le n$ and $S_X$ is a subgroup of $S_n$. The homomorphism from $\Gal(f)$ to $S_X$ is also a homomorphism from $\Gal(f)$ to $S_n$. Further, the kernel of the homomorphism is trivial, as if $\sigma \in \Gal(f)$ fixes all elements in $X$, it is the identity map by Theorem \ref{thm:galois-group-determined-by-generator}. Therefore, by Theorem \ref{thm:first-iso}, $\Gal(f)$ is isomorphic to its image in $S_n$, which is a subgroup of $S_n$. 
\end{proof}

Recall Definition \ref{def:transitive-action} for transitive group actions. The next theorem, adapted from \cite{galois-permutation}, states that the irreducibility of a separable polynomial is equivalent to the transitivity of its Galois group action.

\begin{theorem} \label{thm:galois-action-transitive-irreducible}
	Let $f$ be a separable polynomial over field $K$. Let $X$ be the set of the roots of $f$. Then $\Gal(f)$ acts transitively on $X$ if and only if $f$ is irreducible. 
\end{theorem}

\begin{proof}
	Suppose $f$ is irreducible. Then for any roots $\alpha, \beta$ of $f$, by Theorem \ref{thm:automorphism-from-zeros}, there exists $\sigma \in \Gal(f)$ such that $\sigma(\alpha) = \beta$. Thus $\Gal(f)$ acts transitively on $X$ by Definition \ref{def:transitive-action}. 
	
	Now suppose $f$ is reducible. Then $f$ is a product of distinct irreducible polynomials, since $f$ is separable. Let $\alpha_1$ and $\alpha_2$ be roots of two different irreducible factors $p_1$ and $p_2$ of $f$, respectively. Then $p_i$ is the minimal polynomials of $\alpha_i$, for $i = 1, 2$. For any $\sigma \in \Gal(f)$, $\sigma(\alpha_1)$ has the same minimal polynomial as $\alpha_1$ by Theorem \ref{thm:galois-group-permutes-zeros}, which is $p_1$ and is not $p_2$. Thus $\sigma(\alpha_1) \neq \alpha_2$ and thus $\Gal(f)$ does not act transitively on $X$. 
\end{proof}

\begin{example} \label{exm:galois-group}
	Let $f(t) = (t^2 + 1)(t^2 - 2)$ be a polynomial over $\Q$. The set of roots of $f$ is $X = \{i, -i, \sqrt 2, -\sqrt 2\}$. The splitting field is $/F = \Q(\sqrt 2, i) = \{p+\sqrt 2 q + i r + \sqrt 2 i s : p, q, r, s \in \Q\}$ and $[F : \Q] = 4$. 
	
	Let us determine $\Gal(f)$. Let $\sigma \in \Gal(f)$, then $\sigma$ is completely determined by $\sigma(\sqrt 2)$ and $\sigma(i)$. Since $\sigma(\sqrt 2)$ has the same minimal polynomial over $\Q$ as $\sqrt 2$, which is $t^2 - 2$, we see that $\sigma(\sqrt 2) = \pm \sqrt 2$. Similarly, $\sigma(i) = \pm i$. Define $\phi, \psi \in \Gal(f)$ such that 
	$$
		\phi(\sqrt 2) = -\sqrt 2, \, \phi(i) = i; \quad
		\psi(\sqrt 2) = \sqrt 2, \, \psi (i) = -i. 
	$$
	For example, we can explicitly write $\phi(p + \sqrt 2 q + i r + \sqrt 2 i s) = p - \sqrt 2 q + i r - \sqrt 2 i s$ where $p,q,r,s \in \Q$. Then $\phi ^ 2 = \psi ^ 2 = \Id$. Also, $\phi\psi  = \psi \phi \in \Gal(f)$ and $$\phi\psi(\sqrt 2) = -\sqrt 2, \, \phi\psi(i) = -i. $$
%	as $\phi(p +  \sqrt 2 q + i r + \sqrt 2 i s) = p - \sqrt 2 q + ir - \sqrt 2 i s $ and $\psi(p +  \sqrt 2 q + i r + \sqrt 2 i s) = p + \sqrt 2 q - ir  - \sqrt 2 i s $, where $p,q,r,s \in \Q$, 
	Then $\Gal(f) = \{\Id, \phi, \psi, \phi \psi \} \cong \Z_2 \times \Z_2$. If we label $i, -i, \sqrt 2, -\sqrt 2$ as $x_1, x_2, x_3, x_4$, then $\phi$ and $\psi$ correspond to transpositions $(34)$ and $(12)$, respectively, and $\Gal(f) \cong \{\Id, (34), (12), (12)(34)\} \le S_4$. Also, $\Gal(f)$ does not act transitively on $X$, as there is no $\sigma \in \Gal(f)$ such that $\sigma(\sqrt 2) = i$, and clearly $f$ is reducible over $\Q$.
\end{example}

We now prove that the upper bound for the order of the Galois group of a finite field extension is the degree of the extension, formalizing ideas from  \cite{galois-theory-lectures}. As we shall see in the next subsection, this upper bound is reached exactly when $L/K$ is a Galois extension. 

\begin{theorem} \label{thm:galois-group-order-upper-bound}
	Let $L/K$ be a finite field extension and let $G = \Gal(L/K)$, then $|G| \le [L:K]$. 
\end{theorem}

\begin{proof}
	Consider a $K$-monomorphism $\phi_0 : K \to X$ where $K \subseteq X$. Since $L/K$ is finite, by Theorem \ref{thm:finite-equi-def}, we can write $L = K(\alpha_1, \ldots, \alpha_r)$ for $r$ finite and $\alpha_i$ algebraic over $K$. Denote $K_0 = K$ and
	$K_i = K(\alpha_1, \dots, \alpha_i)$ for $i = 1, \ldots, r$. In particular, $K_r = L$. 
	
	
	For $i = 1, \ldots, r$, consider the extension $K_i / K_{i-1}$, where $K_i = K_{i-1} (\alpha_i)$. Let $\alpha_i$ be a root of an irreducible polynomial $f_i$ over $K$ of degree $\partial f_i = [K_i : K_{i-1}]$ by Theorem \ref{thm:degree-theorem}. $f_i$ is over $K$ and hence is fixed under $\phi_i$. We now extend $K$-monomorphism $\phi_{i-1} : K_{i-1} \to X$ to $K$-monomorphism $\phi_i: K_i \to X$, then $\phi_i$ must map $\alpha_i$ to a root of $f_i$ in $X$ by Theorem \ref{thm:galois-group-permutes-zeros}, and the number of roots of $f_i$ in $X$ is \textit{at most} $\partial f_i$. Hence there are at most $[K_i : K_{i-1}]$ ways to extend $\phi_{i-1}$ to $\phi_i$. We now see that there are at most $\prod_{i=1} ^r [K_i : K_{i-1}] = [K_r : K_0] = [L : K]$ ways to extend $\phi_0$ to $\phi_r$ by Thorem \ref{thm:tower-theorem}. Since $\phi_r : L \to X$, if we let $X = L$, then $\phi_r \in \Gal(L/K)$. Also, there is only one way to construct $\phi_0$, which is the identity map. Thus each way of extending $\phi_0$ to $\phi_r$ corresponds to an element in $\Gal(L/K)$, and thus $\Gal(L/K) \le [L:K]$. 
\end{proof}



We now turn to the idea of fixed fields.
\begin{definition}
    Let $\Aut(L)$ be the group of automorphisms of a field $L$ and let $G \leq \Aut(L)$. Then the fixed field of $G$ is the set $\Fix(G) = \{ \alpha \in L : \sigma(\alpha) = \alpha \quad \forall \sigma \in G \}. $ The fixed field is always a field (proof omitted) \cite{morandi_field_1996}.
\end{definition}

In particular, for a finite field extension $L/ K$, the Galois group $G = \Gal(L / K)$ is a subgroup of $\Aut(L)$ such that $K \subseteq \Fix(G) \subseteq L. $ We are mainly interested in cases where $K = \Fix(G) = \Fix(\Gal(L/K))$. This is true exactly when $L/K$ is a Galois extension, as we shall see in the next subsection. To show this is not true in general, we now look at a counterexample. 

\begin{example}
    For $G = \Gal(L/K)$, it is not always the case that $K = \Fix(G)$. For example, consider when $K = \Q$ and $L = \Q (\alpha)$ where $\alpha = \sqrt[3]{2}$. If $\sigma \in G$, then $\sigma(\alpha)$ is a root of the polynomial $t^3 - 2$ by Theorem \ref{thm:galois-group-permutes-zeros}, but the two complex roots are not contained in $L$. Thus $\sigma(\alpha) = \alpha$ and $G = \{ \Id \}$. We see that $\Fix(G) = L \neq K$. 
\end{example}



% Before ending our discussion on fixed fields, we prove the next theorem to show that fix fields naturally lead to a normal field extension. This proof is original.
