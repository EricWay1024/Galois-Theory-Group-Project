\section{Galois Extensions and Galois Correspondence}

\subsection{Galois Extension}


We are now ready to present a special kind of field extensions:

\begin{definition}
    A finite field extension is \textit{Galois} if it is both normal and separable. 
\end{definition}

By Theorem \ref{thm:separable-extension-in-C}, for a finite field extension in $\C$, being ``Galois'' is a synonym of being ``normal'', as its separability is automatic. 

We now give some further properties for Galois extensions in general fields. The statements in the following theorem should not seem surprising, as we have mentioned (ii) and (iii) in the previous section and (iv) is apparently based on the definitions of normal and separable extensions. 


\begin{theorem} \label{thm:fixed}
	Let $L/K$ be a finite field extension, and let $G = \Gal(L/K)$. Then the following statements are equivalent:
	\begin{enumerate}[label=(\roman*)]
		\item $L/K$ is a Galois extension;
		\item $[L:K] = |G|$;
		\item $K = \Fix(G)$;
		\item $L$ is a splitting field of a separable polynomial over $K$;
		%        \item There is a natural bijection between the intermediate fields $M$ such that $K \subseteq M \subseteq L$ and the subgroups $H \le G$, where $\alpha$ and $\beta$ defined by $\alpha(M) =  \Gal (L/M)$ and $\beta(H) =  \Fix(H)$ are inverses of each other.
	\end{enumerate}
	% Given a finite Galois extension $L/K$ and its Galois group $G=\Gal(L/K)$, we have $K = \Fix(G)$.
\end{theorem}

%The proof has two parts: the first part establishes the equivalence of the first four statements and the second part works on the equivalence of the fifth and the rest. 

%\begin{proof}[Proof, first part]
\begin{proof}
	
	%	We now prove the equivalence of the first four statements. We will work on the fifth statement separately shortly afterwards.
	
	
	(i) $\Rightarrow$ (ii). Consider a $K$-monomorphism $\phi: K \to \overline{K}$. We now extend $\phi$ to a $K$-monomorphism $\phi':L \to \overline{K}$ in the same pattern as in the proof of Theorem \ref{thm:galois-group-order-upper-bound}. However, at each step, we claim that the number of ways to extend is equal to $[K_i : K_{i-1}]$. Indeed, $\alpha_i$ is separable and hence $f_i$ has exactly $[K_i : K_{i-1}]$ distinct zeros, and $\overline K$ contains all these zeros. Therefore, combining all the steps gives us in total $[L:K]$ ways to extend $\phi$ to $\phi'$. Also, since $L/K$ is normal, by Theorem \ref{thm:normal-equiv-def}, each $K$-monomorphism $\phi': L \to \overline{K}$ is a $K$-automorphism of $L$, and thus each $\phi'$ corresponds to an element in $\Gal(L/K)$. Hence $[L:K] = \Gal(L/K)$. 
	
	(ii) $\Rightarrow$ (iii). Define $F:= \Fix(G)$. We have $K \subseteq F \subseteq G$. Then by assumption, we have $[L:K] = |G|$; each $K$-automorphism of $L$ in $G$ is also a $F$-automorphism of $L$ in $\Gal(L/F)$, and thus $G \subseteq \Gal(L / F)$ and $|G| \le |\Gal(L / F)|$; by Theorem \ref{thm:galois-group-order-upper-bound}, $|\Gal(L/F)| \le [L:F]$; by Theorem \ref{thm:tower-theorem}, we see that $ [L:F] \le [L:F][F:K] = [L:K]$. Combining them gives $$
	[L:K] = |G| \le |\Gal(L/F)| \le  [L:F] \le [L:K],
	$$    
	and thus all the $\le$ signs must take equality.  
	Thus $[L:K]=[L:F]$ and therefore by Theorem \ref{thm:tower-theorem}, we see that $[F:K]$ must equal $1$, and thus $F:=\Fix(G) = K$.
	
	(iii) $\Rightarrow$ (iv). Let $\alpha \in L$. Consider $\Orb_G(\alpha) \subseteq L$, the orbit of $\alpha$ under the $G$-action on $L$. Then since $G$ is finite, $\Orb_G(\alpha) = \{ \alpha_1 = \alpha, \alpha_2, \ldots, \alpha_n\}$. Consider the polynomial $f(t) = (t-\alpha_1) \ldots (t-\alpha_n)$ over $L$.  $f$ is separable in $L$, as $\alpha_i$ are distinct. Also, any $\sigma \in G$ acting on $\alpha_1, \dots, \alpha_n$ only permutes the elements, and thus $\sigma$ fixes $f$. Then the coefficients of $f$ must be in $\Fix(G)$, but $\Fix(G) = K$ by assumption. Hence $f$ is over $K$. Clearly, $L$ is a splitting field of $f$ over $K$. 
	
	(iv) $\Rightarrow$ (i) is trivial.
\end{proof}


\begin{corollary} \label{thm:galois-intermediate}
	If $L/K$ is a Galois extension, then for an intermediate field $M$ such that $K \subseteq M \subseteq L$, $L/M$ is also a Galois extension. 
\end{corollary}
\begin{proof}
	By (iv) of Theorem \ref{thm:fixed}, $L$ is the splitting of some polynomial $f$ over $K$, but $f$ is also a polynomial over $M$. Hence $L/M$ is Galois.
\end{proof}

Note that in the above corollary, $L/K$ being a Galois extension does not imply that $M/K$ is Galois. This is in general false; consider for example $K = \Q, M = \Q(\sqrt[3]{2})$ and $L = \Q(\sqrt[3]{2}, \zeta)$, where $\zeta = e^{2\pi i / 3}$. Then $L/K$ and $L/M$ are Galois but $M/K$ is not. 


%\begin{theorem}
%	Let $L$ be a field and let $G \le \Aut(L)$ be finite. Let $K = \Fix(G)$. Then $[L:K] =|G|$. 
%\end{theorem}
%
%\begin{proof}
%	\TODO(10.5)
%\end{proof}



\subsection{Examples of Galois Extensions}

\TODO

\subsection{Fundamental Theorem of Galois Theory}

%\begin{theorem}
%	Let $M$ be a field and let $G \le \Aut(M)$. Then $M / \Fix(G)$ is a Galois extension.
%\end{theorem}
%
%\begin{proof}
%	\TODO
%\end{proof}
%
%\begin{theorem}
%    Let $G$ be a finite subgroup of the group of automorphisms of a field $L$, and let $K = \Fix(G)$. Then $[L : K] = |G|$. 
%\end{theorem}
%
%\begin{proof}
%    \TODO
%\end{proof}

% \noindent We now look at a theorem which helps to characterise and identify fields, extensions and Galois groups.



Group theory tends to look at patterns and symmetries in mathematical objects, and Galois groups have an interesting symmetry. The following theorem describes how the structure of an extension of a field is the same as the subgroups of the Galois groups.

To help us prove the fundamental theorem, we first need to show a relationship between a subgroup of a Galois group and its subgroups

\begin{theorem}\label{thm:equal-subgroup-fixed-points}
    Let $L/F$ be a finite Galois extension, with $G=\Gal(L/F)$, then if we have a subgroup $H\leq G$ with $\Fix(H)=F$ then $H=G$.
\end{theorem}

\begin{proof}
    Suppose $H\leq G$ with $\Fix(H)=F$. Then, let $L = F(a)$, then $H \cdot a$ denotes the orbit under the group action of $H$ in $L$, and we can assume that $H \cdot a = \{a_1,...,a_m\}.$ From this we can see that $|H \cdot a| = m$, and let $[L:F]=n$.
    Then we know that the $m$ cannot be larger than the size of $H$, and since $H\leq G \implies |H|\leq|G|$, we get $|H \cdot a| = m \leq |H| \leq |G| = [L:F] = n$. 
    
    \textcolor{red}{Put a section in polynomials regarding symmetric polynomials and the elementary symmetric polynomials and minimal polynomials.} 
    
    \noindent
    Then by evaluating the elementary symmetric polynomials at each of the elements in $H \cdot a$, we get that these are just the elements in $\Fix(H)=F$. Therefore, we see that the coefficients in the polynomial: $f(X) := (X-a_1)(X-a_2)...(X-a_m)$ lies in $F$, and we have $f(a)=0$. Then, our minimal polynomial $\mu_a(X)$ divides $f(X)$ and $\deg(f) \geq \deg(\mu_a(X)) \implies m \geq n$. Hence we have $m \geq n$ and $n \geq m$, so therefore we have that $m=n$, which tells us $|H|=|G|$ and so $H=G.$
\end{proof}
\TODO for a finite field extension $L/K$, we can always write $L = K(a)$ for some $a \in L$, where $a$ is the primitive element of the extension. 

\begin{theorem}[The Fundamental Theorem of Galois Theory] \label{thm:fundamental-theorem} Given a Galois extension $L/K$ and its Galois group $G = \Gal(L/K)$, there is a natural bijection between subgroups $H\leq G$ and the intermediate fields $M$ such that $K \subseteq M \subseteq L$. Define

\begin{itemize}
    \item $\alpha:M \mapsto \Gal(L/M) \leq G$ and
    \item $\beta:H \mapsto \Fix(H) \subseteq L$,
\end{itemize}
then $\alpha$ and $\beta$ are inverses of each other.
\end{theorem}
\begin{proof}
\begin{enumerate}[label=(\roman*)]

 \item ($\beta \circ \alpha = \Id$) Let $M$ be an intermediate field, and by Theorem \ref{thm:galois-intermediate} we can see that $L/M$ is a Galois Extension. Then if we consider the Galois group $\Gal(L/M)$, we can see that via Theorem \ref{thm:fixed}, $M = \Fix(\Gal(L/M))$. Also, $\alpha(M) = \Gal(L/M)$. Then we look at $(\beta \circ \alpha)(M)$, which is equal to $\beta(\Gal(L/M)) = \Fix(\Gal(L/M))$, but again as see in Theorem \ref{thm:fixed}, we have $\Fix(\Gal(L/M)) = M$. Thus $(\beta \circ \alpha)(M) = M$ and hence we see that $\beta \circ \alpha = \Id$.

 \item ($\alpha \circ \beta = \Id$) Now, let $H$ be a subgroup of $G$ and then set $M :=\Fix(H)$. Then $H$ is a subgroup of the Galois group $\Gal(L/M)$. Since $K \subseteq M \subseteq L$, by Theorem \ref{thm:galois-intermediate} we have that $L/M$ is a Galois extension. Thus, by Theorem \ref{thm:equal-subgroup-fixed-points}, since we have $H\leq \Gal(L/M)$ and we have $\Fix(H)=M$, then we can say that $H=\Gal(L/M)=\Gal(L/\Fix(H))$. Thus if we consider $(\alpha \circ \beta)(H) = \alpha(\Fix(H)) = \Gal(L/\Fix(H))= H $, we have $(\alpha \circ \beta)(H) = H$ and so $\alpha \circ \beta = \Id$.
\end{enumerate}
Hence we can see that the functions $\alpha$ and $\beta$ are inverse to each other.
\end{proof}



\begin{theorem} \label{thm:correspondence-quotient}
    Let $L / K$ be a Galois extension and let $M$ be an intermediate field such that $M /K$ is a Galois extension, then 
    $$\Gal(M / K) \cong \Gal(L / K) / \Gal(L / M). $$
\end{theorem}

\begin{proof}
	% Theoerm 58, Rotman
	Let $G = \Gal(L/K)$ and $G' = \Gal(M/K)$. Define a map $\psi: G \to G'$ as $\psi(\sigma) = \sigma | _M$ for $\sigma \in G$.
	
	We first show that indeed $\psi(\sigma) \in G'$ for all $\sigma \in G$. We only need to prove that $\sigma(M) = M$. $M/K$ is Galois, so by Theorem \ref{thm:fixed}, $M$ is a splitting field for a separable polynomial $f$ over $K$. Let $\alpha_1, \dots, \alpha_n$ be the distinct zeros of $f$, then $M = K(\alpha_1, \ldots, \alpha_n)$. But $\sigma$ permutes $\alpha_i$ by Theorem \ref{thm:galois-group-acts-on-zeros} and $\sigma(K) = K$. Therefore $\sigma(M) = \sigma(K(\alpha_i, \ldots, \alpha_n)) = K(\sigma(\alpha_1), \ldots, \sigma(\alpha_n)) = M. $
	
	
	 $\psi$ is clearly a group homomorphism. It is onto due to the normality of $L/K$ and Theorem \ref{thm:monomorphism-extend-automorphism}: each $K$-automorphism $\tau$ of $M$ is also a $K$-monomorphism from $M$ to $L$ and thus there exists a $K$-automorphism $\sigma$ of $L$ such that $\sigma | _M = \tau$. Also, $\sigma \in \operatorname{Ker}(\psi)$ if and only if $\psi(\sigma) = \sigma |_M = \Id$, which is equivalent to $\sigma \in \Gal(L/M)$. Thus $\operatorname{Ker}(\psi) = \Gal(L/M)$. 
	 
	 Thus by Theorem \ref{thm:first-iso}, 
	$$G / \operatorname{Ker}(\psi) \cong \operatorname{Im}(\psi) = G',$$
	and the required result thus follows. 
\end{proof}
