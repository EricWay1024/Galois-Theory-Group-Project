
\section{Introduction}
%%\textcolor{red}{Here we need to introduce the topics we talk about, the historical context behind Galois Theory and possibly the extensions we look at, as well as talking about the extensions}
%%\textcolor{blue}{The basis behind Galois Theory was an open question in mathematics until the start of the 1800s, and the theory originated and was developed due to the following question: ``Does there exist a general solution to a polynomial equation of order five?"  For equations of orders less than or equal to four, the Babylonians had already found a solution to this. We first look at fields and extensions of fields in order to have a basis.}

As studied in secondary schools, it is common knowledge that a polynomial of the form $ax^2+bx+c$ has a general formula to find the roots, namely the quadratic formula. Similar formulae also exist to find the roots in cubic and quartic equations. One may wonder whether there is a formula to find the roots of a polynomial of any degree $n \in \N$.

In fact, this was an open mathematical problem until the start of the 19th century, when Niels Abel refined the theory of Paolo Ruffini and showed that a general polynomial of degree five does not have a solution. Evariste Galois had actually found a condition to when a polynomial could or could not be solved, which was the basis of Galois Theory. He looked at whether the permutation group of the roots of a polynomial (known as the Galois Group) had specific properties, which in turn led to the proof that the general polynomial of degree five and above do not have a general solution.

In this report we begin by discussing polynomials over an integral domain to build up knowledge of the properties of polynomials in certain fields in Section \ref{sec:poly}. We reintroduce some key topics, such as Eisenstein’s Criterion, to have the tools to be able to solve some key problems later in the report. We look at roots and irreducibility over fields in order to show some key results.

Working on fields with characteristic $0$, we then look at the concept of \textit{field extensions} and the key properties of them in Section \ref{sec:ext}. A key result in this section, the Tower Theorem, helps to solve many more advanced problems further down the line. We then categorise field extensions as \textit{simple} or \textit{algebraic}, as well as stating and proving key properties of different types of field extensions. In Section \ref{sec:normal}, we further our exploration into \textit{normal}, \textit{separable} and \textit{splitting fields}, which helps us to prove some key results regarding polynomials over a field.

After building up a repertoire of knowledge of fields, we move on to the key point of this report, \textit{Galois groups}, in Section \ref{sec:galois-extensions}. We give the definition of a Galois group, before applying this to polynomials, demonstrating properties of the Galois group of a polynomial as well as looking at examples of this. We also look at \textit{fixed fields}.  We then use the knowledge to prove the Fundamental Theorem of Galois Theory, which helps to reduce problems in field theory to much simpler ones in group theory. We conclude Section \ref{sec:galois-extensions} with a proof of the Fundamental Theorem of Algebra.

In Section \ref{sec:galois-groups-and-polynomials}, we then develop the concept of \textit{solvability} of a group, as well as the \textit{solvability} of a polynomial over a field, looking at Galois’ condition for a polynomial to be solvable. In this section it is shown that symmetric groups in general are not solvable. This then leads us to solve the open problem in mathematics in the early 1800s, why is there no solution to a general polynomial of degree five and higher.

We finish our report by looking at an application of what we have discussed, with compass and straight-line constructions in Section \ref{sec:applications}. We start from the basics before integrating fields and field extensions in to our definitions, before proving solving problems posed by the ancient Greeks – such as ‘Is it possible to construct a square with the same area as a circle?’ The problem can be solved using fields and we explore the theorems.
