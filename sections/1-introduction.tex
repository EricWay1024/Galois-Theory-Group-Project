\textit{Abstract: In this paper we introduce the basics of Field Theory which lead up to field extensions and Galois Groups. We use these to look at the solvablity of polynomials over different fields, and give conditions to classify whether certain polynomials are solvable. We use this to look at constructable numbers and shapes.}

\section{Introduction}
%%\textcolor{red}{Here we need to introduce the topics we talk about, the historical context behind Galois Theory and possibly the extensions we look at, as well as talking about the extensions}
%%\textcolor{blue}{The basis behind Galois Theory was an open question in mathematics until the start of the 1800s, and the theory originated and was developed due to the following question: ``Does there exist a general solution to a polynomial equation of order five?"  For equations of orders less than or equal to four, the Babylonians had already found a solution to this. We first look at fields and extensions of fields in order to have a basis.}

In this report, we will talk about the basis behind Galois Theory as well as the historical context behind Galois Theory. We will start off with an introduction to polynomials and set out some basic theorems and proofs of polynomials. This introduction will include theorems such as Eisenstein's Criterion, The Fundamental Theorem of Algebra and Gauss Lemma. We then look at the basics of field extensions such as the Tower Theorem followed by a deeper look at normal and separable extensions. After this, we start to look into Galois Groups, which are denoted $Gal(f)$, and later Galois Extensions and Galois Correspondence. This section has the Fundamental Theorem of Galois Theory in it. Next we look further into Galois Groups and Polynomials. This involves looking at Solvable Groups and solvability as well as later looking at the Galois Theorem and then the Abel-Ruffini Theorem. We finish off this report by looking at the applications of Galois Theory and Galois Groups. In the applications, we look at squaring the circle as well as other impossibilty proofs.

Everyone knows the equation to find the two roots of a polynomial of order 2. This is taught in secondary schools all around the world. A polynomial of order 1 is even easier to find the root of by just rearranging the polynomial. Some that have studied maths to a higher level even know the equation to find all 3 roots of a polynomial with order 3. There even exists an equation to find all 4 roots of every polynomial of order 4. The Babylonians found all of these equations. The next question is 'Is there a general equation to find the roots of all polynomials with an order of at least 5?' This is where Galois Theory comes in. The basis behind Galois Theory was an open question in mathematics until the start of the 1800s.

Evariste Galois was born on the 25 October 1811. He would go onto to lay the foundations for Galois Theory and Group Theory. Galois was one of the founders of modern algebra and he had to invent idea of a group. This was very abstract at time and many famous mathematicians refused to believe it at first. Galois started to take an interest in mathematics at the age of 14 and in 1828, he attempted to get into Ecole Polytechnique, a very prestigious institution at the time, but failed the oral exam. This then led him to enter Ecole Normale, a much less prestigious institution. His first ever paper was on continued fractions which was published in 1829. A couple of days after Galois's father commited suicide, Galois tried for a second time to get into Ecole Polytechnique. He failed again despite being more than qualified. It is believed that he failed because he made too many logical leaps and confused the examiner, which then angered Galois. His behavior may have also been worse than usual as a result of his fathers recent death. Galios was expelled from Ecole Normale in January 1831 as a result of a letter heavily criticising his school's director for restricting him in political actions. After this, he split his time between his mathematical work and politics. He spent much of his life focused on politics instead of maths and died in duel in 1832 aged just 20 after getting into a disagreement. His main paper left unpublished until 1846 when Joseph Liouville published his paper with explainations for parts of Galois's paper. Galois superseded the work of Abel and Ruffini and the Abel-Ruffini Theorem.