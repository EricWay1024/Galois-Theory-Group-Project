\section{Introduction}
\textcolor{red}{Here we need to introduce the topics we talk about, the historical context behind Galois Theory and possibly the extensions we look at, as well as talking about the extensions}
\textcolor{blue}{The basis behind Galois Theory was an open question in mathematics until the start of the 1800s, and the theory originated and was developed due to the following question: ``Does there exist a general solution to a polynomial equation of order five?"  For equations of orders less than or equal to four, the Babylonians had already found a solution to this.
We first look at fields and extensions of fields in order to have a basis.}

In this report, we will talk about the basis behind Galois Theory as well as the historical context behind Galois Theory.




Notes:

Galois started to take an interest in mathematics at the age of 14 and in 1828, he attempted to get into Ecole Polytechnique, a very prestigious institution at the time, but failed the oral exam. This then led him to enter Ecole Normale, a much less prestigious institution.

His work laid the foundations for group theory and Galois Theory, which was named after him.

Galois was one of the founders of modern algebra and he had to invent idea of a group. This was very abstract at time and many famous mathematicians refused to believe it at first.

Galois died in duel in 1832 aged just 20 after getting into a disagreement.

His first ever paper was on continued fractions which was published in 1829. 

Cauchy recognised the importance of Galois's work

A couple of days after Galois's father commited suicide, Galois tried for a second time to get into Ecole Polytechnique. He failed again despite being more than qualified. It is believed that he failed because he made too many logical leaps and confused the examiner, which then angered Galois. His behavior may have also been worse than usual as a result of his fathers recent death.

Galios was expelled from Ecole Normale in January 1831 as a result of a letter heavily criticising his school's director for restricting him in political actions. After this, he split his time between his mathematical work and politics.

He spent much of his life focused on politics instead of maths. 

Main paper left unpublished until 1846-Joseph Liouville explained things

Superseded work of Abel-Ruffini