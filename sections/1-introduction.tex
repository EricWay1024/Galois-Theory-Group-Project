\section{Introduction}
\textcolor{red}{Here we need to introduce the topics we talk about, the historical context behind Galois Theory and possibly the extensions we look at, as well as talking about the extensions}
\textcolor{blue}{The basis behind Galois Theory was an open question in mathematics until the start of the 1800s, and the theory originated and was developed due to the following question: ``Does there exist a general solution to a polynomial equation of order five?"  For equations of orders less than or equal to four, the Babylonians had already found a solution to this.
We first look at fields and extensions of fields in order to have a basis.}

In this report, we will talk about the basis behind Galois Theory as well as the historical context behind Galois Theory.

\section*{Definitions We Will Need}
\begin{itemize}
    \item Field - Let K be a field. This means that K is a commutative ring with 1 st every element in $K \backslash \{0\}$ has a multiplicative inverse.
    \item Tower of Fields
    \item Automorphism - A group automorphism is an isomorphism from a group to itself
    \item Monomorphism - A monomorphism is an injective homomorphism of a group $G \rightarrow{} H$
    \item Galois Group - Let $L/K$, and $G$ be the set of automorphisms of $L/K$. then $G$ is a group of transformations $L$, called the Galois Group of $L/K$ denoted $Gal(L/K)$.
    \item Polynomials, Chapter 3, including Eisenstein’s Criterion (p40 Stewart 5e)
    \item Field extensions and simple extensions, Chapters 4 and 5, including Corollary 5.13 Stewart and ``degree theorem'' (if $f$ is the minimal polynomial of $\alpha$ over $K$ then the degree of $f$ equals $[K(\alpha) : K]$ ) (Lemma 5.14 Stewart)
    \item normal and separable field extensions (Chapter 9 Stewart), including 9.9, 9.13 and 9.14
    \item Proposition 11.4 Stewart (Galois group as a permutation group of the zeros)
\end{itemize}
