\section{Group Theory Prerequisites}
Here we list some group theory results used in the text without proof. 

\subsection{Isomorphism Theorems}

\begin{theorem}[First Isomorphism Theorem] \label{thm:first-iso}
	Suppose $\phi: G \to H$ is a group homomorphism. Then its kernel $\operatorname{Ker}(\phi)$ is a normal subgroup in $G$, its image $\operatorname{Im}(\phi)$ is a subgroup in $H$, and 
	$$
	G / \operatorname{Ker}(\phi) \cong \operatorname{Im}(\phi).
	$$
\end{theorem}

\begin{theorem}[Second Isomorphism Theorem] \label{thm:second-iso}
	Suppose $H \le G$ and $J \trianglelefteq G$. Then $HJ \le G$, $H \cap J \triangleleft H$ and $$
	(HJ) / J \cong H / (H \cap J). 
	$$
\end{theorem}
\begin{theorem}[Third Isomorphism Theorem] \label{thm:third-iso}
	Suppose $H, J \trianglelefteq G$ and $H \le J$. Then $J/H \trianglelefteq G/H$ and $$
	(G/H)/(J/H) \cong G / J.    $$
\end{theorem}



\subsection{Group Actions}

\begin{definition} \label{def:permutation}
	Given a set $X$, the \textit{permutation group} of $X$ is the group of $\{ (\phi : X \to X) : \phi \text{ is a bijection} \}$ under map composition (clearly this forms a group), denoted as $S_X$. 
\end{definition}

\begin{definition} \label{def:action}
	Given a group $G$ and a set $X$, if there is a homomorphism
	$$
	T: G \rightarrow S_X, \quad g \mapsto T_g,
	$$
	where $T_g(x) \in X$, then the group \textit{acts} on $X$ and $X$ is a $G$-set. 
\end{definition}

\begin{definition}
	For any $x \in X$, the $G$-orbit of $x$ in $X$ is
	$$\operatorname{Orb}_G(x) = \{ y = T_g(x) : g \in G \} \subset X.$$

	
%	\paragraph{Stabiliser} For any $x \in X$, the stabiliser of $x$ in $G$ is 
%	$$\Stab_G(x) = \{ g \in G : T_g(x) = x \} \le G. $$
%	It is a subgroup of $G$.
\end{definition}

\begin{theorem}[Cauchy's Theorem] \label{thm:cauchy}
	Let $p$ be a prime number. Let $G$ be a finite group such that $p$ divides the order of $G$. Then $G$ contains an element of order $p$. 
\end{theorem}

\subsection{Soluble Groups}
This is a restatement and proof of Theorem \ref{thm:soluble-main}.

\begin{theorem} \label{thm:soluble-main-appendix}
	Let $G$ be a group, $H \le G$ and $N \trianglelefteq G$. Then 
	\begin{enumerate}
		\item If $G$ is soluble, then $H$ is soluble;
		\item If $G$ is soluble, then $G / N$ is soluble; 
		\item If $N$ and $G / N$ are soluble, then $G$ is soluble. 
	\end{enumerate}
\end{theorem}
\begin{proof}
	\begin{enumerate}
		\item If $G$ is soluble, then there exists a finite series of subgroups $G_i$ of $G$ for $i = 0, 1, \dots, n$ satisfying Definition \ref{def:soluble}. Let $H_i = G_i \cap H$. Then $H$ has a series of subgroups $H_i$ such that $\{ e \} = H_0 \triangleleft H_1 \triangleleft \dots \triangleleft H_n = H.$
		For each $i = 0, 1, \dots, n - 1$, 
		$$
		\frac{H_{i+1}}{H_i} 
		= \frac{G_{i+1} \cap H}{G_i \cap (G_{i+1} \cap H)}
		\cong \frac{G_i(G_{i+1} \cap H)} {G_i}
		$$
		by Theorem \ref{thm:second-iso}, and ${G_i(G_{i+1} \cap H)}/{G_i}$ is a subgroup of the abelian group $G_{i+1} / G_{i}$. Hence $H_{i+1} / H_{i}$ is abelian and $H$ is soluble.
		\item Take $G_i$ as before. Then $G / N$ has a series
		$N/N = G_0 N / N \triangleleft G_1 N / N \triangleleft \dots \triangleleft G_n N / N  =  G / N. $
		For each $i = 0, 1, \dots, n - 1$, 
		$(G_{i+1} N / N) / (G_{i} N / N) \cong (G_{i+1} N) / (G_i N)$
		by Theorem \ref{thm:third-iso}. Then 
		$$
		\frac{G_{i+1} N}{G_i N} =\frac{G_{i+1}\left(G_i N\right)}{G_i N} \cong \frac{G_{i+1}}{G_{i+1} \cap\left(G_i N\right)} \cong \frac{G_{i+1} / G_i}{\left(G_{i+1} \cap\left(G_i N\right)\right) / G_i},
		$$
		which is a quotient of the abelian group $G_{i+1} / G_i$, so is abelian. Hence $G / N$ is soluble.
		\item There exist two series
		$$
		\begin{aligned}
			\{ e \} & =N_0 \triangleleft N_1 \triangleleft \ldots \triangleleft N_r=N, \\
			N / N & =G_0 / N \triangleleft G_1 / N \triangleleft \ldots \triangleleft G_s / N=G / N
		\end{aligned}
		$$
		with $N_{i+1} / N_{i}$ abelian for each $i = 0, 1, \dots, r-1$ and $(G_{i+1} / N)  / (G_{i} / N) \cong G_{i+1} / G_i $ abelian for each $i = 0,1, \dots, s-1$. Combining them gives the series of subgroups of $G$:
		$$
		\{ e \}=N_0 \triangleleft N_1 \triangleleft \ldots \triangleleft N_r=N=G_0 \triangleleft G_1 \triangleleft \ldots \triangleleft G_s=G .
		$$
		The quotients are either $N_{i+1} / N_i$  or $G_{i+1} / G_i$ and are all abelian. Therefore $G$ is soluble.
	\end{enumerate}
\end{proof}


\begin{definition}
	A group $G$ is \textit{simple} if it is nontrivial and its only normal subgroups are $\{ e \}$ and $G$. 
\end{definition}

\begin{theorem} \label{thm:soluble-and-simple}
	A soluble group is simple if and only if it is a cyclic group of prime order.
\end{theorem}

\begin{proof}
	Let $G$ be a soluble group and suppose $G$ is simple. Consider the series of subgroups of $G$
	$$
	\{ e \}=G_0 \triangleleft G_1 \triangleleft \ldots \triangleleft G_n=G,
	$$
	with abelian quotients, and without loss of generality we assume $G_{i+1} \neq G_i$. Since $G$ is simple, $G_{n-1}$, which is a proper normal subgroup of $G_n = G$, must be $\{ e \}$. Solubility of $G$ gives that $G_n / G_{n -1 }$ is abelian, but $G_n / G_{n - 1} = G$, and thus $G$ is abelian. Thus for any element $g \in G$, the cyclic group $\langle g\rangle$ is a normal subgroup in $G$. Hence  $G = \langle g\rangle$ for any $g \neq e$. Hence $G$ is cyclic of prime order.
	
	The converse is trivial.
\end{proof}


\begin{theorem} \label{thm:simple-alternating}
	The alternating group $A_n$ is simple when $n \ge 5$. 
\end{theorem}

\begin{proof}
    Due to the length and technical nature of the complete proof, only a concise summary is presented here. 
	Suppose that $\{ e \} \neq N \triangleleft A_n$. We can prove that $N$ must contain a $3$-cycle using case-by-case analysis. Next, we can show that if $N$ contains a $3$-cycle, then it contains all $3$-cycles. Since $A_n$ is generated by the $3$-cycles when $n \ge 3$, this means $N = A_n$.
\end{proof}

This is a restatement and proof of Theorem \ref{thm:symmetric-not-soluble}. 

\begin{theorem} \label{thm:symmetric-not-soluble-appendix}
	The symmetric group $S_n$ is not soluble when $n \ge 5$. 
\end{theorem}

\begin{proof}
	Suppose $S_n$ is soluble for $n \ge 5$. Then since $A_n \trianglelefteq S_n$, by Theorem \ref{thm:soluble-main}, $A_n$ is soluble. By Theorem \ref{thm:simple-alternating}, $A_n$ is also simple, so $A_n$ is cyclic of prime order by Theorem \ref{thm:soluble-and-simple}. But $|A_n| = n! / 2$ which obviously is not prime. Contradiction.
\end{proof}


\subsection{Symmetric Groups}
We look at how symmetric groups can be generated by its elements. 

\begin{theorem} \label{thm:symmetric-12-12n}
	The transposition $(12)$ and $n$-cycle $(12 \dots n)$ together generate $S_n$. 
\end{theorem}
\begin{proof}
	Consider the cyclic decomposition of each element in $S_n$, where each cyclic permutation can be written as 
	$
	(a_1a_2\dots a_k) = (a_1 a_k) \dots (a_1 a_3) (a_1 a_2). 
	$
	Note that $(ab) = (1a)(1b)(1a)$, therefore $(12), (13), \ldots (1n)$ together generate $S_n$. Also note that  
	$$(1k)=((k-1)k)\dots(34)(23)(12)(23)(34)\dots((k-1)k),$$
	therefore $(12), (23), \dots, ((n-1)n)$ together generate $S_n$. Finally note that 
	$$
	(k(k+1)) = (12\dots n)^{k-1} (12) (12\dots n)
	$$
	for $k = 2, 3, \dots n - 1$. Therefore $(12)$ and $(12 \dots n)$ together generate $S_n$. 
\end{proof}

\begin{theorem} \label{thm:symmetric-ab-12n}
	For $1 \le a < b \le n$ such that $(b - a, n) = 1$, the transposition $(ab)$ and $n$-cycle $(12 \dots n)$ together generate $S_n$.
\end{theorem}
\begin{proof}
	Let $\sigma=(12 \ldots n)$, so $\sigma^i(a) \equiv a+i \bmod n$. Therefore $\sigma^{b-a}(a) \equiv b \bmod n$, and since $1 \le b \le n$, we have $\sigma^{b-a}(a)=b$. Since $(b-a, n)=1,\langle\sigma\rangle=\left\langle\sigma^{b-a}\right\rangle$ and $\sigma^{b-a}$ is an $n$-cycle sending $a$ to $b$, so $\sigma^{b-a}$ is of the form $(a b \ldots)$. Then
	$
	\langle(a b), \sigma\rangle=\left\langle(a b), \sigma^{b-a}\right\rangle=\langle(a b),(a b \ldots)\rangle .
	$
	Relabel the numbers $1,2 \ldots, n$ so that $(a b)$ turns into $(12)$ and $(a b \ldots)$ into $(12 \ldots n)$. Therefore $\langle(a b), \sigma\rangle=S_n$ by Theorem \ref{thm:symmetric-12-12n}.
\end{proof}

\begin{theorem} \label{thm:symmetric-prime}
	For a prime number $p$, any transposition and $p$-cycle together generate $S_p$.
\end{theorem}
\begin{proof}
	Relabel the numbers so that the $p$-cycle turns into $(12 \dots p)$. Suppose the transposition turns into $(ab)$, where $1 \le a < b \le n$. Since $p$ is prime and $1 \le b - a < p$, we have $(b - a, p) = 1$. The result thus follows from Theorem \ref{thm:symmetric-ab-12n}.
\end{proof}
