\section{Field Extensions} \label{sec:ext}
We assume that all fields in this report has characteristic $0$.  This section is based on books and reports by Stewart \cite[Chapters~4-5]{Stewart}, Moy \cite{Moy} and Morandi \cite[Chapter~1]{ morandi_field_1996}. 

In order to look at field extensions we must first define some useful terms. 
\begin{definition}
Let $K$ be a field. Let \(F\) be a subset of \(K\) such that $F$ is a field. Then $F$ is a \textit{subfield}
 of $K$.
A field \(F\) is called a \textit{prime field} of \(K\) if it has no proper (strictly smaller) subfield.
\end{definition}
\begin{definition}
	A \textit{field homomorphism} from a field $F$ to another field $K$ is a function $\theta: F \to K$ which obeys the following homomorphism properties: $\theta(x + y) = \theta(x) + \theta(y)$, $\theta(x\cdot y) = \theta(x) \cdot \theta(y)$, $\theta(1) = 1$, and $\theta = 0$ for any $x, y \in F$. 
    A \textit{monomorphism} is an injective homomorphism. 
	When a monomorphism is also surjective, it is called a \textit{field isomorphism}. An \textit{automorphism} of a field $F$, is an isomorphism from $F$ to itself.
\end{definition}

An important insight that can be made here is that it can be shown that a field homomorphism is always injective, and hence a field homomorphism is always a field monomorphism.
We can also find that all automorphisms of a field $F$ form a group $\Aut(F)$, the automorphism group of $F$, under composition \cite[p.~15]{morandi_field_1996}.

\begin{definition} \label{def:automorphism}
	Let $K \subseteq M \subseteq L$ be fields. A $K$-automorphism $\phi$ of $L$ is an automorphism of $L$ such that $\phi(a) = a$ for all $a \in K$. 
	A $K$-monomorphism $\phi : M \to L$ is a monomorphism from $M$ to $L$ such that $\phi(a) = a$ for all $a \in K$. 
\end{definition}
%\begin{definition}
    %A group \textit{automorphism} is an isomorphism from a group to itself
%\end{definition}

\begin{definition}
A field \(K\) is said to be a \textit{field extension} of \(F\), denoted \(K / F\), if \(F\) is a subfield of \(K\).  A field extension is a monomorphism \(\iota: F \to K\). If $L/K$ is a field extension, we call any field $M$ with $K \subseteq M \subseteq L$ an \textit{intermediate field.}
\end{definition}
In other words, a field extension \(K\) of \(F\) will contain the field \(F\) and some more elements. We can start to see how this will be useful for finding roots of polynomials over a field as we can extend the field to include the roots.

\begin{definition}
    Let $K$ be a field. Then $K(\alpha_1, \dots, \alpha_n)$ is defined as the smallest field containing $K$ and $\alpha_1, \dots, \alpha_n$ and is said to be obtained from $K$ \textit{adjoining} $\alpha_1, \dots, \alpha_n$. The extension $K(\alpha_1, \dots, \alpha_n) / K$ is called a \textit{finitely generated extension}, and the set $\{\alpha_1, \dots, \alpha_n\}$ is called the \textit{generating set}. 
\end{definition}

\begin{example}
The field \(\mathbb{Q}(\sqrt{2})\) is a field extension of \(\mathbb{Q}\), as \(\mathbb{Q}(\sqrt{2}) = \{a + \sqrt{2}b : a,b \in \mathbb{Q}\}\) we can clearly see that when \(b = 0\) this gives us the rational numbers so \(\mathbb{Q}\) $\subset$ \(\mathbb{Q}(\sqrt{2})\).
\end{example}

Often, a polynomial over a field $K$ has no roots in $K$. For example, $f(t) = t^2 - 2$ over $\Q$ has no roots in $\Q$. However, we can extend the field so that the roots are included. In this case, the field $\Q(\sqrt 2)$ contains all roots of $f$, which are $\pm \sqrt 2$.

\begin{definition}
For a field extension $L/K$, if there exists an \(\alpha \in L\) and \(p(\alpha)=0\) for some polynomial \(p\) over \(K\), then we say that \(\alpha\) is \textit{algebraic} over $K$. If \(\alpha\) is not algebraic, it is called \textit{transcendental} over \(K\). More specifically, if $\alpha \in \C$ is algebraic over $\Q$, then we call $\alpha$ an \textit{algebraic number}. If every $\alpha \in L$ is algebraic over the field $K$, then we can call the field extension $L/K$ \textit{algebraic}.

\end{definition}


\begin{definition}
    $[L:K]$ is defined to be the dimension of the vector space $L$ over the space $K$, and is called the degree of the field extension $L/K$

    A field extension is called \textit{finite} if its degree is finite; otherwise, it is \textit{infinite}. 
\end{definition}

\begin{example}
	$[\Q(\sqrt 2): \Q] = 2$. 
\end{example}
\begin{theorem}[Tower Theorem] \label{thm:tower-theorem}
    If M is an intermediate field of a finite field extension $L/K$ then
$
    [L:K] = [L:M]\cdot[M:K]
$. 
\end{theorem}
\begin{proof}
$M$ is an intermediate field, so we have $K \subseteq M \subseteq L$. Suppose that $[L:M]=m$ and $[M:K]=n$. Then since every field is a vector spaces, we know that each field must have a basis. Thus, take $\lambda = \{\lambda_1,\dots,\lambda_m\}$ as a basis for $L/M$, and similarly take $\mu = \{\mu_1,\dots,\mu_n\}$ as a basis for $M/K$. Then $\forall x \in L$, $x = \sum^m_{i=1}x_i\lambda_i$ for some $x_i \in M$, as $\lambda$ is a basis for $L/M$. Now define $b:=\sum^n_{j=1}\mu_j$ and $d_i:=\frac{x_i}{b}$. Then we have that $x=\sum^m_{i=1}\frac{x_i}{b}\cdot b \cdot \lambda_i = \sum^m_{i=1}\sum^n_{j=1}d_i\cdot \mu_j \cdot \lambda_i$. Now consider the set $\gamma=\{\lambda_i\mu_j : 1\leq i \leq m, 1\leq j \leq n\}$, and consider the result that we have seen above, we can write any $x \in L$ as a linear combination of elements in $\gamma$, therefore, $\gamma$ spans $L/K$.

Now we show that $\gamma$ is linearly independent, and hence confirm that it is a basis for $L/K$. Suppose that for some $c_{ij} \in K$ we have $\sum^m_{i=1} \sum^n_{j=1} c_{ij}\lambda_i\mu_j = 0 $. Then since $\lambda$ is a linearly independent set, as $\lambda$ is a basis, over $M$, for all $i \in \{1,\dots,m\}$ we have $\sum^n_{j=1} c_{ij}\mu_j = 0 $. By seeing that $\mu$ is also a linearly independent set over $K$ we get that $\forall i \in \{1,\dots,m\}$ and $\forall j \in \{1,\dots,n\}$ we have $c_{ij} = 0$. Therefore, $\gamma$ is linearly independent and we  know it spans $L/K$, thus it is a basis for $L/K$ and since $|\gamma|=mn$ this implies that $[L:K] = mn = [L:M]\cdot[M:K]$.
\end{proof}

\subsection{Simple Field Extensions}
\begin{definition}
An extension field \(F\) of \(K\) is called \textit{simple} if \(F = K(\alpha)\) for some \(\alpha \in F\).
\end{definition}

For example, \(\mathbb{Q}(\sqrt{2})\) is a simple field extension over \(\Q\). We can also look at some less obvious extension fields which are simple.
\begin{example}
Taking the extension field \(K = \mathbb{Q}(\sqrt{2}, i)\), we prove that this can be rewritten as a simple field extension \(K' := \mathbb{Q}(\sqrt{2} + i)\). We know \(K'\) contains
$(\sqrt{2} + i)^2 = 1 + 2i\sqrt{2}$, $(i + \sqrt{2})(1+2i\sqrt{2}) = 5i - \sqrt{2}$, and 
$(5i - \sqrt{2}) + (i + \sqrt{2}) = 6i$, 
so we know that \(K'\) contains \(i\). We can also see that \(K'\) contains \(\sqrt{2}\) as \((i+\sqrt{2})-i = \sqrt{2}\). Hence we can see that \(K = K'\), so \(\mathbb{Q}(\sqrt{2},i)\) is a simple field extension.
\end{example}

%\begin{definition}
   % Let $K$ be a subfield of $\mathbb{C}$ and let $\alpha \in \mathbb{C}$. Then $\alpha$ is algebraic over $K$ if there exists a non-zero polynomial $p$ over $K$ such that $p(\alpha)=0$. Otherwise, $\alpha$ is transcendental over $K$.
%\end{definition}

\begin{definition}
    Let $L / K$ be a field extension, and suppose that $\xi \in L$ is algebraic over $K$. Then the \textit{minimal polynomial} $\mu_\xi$  of $\xi$ over $K$ is the unique monic polynomial over $K$ of smallest degree such that $\mu_\xi(\xi)=0$.



\end{definition}
% Readers who know about ideals in rings will see at once that $K[t] /\langle m\rangle$ is a thin disguise for the quotient ring of $K[t]$ by the ideal generated by $m$, and the equivalence classes are cosets of that ideal, but at this stage of the book these concepts are more abstract than we really need.


\begin{example}
    Consider the field extension $\C / \Q$ and $\xi = \sqrt{2} \in \C$, then the minimal polynomial of $\xi$ over $\Q$ is $\mu_\xi(t)=t^2-2$, since if it were of the form $at+b$, this would require either $a \in \Q^c$ or $b \in \Q^c$, and thus $at+b \notin \Q[t]$.
\end{example}

Let $m$ be an irreducible polynomial over a field $K$. We write
$
K[t] /\langle m\rangle
$
for the set of equivalence classes of $K[t]$ modulo $m$. Clearly, $K[t] /\langle m\rangle$ is a ring. 
The following theorem further indicates that $K[t] / \langle m \rangle$ is a field. 


\begin{theorem} \label{thm:irreducible-mod-field}
	Every non-zero element of $K[t] /\langle m\rangle$ has a multiplicative inverse in $K[t] /\langle m\rangle$ if and only if $m$ is irreducible in $K[t]$.
\end{theorem}

\begin{proof}
	Assume that \(m\) is reducible in \(K[t]\), this means that there exists \(a,b \in K[t]\) such that \(m = ab\) where \(\partial a,\partial b < \partial m\). Then we know that \([a]_m[b]_m = [ab]_m = [m]_m = [0]_m\). Suppose that \([a]_m\) has an inverse \([a]_m^{-1} \in K[t]\) such that \([a]_m[a]_m^{-1} = [1]\). Then \([0]_m = [a]_m^{-1}[0]_m = [a]_m^{-1}[a]_m[b]_m = [b]_m\), as we know that \([0]_m = [m]_m\), this implies that \(m\) divides \(b\), but we know that \(b < m\), so this is a contradiction, hence \(m\) is not reducible. 
	If \(m\) is irreducible, let \(a \in K[t]\) with \([a]_m \neq [0]_m\), this implies that \(\gcd(a,m)=1\) and, by Bézout's Lemma, we know that there exists \(p,q \in K[t]\) such that \(ap + mq = 1\). Thus \([a]_m[p]_m + [m]_m[q]_m = [1]_m\) and we know that \([m]_m = [0]_m\), hence \([a]_m[p]_m = [1]_m\). Thus \([a]_m\) has an inverse \([p]_m\) when \(m\) is irreducible.
\end{proof}

The field $K[t] / \langle m \rangle$ contains all constant polynomials over $K$ and hence contains $K$, and thus $\left(K[t] / \langle m \rangle \right) / K$ is a field extension. The next theorem indicates that this extension has the same structure as a simple field extension.

\begin{theorem}
 Let $K(\alpha) / K$ be a simple algebraic extension, and let \(m\) be the minimal polynomial of \(\alpha\) over \(K\). Thus we have that $K(\alpha)$ is isomorphic to $K[t] /\langle m\rangle$. We can choose the isomorphism $K[t] /\langle m\rangle \rightarrow K(\alpha)$ to map $t$ to $\alpha$ and to be the identity on $K$.
\end{theorem}

\begin{proof}
We define the isomorphism by $[p(t)] \mapsto p(\alpha)$, where $[p(t)]$ represents the equivalence class of $p(t)\mod m$. We can check that this map is well-defined because $p(\alpha)=0$ if and only if $m \mid p$. We can see clearly that this is a monomorphism. This maps \(t\) to \(\alpha\) preserving the identity.
\end{proof}

\begin{corollary}
	If \(\alpha\) is an algebraic number, then \(\mathbb{Q}(\alpha)\) is a field.
\end{corollary}

\begin{corollary} \label{thm:minimal-polynomial-roots-isomorphic}
    Suppose $K(\alpha) / K$ and $K(\beta) / K$ are simple algebraic extensions such that $\alpha$ and $\beta$ have the same minimal polynomial $m$ over $K$. Then $K(\alpha)$ and $K(\beta)$ are isomorphic, and the isomorphism can be taken to map $\alpha$ to $\beta$ and to be the identity on $K$.
\end{corollary}


%\begin{theorem}
%	Suppose that $K$ and $L$ are subfields of $\C$ and $\iota : K \to L$ is an isomorphism. Let $K(\alpha),L(\beta)$ be simple algebraic extensions of $K$ and $L$, such that $\alpha$ has minimal polynomial $m_{\alpha}(t)$ over $K$. $\beta$ has minimal polynomial $m_{\beta}(t)$ over $L$. Suppose that $m_{\beta}(t) = \iota(m_{\alpha}(t))$. Then there exists an isomorphism $j : K(\alpha) \to L(\beta)$ such that $j|_{K} = \iota$ and $j(\alpha) = \beta$.
%\end{theorem}
%\begin{proof}
%We can write \(\mathbb{Q}(\alpha) = \{p + \alpha q : p,q \in \mathbb{Q}\}\). First, checking the axioms for addition: \\
%Let \((p + \alpha q)\),  \((s + \alpha t)\) \(\in \mathbb{Q}(\alpha)\), we have \((p + \alpha q) + (s +\alpha t) = (p + s) + \alpha(q + t)\), this satisfies additive closure as \((p+s)\), \((q+t)\) \(\in\mathbb{Q}\). We can clearly see that the addition is commutative and associative from the properties of addition on the rationals. We know that \(\mathbb{Q}(\alpha)\) will contain a zero element when \(p = q = 0\) and will contain an inverse \((p + \alpha q)^{-1} = -p - \alpha q\), as \(-p, -q\in \mathbb{Q}\) and \((p + \alpha q) + (-p - \alpha q) = 0\).
%We can now check the multiplicative axioms: 
%For \((p + \alpha q) , (s + \alpha t ) \in \mathbb{Q}(\alpha)\) , \((p + \alpha q)(s + \alpha t) = ps + tq \alpha^2 + (pt + sq) \alpha = ps + (tq \alpha + pt + sq) \alpha\), we can see that \(tq \alpha + pt + sq \in\mathbb{Q}(\alpha)\) as \((pt + sq), tq \in \mathbb{Q}\).
%\end{proof}

% The above theorem indicates that any two zeros of an irreducible polynomial are ``indistinguishable'' in its splitting field.

The next theorem links the degree of a simple algebraic field extension with the degree of the minimal polynomial of the adjoined element. 
\begin{theorem}[The Degree Theorem] \label{thm:degree-theorem}
    Let $K(\alpha) / K$ be a simple algebraic extension, let the minimal polynomial of $\alpha$ over $K$ be $m$, and let $\partial m=n$. Then $\left\{1, \alpha, \ldots, \alpha^{n-1}\right\}$ is a basis for $K(\alpha)$, considered as a vector space over $K$. In particular, \([K(\alpha):K]=n\). \label{algebraic case}
\end{theorem}

\begin{proof}
    We first want to show that every polynomial \(f \in K[t]\) is congruent modulo \(m\) to a unique polynomial of degree less that the degree of \(m\). We can divide the polynomial \(f\) by \(m\) and we get \(f = qm + r\) with \(q,r \in K[t]\) and \(\partial r < \partial m\), this implies that \(f - r =qm\) and so we can see that \(f \equiv r (\text{mod } m)\). Suppose that there exists \(s \in K[t]\) with \(r \equiv s(\text{mod }m)\), where \(\partial r, \partial s < \partial m\). Then \((r - s)\) divides \(m\), but \((r-s)\) has a smaller degree than \(m\), so \(r-s=0\), which implies that \(r=s\), which proves uniqueness. We have proved that $\left\{1, \alpha, \ldots, \alpha^{n-1}\right\}$ is a basis for \(K(\alpha)\) as \(n = \partial m\), so any higher powers would be congruent to these powers which are less than \(n\).
    \end{proof}

%\begin{theorem}
%    Let $K(\alpha) / K$ be a simple extension. If it is transcendental then $[K(\alpha): K]=\infty$. If it is algebraic then $[K(\alpha): K]=\partial m$, where $m$ is the minimal polynomial of $\alpha$ over $K$. \label{finite degree theorem}
%\end{theorem}
%
%\begin{proof}
%For the transcendental case we know that the elements $1, \alpha, \alpha^2, \ldots$ are linearly independent over $K$. For the algebraic case, we see Lemma \ref{algebraic case}.
%\end{proof}
As a consequence, we now show an equivalent condition for a field extension to be finite. 

\begin{theorem} \label{thm:finite-equi-def}
    A field extension $L/K$ is finite if and only if $L = K(\alpha_1, \dots, \alpha_r)$ for $r$ finite and $\alpha_i$ algebraic over $K$. 
\end{theorem}

\begin{proof}
    Denote $K_s = K(\alpha_1, \dots, \alpha_s)$ for $s = 1, \dots, r$ and let $K_0 = K$. If \(\alpha_1, \dots , \alpha_r \in K\) are algebraic we know that for any \(\alpha_s\) with \(1 \leq s \leq r\) \( [K_s: K_{s-1}] = \partial m\), where \(m\) is the minimal polynomial of \(\alpha_s\) over \(K\) by Theorem \ref{thm:degree-theorem}. Then, using theorem \ref{thm:tower-theorem}, we see that \([K_r : K] = [K_r:K_{r-1}]\dots[K_1:K]\) is finite. Conversely, let \(L/K\) be a finite extension. Then there exists a basis \(\{\alpha_1, \dots, \alpha_r\}\) for \(L\) over \(K\). Let \(t \in L\)  and let \(n = [L:K]\). The set \(\{1,t,\dots,t^n\}\) contains \(n+1\) elements, which have to be linearly dependent over \(K\). Hence, \(k_0 + k_1 t + \dots + k_n t^n = 0\) for \(k_0, k_1, \dots, k_n \in K\), and \(t\) is algebraic over \(K\).
\end{proof}

% NB the definition of an algebraic extension is used in defining normal extensions


