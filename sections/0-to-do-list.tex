
\section*{Definitions We Will Need}
\begin{itemize}
    \item Field - Let K be a field. This means that K is a commutative ring with 1 st every element in $K \backslash \{0\}$ has a multiplicative inverse.
    \item Tower of Fields
    \item Automorphism - A group automorphism is an isomorphism from a group to itself
    \item Monomorphism - A monomorphism is an injective homomorphism of a group $G \rightarrow{} H$
    \item Galois Group - Let $L/K$, and $G$ be the set of automorphisms of $L/K$. then $G$ is a group of transformations $L$, called the Galois Group of $L/K$ denoted $Gal(L/K)$.
    \item Polynomials, Chapter 3, including Eisenstein’s Criterion (p40 Stewart 5e)
    \item Field extensions and simple extensions, Chapters 4 and 5, including Corollary 5.13 Stewart and ``degree theorem'' (if $f$ is the minimal polynomial of $\alpha$ over $K$ then the degree of $f$ equals $[K(\alpha) : K]$ ) (Lemma 5.14 Stewart)
    \item normal and separable field extensions (Chapter 9 Stewart), including 9.9, 9.13 and 9.14
    \item Proposition 11.4 Stewart (Galois group as a permutation group of the zeros)
\end{itemize}

\TODO List

\begin{itemize}
    \item Add all the references
    \item use \textit{Italic} instead \textbf{bold} for emphasis
    \item use $t$ instead of $x$ or $X$ in polynomials
    \item use $\Orb_G(\alpha)$ for the orbit of a group action
    \item introduction
    \item use $\mu_{\alpha}(t)$ for the minimal polynomial
    \item use the word solvable instead of soluble (and solvability instead of solubility)
    \item use ``roots'' of the polynomial instead of ``zeros''
    \item use $\partial f$ for the degree of $f$ instead of $deg(f)$
    \item Try to standaride which letters we are using for fields and extensions. Maybe K when using towers, F as the base field, E as the extension and M and an intermediate field?
    \item Start making our powerpoint and slides.
    \item Make sure we talk about what fields we are working in
    \item Ensure we are finishing sentences with full stops
    \item Make an abstract
    \item fix any line issues
    \item Make it 25 pages
    \item paraphrase any proof copied from the books (probably submit for similarity check)
    \item cut down subsections
    \item try to merge Abel-Ruffini Theorem (rewrite section 7)
    \item use ref for previous theorems whenever possible
\end{itemize}
