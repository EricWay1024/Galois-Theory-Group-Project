\documentclass[12pt]{article}
\usepackage[utf8]{inputenc}
\usepackage{amsmath}
\usepackage{amsthm}
\usepackage{amsfonts}
\usepackage{graphicx}
\usepackage{enumitem}
\usepackage{hyperref}
\usepackage{tikz}
\usepackage{tikz}
\usetikzlibrary{matrix,arrows}
\usepackage{amssymb}
\usepackage{nicematrix}
\usepackage{wrapfig}
\usepackage[a4paper, total={6.5in, 9.4in}]{geometry}
\usepackage[labelsep=space]{caption}
\usepackage{enumitem}
\usepackage{natbib}
\usepackage{placeins}
\newtheorem{theorem}{Theorem}
\newtheorem{corollary}[theorem]{Corollary}
% \newtheorem{corollary}{Corollary}[theorem]
\newtheorem{lemma}[theorem]{Lemma}
\theoremstyle{definition}
\newtheorem{definition}[theorem]{Definition}
\newtheorem{example}[theorem]{Example}
\usepackage[english]{babel}
\newtheorem{observation}[theorem]{\textbf{Observation}}
\bibliographystyle{unsrt}
\usepackage[none]{hyphenat}
\usepackage{fdsymbol}

\newcommand{\Gal}{\operatorname{Gal}}
\newcommand{\Aut}{\operatorname{Aut}}
\newcommand{\Fix}{\operatorname{Fix}}
\newcommand{\Id}{\operatorname{Id}}
\newcommand{\Orb}{\operatorname{Orb}}
\newcommand{\Z}{\mathbb Z}
\newcommand{\Q}{\mathbb Q}
\newcommand{\C}{\mathbb C}
\newcommand{\R}{\mathbb R}
\newcommand{\N}{\mathbb N}

\newcommand{\TODO}{\textbf{\textcolor{red}{TODO}}}

\title{\textbf{Galois Theory}}
\author{Reece Wood, Adam Black, Yuhang Wei, Grace Anderson}
\date{January 2023}

\begin{document}

\maketitle

\tableofcontents

\newpage
\section{Introduction}
\textcolor{red}{Here we need to introduce the topics we talk about, the historical context behind Galois Theory and possibly the extensions we look at, as well as talking about the extensions}
\textcolor{blue}{The basis behind Galois Theory was an open question in mathematics until the start of the 1800s, and the theory originated and was developed due to the following question: ``Does there exist a general solution to a polynomial equation of order five?"  For equations of orders less than or equal to four, the Babylonians had already found a solution to this.
We first look at fields and extensions of fields in order to have a basis.}
\section{Definitions We Will Need}
\begin{itemize}
    \item Field - Let K be a field. This means that K is a commutative ring with 1 st every element in $K \backslash \{0\}$ has a multiplicative inverse.
    \item Tower of Fields
    \item Automorphism - A group automorphism is an isomorphism from a group to itself
    \item Monomorphism - A monomorphism is an injective homomorphism of a group $G \rightarrow{} H$
    \item Galois Group - Let $L/K$, and $G$ be the set of automorphisms of $L/K$. then $G$ is a group of transformations $L$, called the Galois Group of $L/K$ denoted $Gal(L/K)$.
    \item Polynomials, Chapter 3, including Eisenstein’s Criterion (p40 Stewart 5e)
    \item Field extensions and simple extensions, Chapters 4 and 5, including Corollary 5.13 Stewart and ``degree theorem'' (if $f$ is the minimal polynomial of $\alpha$ over $K$ then the degree of $f$ equals $[K(\alpha) : K]$ ) (Lemma 5.14 Stewart)
    \item normal and separable field extensions (Chapter 9 Stewart), including 9.9, 9.13 and 9.14
    \item Proposition 11.4 Stewart (Galois group as a permutation group of the zeros)
\end{itemize}

\section{Polynomials}
\subsection{Basics of Polynomials}

\begin{definition}
    A polynomial over $\C$ in the indeterminate $t$ to be an expression 

    $r_0 + r_1 t +...+ r_n t^n$

    where $r_0,r_1,...,r_n \in \C$, $0 \le n \in \Z$ and $t$ is undefined.
\end{definition}

\begin{definition}
    If $f$ is a polynomial over $\C$ and $f \neq 0$, then the degree of $f$ is the highest power of $t$ occurring in $f$ with a non-zero coefficient.
\end{definition}

\begin{theorem}
    Two polynomials $f,g$ over $\C$ define the same function if and only if they are equal polynomials, so they have the same coefficients.
\end{theorem}

\begin{proof}
    This is a basic proof to an obvious theorem. By taking the difference of the two polynomials, we must prove that if $f(t)$ is a polynomial over $\C$ and $f(t) = 0$ for all $t$, then the coefficients of $f$ are all 0. Let $P(n)$ be the statement: If a polynomial $f(t)$ over $\C$ has degree $n$, and $f(t) = 0$ for all $t \in \C$, then $f=0$. We prove $P(n)$ for all $n$ by induction on $n$. 
    
    Both $P(0)$ and $P(1)$ are obvious. Suppose that $P(n-1)$ is true. 

    With $f(t) = a_n t^n +...+ a_0$. In particular, $f(0) = 0$, so $a_0 = 0$ and,

    $f(t) = a_n t^n +...+ a_1 t= t(a_n t^{n-1} +...+ a_1) = tg(t)$

    where $g(t) = a_n t^{n-1} +...+ a_1$ has degree $n-1$. Now $g(t)$ vanishes for all $t \in \C$ except $t=0$. However is $g(0) = a_1 \neq 0$ then $g(t) \neq 0$ for sufficiently small $t$. Therefore $g(t)$ vanishes for all $t \in \C$. By induction, $g=0$. Therefore $f=0$ so $P(n)$ is true and the induction is complete
\end{proof}

Polynomials will often be written in descending order $r_n t^n + r_{n-1} t^{n+1} + ... + r_1 t + r_0$ for $n$

Two polynomials are defined to be equal if and only if all corresponding co-efficients are equal.

If $r = \sum (r_i t^i)$ and $s = \sum (s_i t^i)$ then $r+s = \sum (r_i + s_i)t^i$

and $rs = \sum (q_j t^j)$ where $q_j = \sum_{h+i=j} r_h s_i$

\subsection{Fundamental Theorem of Algebra} 
This theorem from the module \textit{Complex Functions} is stated without proof. 

\begin{theorem}[Fundamental Theorem of Algebra] \label{thm:fundamental-algebra}
    If $p(z)$ is a non-constant polynomial over $\mathbb{C}$, then there exists $z_0 \in \mathbb{C}$ such that $p\left(z_0\right)=0$.
\end{theorem}

Such a number $z$ is called a root of the equation $p(t)=0$, or a zero of the polynomial $p$. 

\begin{example}
    $i$ is a root of the polynomial equation $t^2+1=0$ and a zero of $t^2+1$. Polynomial equations may have more than one root; indeed, $t^2+1=0$ has at least one other root, $-i$.
\end{example}
% For example, $\mathrm{i}$ is a root of the equation $t^2+1=0$ and a zero of $t^2+1$. Polynomial equations may have more than one root; indeed, $t^2+1=0$ has at least one other root, $-i$.

Proposition 2.7. Let $p(t) \in \mathbb{C}[t]$ with $\partial p=n \geq 1$. Then there exist $\alpha_1, \ldots, \alpha_n \in \mathbb{C}$, and $0 \neq k \in \mathbb{C}$, such that
$$
p(t)=k\left(t-\alpha_1\right) \ldots\left(t-\alpha_n\right).
$$

\begin{proof}
    Use induction on $n$. When $n = 1$ then it is obvious that this proposition follows. If $n > 1$ then we know by the Fundamental Theorem of Algebra, that $p(t)$ has at least one zero $\alpha_n$. By the Remainder Theorem above, there exists $q(t) \in \C[t]$ such that,

    $p(t) = (t-\alpha_n) q(t)$

    Then $\partial$
\end{proof}
% Proof. Use induction on $n$. The case $n=1$ is obvious. If $n>1$ we know, by the Fundamental Theorem of Algebra, that $p(t)$ has at least one zero in $\mathbb{C}$ : call this zero $\alpha_n$. By the Remainder Theorem, there exists $q(t) \in \mathbb{C}[t]$ such that
% $$
% p(t)=\left(t-\alpha_n\right) q(t)
% $$

\subsection{Factorisation of Polynomials}
\begin{definition}
     A non-constant polynomial over a subring $R$ of $\mathbb{C}$ is \textit{reducible} if it is a product of two polynomials over $R$ of smaller degree. Otherwise it is \textit{irreducible}.
\end{definition}

\begin{theorem}
    Any non-zero polynomial over a subring $R$ of $\mathbb{C}$ is a product of irreducible polynomials over $R$.
\end{theorem}

\begin{proof}
    Let $g$ be any non-zero polynomial over $R$. We proceed by in introduction on the degree of $g$. If $\partial g = 0$ or 1 then $g$ is automatically irreducible. If $\partial g > 1$, then either $g$ is irreducible or $g = hk$ where $\partial h$, $\partial k < \partial g$. By induction, $h$ and $k$ are products of irreducible polynomials, where $g$ is such a product. The theorem then follows by induction.
\end{proof}

\begin{example}
    We can use the above theorem to prove irreducibilty for some cases, working very well for cubic polynomials over $\Z$. Let $R = \Z$ then the polynomial $f(t) = t^3 -5 t + 1$ is irreducible. If it was not irreducible then it must have a linear factor $t - \alpha$ over $\Z$ and then $\alpha \in \Z$ and $f(\alpha) = 0$. There also must exist $\beta, \lambda \in \Z$ such that,

    $f(t) = (t-\alpha)(t^2 + \beta t + \lambda) = t^3 + (\beta - \alpha)t^2 + (\lambda - \alpha \beta)t - \alpha \lambda$

    so $\alpha \lambda = -1$. Therefore, $\alpha = +/- 1$. But $f(1) = -3 \neq 0$ and $f(-1) = 5 \neq 0$. This shows that no factor exists.
\end{example}

\begin{theorem}
    Gauss Lemma - Let f be a polynomial over $\Z$ that is irreducible over $\Z$. Then $f$, considered as a polynomial over $\Q$, is also irreducible over $\Q$.
\end{theorem}

\begin{proof}
This lemma is useful because when we extend the subring of coefficients from $\Z$ to $\Q$ then there are new polynomials which may be factors of $f$. However, we now show that they are not. Starting with a contradiction, suppose that $f$ is irreducible over $\Z$ but reducible over $\Q$ so that $f = g h$ where $g$,$h$ are both polynomials over $\Q$, of a smaller degree. Multiplying through by the product of the denominators of the coefficients of $g$ and $h$ means we can rewrite this as $n f = g' h'$, where $n \in \Z$ and $g'$ and $h'$ are polynomials over $\Z$. We can now show that we can cancel out the prime factors of n one by one, without going outside $\Z[t]$

Suppose that $p$ is a prime factor of $n$. We now claim that if $g' = g_0 + g_1 t +...+ g_r t^r$, $h' = h_0 + h_1 t +...+ h_s t^s$ then either $p$ divides all the coefficients of $g_i$, or else $p$ divides all the coefficients $h_j$.

If not then there must be smallest values $i$, $j$ such that $p \nmid g_i$ and $p \nmid h_j$. However, $p$ divides the coefficient of $t^{i+j}$ in $g' h'$, which is 

$h_0 g_{i+j} + h_1 g_{i+j-1} +...+ h_j g_i +...+ h_{i+j} g_0$

and by the choice of $i$ and $j$, the prime $p$ divides every term of this expression except $h_j g_i$. $p$ divides the whole expression, so $p | h_j g_i$. However $p \nmid g_i$ and $p \nmid h_j$, a contradiction.
\end{proof}

\begin{theorem}
    Eisenstein's Criterion - Let
    
    $$f(x) = a_0 + a_1 t + ... + a_n t^n$$
    
    be a polynomial over $\Z$. Suppose there is a prime $q$ such that:
    
    (1) $q \nmid a_n$

    (2) $q | a_i$ for $i = {0, 1,..., n-1}$

    (3) $q^2 \nmid a_0$

    Then $f$ is irreducible over $\Q$
\end{theorem}

\begin{proof}
By Gauss's Lemma it is sufficient to show that $f$ is irreducible over $\Z$. Suppose for a contradiction that $f = gh$, where

$g=b_0+b_1 t+ ... +b_r t^r$ and $h=c_0+c_1 t+ ... +c_s t^s$

are polynomials of smaller degree over $\Z$. Then $r \ge 1, s \ge 1$ and $r+s = n$. Now $b_0 c_0 = a_0$ so by (2), $q | b_0$ or $q|c_0$. By (3), $q$ cannot divide both $b_0$ and $c_0$, so without
loss of generality we can assume $q | b_0$ and $q \nmid c_0$. If all $b_j$ are divisible by $q$, then $a_n$ is divisible by $q$, contrary to (1). Let $b_j$ be the first coefficient of $g$ not divisible by $q$. Then

$a_j = b_j c_0 + ...+ b_0 c_j$ where $j < n$

 This implies that $q$ divides $c_0$, since $q$ divides $a_j, b_0,..., b_{j-1}$, but not $b_j$. This is a contradiction. Hence f is irreducible.
\end{proof}

\begin{example}
Consider

$f(t) = \frac{2}{9} t^5 + \frac{5}{3} t^4 + t^3$ over $\Q$.

This is irreducible over $\Q$ if and only if

$9f(t) = 2t^5 + 15t^4 + 9t^3$ is irreducible over $\Q$. Eisenstein's criterion now applies with $q = 3$, showing that $f$ is irreducible.
\end{example}

\begin{theorem}
If $p$ is prime, the binomial coefficient $\binom{p}{r}$ is divisible by $p$ if $1 \le r \le p-1$
\end{theorem}

\begin{proof}
The binomial coefficient is an integer and $\binom{p}{r} = \frac{p!}{r!(p-r)!}$

The factor $p$in the numerator cannot cancel with any factor in the denominator unless $r=0$ or $r=p$.
\end{proof}

Then we have,

\begin{theorem}
    If $p$ is a prime then the polynomial

    $f(t) = 1 + t + ... = t^{p-1}$

    is irreducible over $\Q$
\end{theorem}

\begin{proof}
First, $f(t) = \frac{t^p - 1}{t - 1}$. Put $t = 1 + u$ where $u$ is a new indeterminate. Then $f(t)$ is irreducible over $\Q$ if and only if $f(1+u)$ is irreducible. But

$f(1+u) = \frac{(1+u)^p - 1}{u}$

$f(1+u) = u^{p-1} + ph(u)$

where $h$ is a polynomial in $u$ over $\Z$ with constant term 1, by the above theorem. By Eisenstein's Criterion, $f(1+u)$ is irreducible over $\Q$

\end{proof}

\section{Field Extensions}

\subsection{Basics of Field Extensions}
In order to look at field extensions we must first define some useful terms.
\begin{definition}
Let \((G, *,\star )\) be a field. Let \(F\) be a subset of \(G\) such that \((F,*,\star)\) is a field. Then \((F, *,\star)\) is a \textit{subfield}
 of \((G, *, \star)\).
\end{definition}
\begin{definition}
A field \(F\) is called a \textit{prime field} of \(K\) if it had no proper (strictly smaller) subfield.
\end{definition}
\begin{definition}
    A \textit{monomorphism} is an injective homomorphism.
\end{definition}
\begin{definition}
A field \(K\) is said to be a \textit{field extension} of \(F\), denoted \(K / F\), if \(F\) is a subfield of \(K\). \cite{Moy} A field extension is a monomorphism \(\iota: F \to K\).
\end{definition}
\begin{example}
The field \(\mathbb{Q}[\sqrt{2}]\) is a field extension of \(\mathbb{Q}\), as \(\mathbb{Q}[\sqrt{2}] = \{a + \sqrt{2}b : a,b \in \mathbb{Q}\}\) we can clearly see that when \(b = 0\) this gives us the rational numbers so \(\mathbb{Q}\) is a subfield of \(\mathbb{Q}[\sqrt{2}]\).
\end{example}
\begin{definition}
Let \(K\) be a subfield of \(L\) if \(\alpha \in L\), and if \(p(\alpha)=0\), for some polynomial \(p(x)\) over \(K\), \(\alpha\) is an \textit{algebraic number}. Otherwise, \(\alpha\) is transcendental over \(K\).
\end{definition}
\begin{theorem}
If \(\alpha\) is an algebraic number, then \(\mathbb{Q}[\alpha]\) is a field.
\end{theorem}
\begin{proof}
We can write \(\mathbb{Q}[\alpha] = \{p + \alpha q : p,q \in \mathbb{Q}\}\). First, checking the axioms for addition: \\
Let \((p + \alpha q)\),  \((s + \alpha t)\) \(\in \mathbb{Q}[\alpha]\), we have \((p + \alpha q) + (s +\alpha t) = (p + s) + \alpha(q + t)\), this satisfies additive closure as \((p+s)\), \((q+t)\) \(\in\mathbb{Q}\). We can clearly see that the addition is commutative and associative from the properties of addition on the rationals. We know that \(\mathbb{Q}[\alpha]\) will contain a zero element when \(p = q = 0\) and will contain an inverse \((p + \alpha q)^{-1} = -p - \alpha q\), as \(-p, -q\in \mathbb{Q}\) and \((p + \alpha q) + (-p - \alpha q) = 0\). \\
We can now check the multiplicative axioms:\\
For \((p + \alpha q) , (s + \alpha t ) \in \mathbb{Q}[\alpha]\) , \((p + \alpha q)(s + \alpha t) = ps + tq \alpha^2 + (pt + sq) \alpha = ps + (tq \alpha + pt + sq) \alpha\), we can see that \(tq \alpha + pt + sq \in\mathbb{Q}[\alpha]\) as \((pt + sq), tq \in \mathbb{Q}\),
\end{proof}


\begin{definition}
    If $L/K$ is a field extensions, we call any field $M$ with $K \subseteq M \subseteq L$ an \textit{intermediate field.}
\end{definition}


\subsection{The Degree of a Field Extension}
\begin{definition}
    The \textit{degree} of the field extension, $\deg(L:K)=[L:K]$ is the dimension of the vector space $L$ over $K$.
\end{definition}
\begin{definition}
    A field extension is called \textit{finite} if its degree is finite. 
\end{definition}
\begin{theorem}[Tower Theorem] \label{thm:tower-theorem}
    If M is an intermediate field of a finite field extension $L/K$ then:
\begin{equation*}
    [L:K] = [L:M]\cdot[M:K]
\end{equation*}
\end{theorem}
\begin{proof}
$M$ is an intermediate field, so we have $K \subseteq M \subseteq L$. Suppose that $[L:M]=m$ and $[M:K]=n$. Then since the subfields are vector spaces, we can takes a basis for each of the field extensions $L/M$ and $M/K$.
Let $\lambda = \{\lambda_1,...,\lambda_m\}$ be a basis for $L/M$, and let $\mu = \{\mu_1,...,\mu_n\}$ be a basis for $M/K$. Then we have that $\forall x \in L$, $x = \sum^m_{i=1}x_i\lambda_i$ for some $x_i \in M$, since $\lambda$ is a basis for $L/M$. Now let $b:=\sum^n_{j=1}\mu_j$ and $d_i:=\frac{x_i}{b}$. Then we have that $x=\sum^m_{i=1}\frac{x_i}{b}\cdot b \cdot \lambda_i = \sum^m_{i=1}\sum^n_{j=1}d_i\cdot \mu_j \cdot \lambda_i$. If we consider the set $\gamma=\{\lambda_i\mu_j : 1\leq i \leq m, 1\leq j \leq n\}$, and consider from what we have seen above, we can write any $x \in L$ as a linear combination of some elements of elements in $\gamma$. Therefore, $\gamma$ spans $L/K$.

Now it is required to show that $\gamma$ is linearly independent, in order to show that it is a basis for $L/K$. Suppose that for some $c_{ij} \in K$ we have $\sum^m_{i=1} \sum^n_{j=1} c_{ij}\lambda_i\mu_j = 0 $. Then since $\lambda$ is a linearly independent set over $M$, for all $i \in \{1,...,m\}$ we have $\sum^n_{j=1} c_{ij}\mu_j = 0 $. By seeing that $\mu$ is also a linearly independent set over $K$ we get that $\forall i \in \{1,...,m\}$ and $\forall j \in \{1,...,n\}$ we have $c_{ij} = 0$. This then implies that $\gamma$ is linearly independent and we already know it spans $L/K$, thus it is a basis for $L/K$ and since $|\gamma|=mn$ this implies that $[L:K] = mn = [L:M]\cdot[M:K]$.

\end{proof}

\subsection{Simple Field Extensions}
\begin{definition}
An extension field \(F\) of \(K\) is called simple if \(F = K(\alpha)\) for some \(\alpha \in F\).
\end{definition}
\begin{example}
Clearly, \(\mathbb{Q}(\sqrt{2}) / \Q\) is a simple field extension.
\end{example}
We can also look at some less obvious extension fields which are simple.
\begin{example}
Taking the extension field \(K = \mathbb{Q}(\sqrt{2}, i)\), we prove that this can be rewritten as a simple field extension \(K' = \mathbb{Q}(\sqrt{2} + i)\), by showing that \(K=K'\) making \(K\) a simple field extension. We know \(K'\) contains
\[(\sqrt{2} + i)^2 = 1 + 2i\sqrt{2}\]
\[(i + \sqrt{2})(1+2i\sqrt{2}) = 5i - \sqrt{2}\]
\[(5i - \sqrt{2}) + (i + \sqrt{2}) = 6i\]
so we know that \(K'\) contains \(i\). We can also see that \(K'\) contains \(\sqrt{2}\) as \((i+\sqrt{2})-i = \sqrt{2}\). Hence we can see that \(K = K'\), so \(\mathbb{Q}(\sqrt{2},i)\) is a simple field extension.
\end{example}

Definition 5.1. Let $K$ be a subfield of $\mathbb{C}$ and let $\alpha \in \mathbb{C}$. Then $\alpha$ is algebraic over $K$ if there exists a non-zero polynomial $p$ over $K$ such that $p(\alpha)=0$. Otherwise, $\alpha$ is transcendental over $K$.

Definition 5.4. Let $L / K$ be a field extension, and suppose that $\alpha \in L$ is algebraic over $K$. Then the minimal polynomial of $\alpha$ over $K$ is the unique monic polynomial $m$ over $K$ of smallest degree such that $m(\alpha)=0$.

We write
$$
K[t] /\langle m\rangle
$$
for the set of equivalence classes of $K[t]$ modulo $m$. 
% Readers who know about ideals in rings will see at once that $K[t] /\langle m\rangle$ is a thin disguise for the quotient ring of $K[t]$ by the ideal generated by $m$, and the equivalence classes are cosets of that ideal, but at this stage of the book these concepts are more abstract than we really need.

Theorem 5.10. Every nonzero element of $K[t] /\langle m\rangle$ has a multiplicative inverse in $K[t] /\langle m\rangle$ if and only if $m$ is irreducible in $K[t]$.

Theorem 5.12. Let $K(\alpha) / K$ be a simple algebraic extension, and let the minimal polynomial of $\alpha$ over $K$ be $m$. Then $K(\alpha) / K$ is isomorphic to $K[t] /\langle m\rangle / K$. The isomorphism $K[t] /\langle m\rangle \rightarrow K(\alpha)$ can be chosen to map $t$ to $\alpha$ and to be the identity on $K$.

% Proof. The isomorphism is defined by $[p(t)] \mapsto p(\alpha)$, where $[p(t)]$ is the equivalence class of $p(t)(\bmod m)$. This map is well-defined because $p(\alpha)=0$ if and only if $m \mid p$. It is clearly a field monomorphism. It maps $t$ to $\alpha$, and its restriction to $K$ is the identity.

Corollary 5.13. Suppose $K(\alpha) / K$ and $K(\beta) / K$ are simple algebraic extensions such that $\alpha$ and $\beta$ have the same minimal polynomial $m$ over $K$. Then the two extensions are isomorphic, and the isomorphism of the large fields can be taken to map $\alpha$ to $\beta$ and to be the identity on $K$.

Lemma 5.14. Let $K(\alpha) / K$ be a simple algebraic extension, let the minimal polynomial of $\alpha$ over $K$ be $m$, and let $\partial m=n$. Then $\left\{1, \alpha, \ldots, \alpha^{n-1}\right\}$ is a basis for $K(\alpha)$, considered as a vector space over $K$.

Proposition 6.7. Let $K(\alpha) / K$ be a simple extension. If it is transcendental then $[K(\alpha): K]=\infty$. If it is algebraic then $[K(\alpha): K]=\partial m$, where $m$ is the minimal polynomial of $\alpha$ over $K$.

% Proof. For the transcendental case it suffices to note that the elements $1, \alpha, \alpha^2, \ldots$ are linearly independent over $K$. For the algebraic case, we appeal to Lemma 5.14.

\begin{theorem}
    Let $M/K$ be a field extension and let $\alpha \in M$. Then $\alpha$ is algebraic over $K$ if and only if $\alpha$ is in a finite field extension of $K$. 
\end{theorem}

\begin{definition}
    A field extension $L/K$ is algebraic if any $\alpha \in L$ is algebraic over $K$. 
\end{definition}

\begin{theorem} \label{thm:finite-equi-def}
    A field extension $L/K$ is finite if and only if $L = K(\alpha_1, \dots, \alpha_r)$ for $r$ finite and $\alpha_i$ algebraic over $K$. 
\end{theorem}

% NB the definition of an algebraic extension is used in defining normal extensions





\section{Normal and Separable Extensions}
In our focus on field extensions, it is crucial to understand the properties of normality and separability as they are the key ingredients for Galois extensions which we will discuss later. Before introducing normal and separable extensions, we first look at splitting fields and algebraic closures, two closely related concepts. 

\subsection{Splitting Fields}
\begin{definition}
    Let $K$ be a subfield of $\C$. A polynomial $f$ over $K$ \textit{splits} in $K$ if it factors into linear factors $$
    f(t) = k \prod _{i=1} ^n (t - \alpha_i),
    $$
    where $k, \alpha_1, \ldots, \alpha_n \in K$. 
\end{definition}
Theorem \ref{thm:fundamental-algebra} indicates that $f$ splits over $K$ if and only if all of its zeros are contained in $K$. 

\begin{definition}
    % A \textbf{\textit{splitting field}} of a polynomial $f$ over a subfield $K$ of $\C$ is a subfield $\Sigma$ of $\C$ such that 
    % \begin{itemize}
    %     \item $K \subseteq \Sigma$,
    %     \item $f$ splits over $\Sigma$ and
    %     \item for any $\Sigma'$ such that $K \subseteq \Sigma' \subseteq \Sigma$ and $f$ splits over $\Sigma'$, we have $\Sigma' = \Sigma$. 
    % \end{itemize}
    The splitting field $\Sigma$ of a polynomial $f$ over a subfield $K$ of $\C$ is the field generated by $K$ and all zeros of $f$. Equivalently, it is the smallest field containing $K$ and all zeros of $f$. 
\end{definition}

\begin{example}
If we have a polynomial \(p(x) = x^4 - 12x^2 + 35\) the splitting field of \(p(x)\) is \(\mathbb{Q}[\sqrt{5},\sqrt{7}]\) as it contains all of the roots of \(p(x)\) and if it was any smaller it would not contain all of the roots or would not be a field.
\end{example}

The splitting fields of a polynomial over two isomorphic fields are isomorphic, in the following sense:

% Theorem 9.6 Stewart
\begin{theorem} \label{thm:splitting-field-unique}
	Let $\iota: K \to K'$ be a field isomorphism. Let $f$ be a polynomial over $f$ with the splitting field $\Sigma$, and then let $\Sigma'$ be the splitting field of $\iota(f)$ over $K'$. Then there is an isomorphism $j : \Sigma \to \Sigma'$ such that $j | _K = \iota$. 
\end{theorem}

\begin{proof}
	\TODO
\end{proof}


\subsection{Algebraic Closures}
We might extend the definition of a splitting field of \textit{a set of polynomials} over a field $K$. The definition would then be the smallest field containing $K$ and all zeros of every polynomial in the set. 
\begin{definition}
    The algebraic closure of a field $K$ is the splitting field of the set of all polynomials over $K$, denoted as $\overline K$. 
\end{definition}

%The existence of $\overline K$ for any field $K$ can be proven but is omitted here. 

\begin{theorem} 
	Let $\iota: K \to K'$ be a field isomorphism. Then there is an isomorphism $j: \overline K \to \overline {K'}$ such that $j |_K = \iota$. 
\end{theorem}

%\begin{definition}
%    A field $K$ is algebraically closed if every polynomial $f$ over $K$ has a root in $K$. 
%\end{definition}
%
%\begin{theorem}
%    For a field $K$, its smallest algebraically closed extension is the algebraic closure of $K$.
%\end{theorem}



\subsection{Normal Extensions}


\begin{definition}
    An algebraic field extension $L/ K$ is \textit{normal} if every irreducible polynomial over $K$ which has a zero in $L$ splits in $L$. 
\end{definition}
\begin{example}
	For any field $K$, $\overline K / K$ is a normal extension, as every polynomial over $K$ splits in $\overline K$. 
\end{example}
\begin{example}
    $\C / \R$ is normal, since every polynomial over $\R$ splits in $\C$ by Theorem \ref{thm:fundamental-algebra}.
\end{example}
\begin{example}
    Let $K = \Q(\sqrt[3]{2})$. Then $K / \Q$ is not normal. The polynomial $f(t) = t^3 - 2$ over $\Q$ is irreducible and has a zero $\sqrt[3]{2}$ in $K$, but $K \subseteq \R$ does not contain the two non-real zeros of $f$. Hence $f$ does not split in $K$.
\end{example}

\begin{theorem} \label{thm:normal-equiv-def}
    Let $L/K$ be an finite field extension. Then the following three statements are equivalent:
    \begin{enumerate}[label=(\roman*)]
        \item $L/K$ is normal;
        \item $L$ is the splitting field for a polynomial $f$ over $K$;
        \item Any $K$-monomorphism $\sigma: L \to \overline K$ is a $K$-automorphism of $L$. 
    \end{enumerate}
\end{theorem}

\begin{proof}
    (i) $\Rightarrow$ (ii). 
    Suppose $L/K$ is normal and finite. The finiteness of $L/K$ implies that $L = K(\alpha_1, \dots, \alpha_r)$ for $r$ finite and $\alpha_i$ algebraic over $K$ by Theorem \ref{thm:finite-equi-def}. For each $\alpha_i$, take the minimal polynomial $f_i$ of $\alpha_i$ over $K$. Each $f_i$ is irreducible over $K$, so the normality of $L/K$ implies that $f_i$ splits in $L$.  Then take $f = f_1 \dots f_r$ and $L$ is the splitting field of $f$.
	
%	    For each $\alpha \in L$, let $f_\alpha$ be its minimal polynomial over $K$. Each $f_\alpha$ is irreducible over $K$, so the normality of $L/K$ implies that $f_\alpha$ splits in $L$. Then clearly $L$ is the splitting field of $\{ f_\alpha \text{ over } K : \alpha \in L \}$.    
    

    (ii) $\Rightarrow$ (iii). Let $\sigma$ be a $K$-monomophism from $L$ to $\overline K$. $\sigma$ can be considered as an isomorphism from $L$ to $\sigma(L)$ which fixes all elements in $K$. Since $L$ is the splitting field for a polynomial $f$ over $K$, $f$ have coefficients in $K$ and thus is fixed under $\sigma$. Hence $\sigma(L) = L$ and $\sigma$ is a $K$-automorphism of $L$. 
\end{proof}

Before we prove (iii) $\Rightarrow$ (i), we pulse and prove the following result, which enables us to ``extend" a $K$-monomorphism $M \to L$ to a $K$-automorphism of $L$ where $K \subseteq M \subseteq L$. 

\begin{theorem} \label{thm:monomorphism-extend-automorphism}
	Let $L/K$ be a normal field extension and let $M$ be an intermediate field such that $K \subseteq M \subseteq L$. Let $\tau$ be a $K$-monomorphism $M \to L$. Then there exists a $K$-automorphism $\sigma$ of $L$ such that $\sigma | _M = \tau$.  
\end{theorem}

\begin{proof}
	By (i) $\Rightarrow$ (ii) of Theorem \ref{thm:normal-equiv-def}, $L$ is the splitting field for a polynomial $f$ over $K$ and thus also over $M$. $\tau$ can be considered as an isomorphism from $M$ to $\tau(M)$ which fixes every element in $K$ and hence $f$. Then $L$ is both the splitting field for $f$ over $M$ and the splitting field for $f$ over $\tau(M)$. Then by Theorem \ref{thm:splitting-field-unique}, there exists an isomorphism $\sigma: L \to L$ such that $\sigma | _M = \tau$. Since $\sigma | _K = \tau |_K = \Id$, $\sigma$ is a $K$-automorphism of $L$. 
\end{proof}

\begin{theorem} \label{thm:automorphism-from-zeros}
    Let $L/K$ be a field extension and let $\alpha, \beta \in L$ be zeros of an irreducible polynomial $f$ over $K$. Then there exists a $K$-automorphism $\sigma$ of $L$ such that $\sigma(\alpha) = \beta$. 
\end{theorem}

\begin{proof}
     $K(\alpha)/K$ and $K(\beta)/K$ are isomorphic extensions by \TODO(Corollary 5.13). The isomorphism from $K(\alpha)$ to $K(\beta)$ is also a $K$-monomorphism from $K(\alpha)$ to $\overline K$ and thus can be extended to a $K$-automorphism $\sigma$ of $\overline K$ by Theorem \ref{thm:monomorphism-extend-automorphism}, where $\sigma(\alpha) = \beta$.
\end{proof}

\begin{proof}[Proof of Theorem \ref{thm:normal-equiv-def}, continued]
	    (iii) $\Rightarrow$ (i). Let $f$ be an irreducible polynomial over $K$ with zero $\alpha \in L$. For any zero $\beta \in \overline K$ of $f$, there exists a $K$-automorphism $\sigma$ of $\overline K $ such that $\sigma(\alpha) = \beta$ by Theorem \ref{thm:automorphism-from-zeros}. By assumption, $\sigma(L) = L$. With $\alpha \in L$ we deduce that $\beta \in L$. Hence $L/K$ is normal.
\end{proof}

%\begin{corollary}
%    $L/K$ is normal and finite if and only if $L$ is the splitting field for a polynomial $f$ over $K$. 
%\end{corollary}
%
%\begin{proof}
%    If $L/K$ is normal and finite, the finiteness of $L/K$ implies that $L = K(\alpha_1, \dots, \alpha_r)$ for $r$ finite and $\alpha_i$ algebraic over $K$ by Theorem \ref{thm:finite-equi-def}. For each $\alpha_i$, take the minimal polynomial $f_i$ of $\alpha$ over $K$. Then similarly as above, take $f = f_1 \dots f_r$ and $L$ is the splitting field of $f$.
%
%    The converse is trivial.
%\end{proof}


\subsection{Separable Extensions}

Galois did not specifically acknowledge the idea of separability, as he solely worked with subfields of $\C$, where separability is inherently present, as we will discover.

\begin{definition}
    Let $f$ be an irreducible polynomial over field $K$. Then $f$ is separable over $K$ if $f$ takes the form 
    $$
        f(t) = k(t - \sigma_1) \dots (t - \sigma_n)
    $$
    where $\sigma_i \in \overline K$ are distinct.
\end{definition}

The following theorem of a computational nature is useful to determine whether a polynomial is separable and is given without proof. 

\begin{theorem} \label{thm:separable-derivative}
    A polynomial $f$ over a subfield $K$ of $\C$ is separable if and only if it is coprime with $Df$ over $K$, where $Df$ is its derivative. 
\end{theorem}

The next theorem implies that we can take for granted the separability of irreducible polynomials over subfields of $\C$. 

\begin{theorem} \label{thm:separable-poly-in-C}
    If $K$ is a subfield of $\C$, then every irreducible polynomial over $K$ is separable. 
\end{theorem}

\begin{proof}
    Let $f$ be an irreducible polynomial over $K$. Suppose that $f$ is not separable, then by Theorem \ref{thm:separable-derivative}, $f$ and $Df$ shares a common factor of degree $\ge 1$. Since $f$ is irreducible, this common factor must be $f$. Since $Df$ has a smaller degree than $f$, $Df$ must be $0$. This implies that $f$ is constant.
\end{proof}

\begin{definition}
    Let $L/K$ be a field extension. An element $\alpha \in L$ is \textit{separable} if the minimal polynomial of $\alpha$ over $K$ is separable over $K$.
\end{definition}

\begin{definition} \label{def:separable-extension}
    A field extension $L / K$ is \textit{separable} if every $\alpha \in L$ is separable.
\end{definition}

\begin{theorem}
    A field extension $L/K$ such that $K \subseteq L \subseteq \mathbb C$ is separable. 
\end{theorem}

\begin{proof}
    This directly follows from Theorem \ref{thm:separable-poly-in-C} and Definition \ref{def:separable-extension}. 
\end{proof}

\section{Galois Groups}
\subsection{Galois Group Actions}

\begin{definition}
	Let $L/K$ be a field extension. A $K$-automorphism $\phi$ of $L$ is an automorphism of $L$ such that $\phi(a) = a$ for all $a \in K$. 
\end{definition}

\begin{definition}
    Let $L/K$ be a field extension. Then all $K$-automorphisms of $L$ form a group under map composition, and this group is called the \textit{Galois group} of $L$, written as \(\Gal(L/K)\).
%    The  is called the \textbf{Galois group} of the field extension \(F\) over \(K\), usually written .
\end{definition}

\begin{definition}
	Let $f$ be a polynomial over $K$ and let $L$ be a splitting field of $f$, then $\Gal(L/K)$ is the \textit{Galois group} of $f$, written as \(\Gal(f)\).
	
%		 In the case that \(F\) is a splitting field of a polynomial \(f\) over $K$, \(G\) is called the Galois group of \(f\),
\end{definition}

This group becomes increasingly useful, especially when dealing with polynomials, as we will see with the following theorem.

\begin{theorem} \label{thm:galois-group-permutes-zeros}
Let $\phi \in \Gal(L/K)$ where $L/K$ is a field extension. Then let $f$ be a polynomial over $K$ with a zero $a \in L$, then $\phi(a)$ is also a zero of $f$.
\end{theorem}

\begin{proof}
    Since $a \in L$ in a zero of $f$ over $K$, $f(a) = 0$. Then $\phi(f(a)) = \phi(0) = 0$. $\phi$ fixes the coefficients of $f$ as they are in $K$ and $\phi\left(a^i\right) = \left(\phi(a) \right) ^i$ as $\phi$ is a homomorphism. Therefore $f(\phi(a)) = 0$. 
\end{proof}

\begin{definition}
	Given a set $X$, the \textit{permutation group} of $X$ is the group of $\{ (\phi : X \to X) : \phi \text{ is a bijection} \}$ under map composition (clearly this forms a group), denoted as $S_X$. 
\end{definition}

\begin{definition}
	Given a group $G$ and a set $X$, we say $G$ \textit{acts} on $X$ if there is a homomorphism from $G$ to $S_X$.
\end{definition}

\begin{theorem} \label{thm:galois-group-acts-on-zeros}
	Let $L$ be a splitting field of a polynomial $f$ over $K$ and let $G = \Gal(f)$. Let $X$ be the set of zeros of $f$. Then $G$ acts on $X$.
\end{theorem}

\begin{proof}
	Take any $\sigma \in G$. Then clearly $\alpha \in X$ if and only if $\sigma(\alpha) \in X$. Restricting $\sigma$ on $X$ gives a bijection $\sigma | _X : X \to X$. 
\end{proof}


\begin{corollary} \label{thm:galois-group-isomorphic-symmetric-subgroup}
	$\Gal(f)$ is isomorphic to a subgroup of $S_n$, where $n$ is the degree of $f$. 
\end{corollary}

\begin{proof}
	Let $X$ be the set of zeros of $f$. Then $|X| \le n$ and $S_X$ is a subgroup of $S_n$. The homomorphism from $\Gal(f)$ to $S_X$ is also a homomorphism from $\Gal(f)$ to $S_n$. Further, the kernel of the homomorphism is trivial, as if $\sigma \in \Gal(f)$ fixes all elements in $X$, it is the identity map. Therefore, by Theorem \ref{thm:first-iso}, $\Gal(f)$ is isomorphic to its image in $S_n$, which is a subgroup of $S_n$. 
\end{proof}

%\begin{theorem}
%	A polynomial $f$ over $K$ is irreducible if and only if $\Gal(f)$ acts transitively on the set of zeros of $f$.
%\end{theorem}





\subsection{Order of Galois Groups}
%A $K$-automorphism is totally determined by its permutation of


%\TODO $\Gal(L/K)$ acts on the zeros of $f$. It is thus isomorphic to a subgroup of $S_n$ (by the first isomorphism theorem). 

\begin{theorem} \label{thm:galois-group-order-upper-bound}
    Let $L/K$ be a finite field extension and let $G = \Gal(L/K)$, then $|G| \le [L:K]$. 
\end{theorem}

\begin{proof}
    Consider a $K$-monomorphism $\phi_0 : K \to X$ where $K \subseteq X$. Since $L/K$ is finite, by Theorem \ref{thm:finite-equi-def}, we can write $L = K(\alpha_1, \ldots, \alpha_r)$ for $r$ finite and $\alpha_i$ algebraic over $K$. Denote $K_0 = K$ and
    $K_i = K(\alpha_1, \dots, \alpha_i)$ for $i = 1, \ldots, r$. In particular, $K_r = L$. 


    For $i = 1, \ldots, r$, consider the extension $K_i / K_{i-1}$, where $K_i = K_{i-1} (\alpha_i)$. Let $\alpha_i$ be a zero of an irreducible polynomial $f_i$ over $K$ of degree $\partial f_i = [K_i : K_{i-1}]$ by Theorem \TODO. $f_i$ is over $K$ and hence is fixed under $\phi_i$. We now extend $K$-monomorphism $\phi_{i-1} : K_{i-1} \to X$ to $K$-monomorphism $\phi_i: K_i \to X$, then $\phi_i$ must map $\alpha_i$ to a zero of $f_i$ in $X$, and the number of zeros of $f_i$ in $X$ is \textit{at most} $\partial f_i$. Hence there are at most $[K_i : K_{i-1}]$ ways to extend $\phi_{i-1}$ to $\phi_i$. We now see that there are at most $$\prod_{i=1} ^r [K_i : K_{i-1}] = [K_r : K_0] = [L : K]$$ ways to extend $\phi_0$ to $\phi_r$ by Thorem \ref{thm:tower-theorem}. Since $\phi_r : L \to X$, if we let $X = L$, then $\phi_r \in \Gal(L/K)$. Also, there is only one way to construct $\phi_0$, which is the identity map. Thus each way of extending $\phi_0$ to $\phi_r$ corresponds to an element in $\Gal(L/K)$, and thus $\Gal(L/K) \le [L:K]$. 
\end{proof}
\cite{galois-theory-lectures}


\subsection{Fixed Fields}

\begin{definition}
    Let $\Aut(L)$ be the group of automorphisms of a field $L$ and let $G \leq \Aut(L)$. Then the fixed field of $G$ is the set $$\Fix(G) = \{ \alpha \in L : \sigma(\alpha) = \alpha \quad \forall \sigma \in G \}. $$ The fixed field is always a field.
\end{definition}
Thus $\Fix(G)$ is the subset of elements in $L$ which remain fixed for all automorphisms in $G$. In particular, for a finite field extension $L/ K$, the Galois group $G = \Gal(L / K)$ is a subgroup of $\Aut(L)$ such that $$K \subseteq \Fix(G) \subseteq L. $$ We are mainly interested in the case where $K = \Fix(G)$, the condition for which is fully developed in the next section. We now look at a counterexample. 

\begin{example}
    It is not always the case that $K = \Fix(G)$. For example, consider when $K = \Q$ and $L = \Q (\alpha)$ where $\alpha = \sqrt[3]{2}$. If $\sigma \in G$, then $\sigma(\alpha)$ is a zero of the polynomial $t^3 - 2$ by Theorem \ref{thm:galois-group-permutes-zeros}, but the two complex zeros are not contained in $L$. Thus $\sigma(\alpha) = \alpha$ and $G = \{ \Id \}$. We see that $\Fix(G) = L \neq K$. 
\end{example}

\subsection{Roots of Unity}
We now take a detour and look at some special cases where the Galois group of a field extension is abelian. The results obtained here will be useful in Section \ref{sec:galois-groups-and-polynomials}. 

\begin{definition}
	Let $n$ be a positive integer. An $n$-th root of unity $\omega$ in $\mathbb C$ is such that $\omega ^ n = 1$. An $n$-th root of unity $\omega$ is primitive if for any $m = 1, 2, \dots, n - 1$, $\omega ^ m \neq 1$.
\end{definition}

\begin{observation}
	Let $n$ be a positive integer. All $n$-th roots of unity form a cyclic group under multiplication, and the group is generated by any primitive $n$-th root of unity. For a prime number $p$, let $\omega$ be a $p$-th root of unity and $\omega \neq 1$. Then $\omega$ is a primitive $p$-th root of unity and generates the multiplicative group of all $p$-th roots of unity.
\end{observation}

\begin{theorem} \label{thm:unity-1}
	Let $n$ be a positive integer and let $\omega$ be a primitive $n$-th root of unity. Let $L$ be a subfield of $\mathbb C$. If $\omega \in L$, then the polynomial $t^n - 1$ splits in $L$.
\end{theorem}
\begin{proof}
	$\omega$ has order $n$ in the multiplicative group of $L$, so the elements $1, \omega, \omega^2, \ldots, \omega^{n-1}$ are distinct $n$-th roots of unity in $L$. Therefore $t^n-1$ splits in $L$.
\end{proof}

\begin{theorem} \label{thm:unity-2}
	Let $n$ be a positive integer and let $\omega$ be a primitive $n$-th root of unity. If $L$ is the splitting field for $t^n - 1$ over a subfield $K$ of $\mathbb C$, then $L = K(\omega)$.
\end{theorem}

\begin{proof}
	The derivative of $t^n-1$ is $n t^{n-1}$, which is prime to $t^n-1$, so the polynomial $t^n-1$ has no multiple zeros in $L$ by Theorem \ref{thm:separable-derivative}. The group of its zeros under multiplication thus has order $n$ and is cyclic. Let $\omega$ be a generator of this group, and thus $\omega$ is a primitive $n$-th root of unity. Then $L=K(\omega)$. 
\end{proof}




\begin{theorem} \label{thm:radical-1}
	Let $p$ be a prime and let $\omega$ be a primitive $p$-th root of unity. Let $K$ be a subfield of $\mathbb C$. Then $\Gal(K(\omega) / K)$ is abelian.
\end{theorem}
\begin{proof}
	Let $L = K(\omega)$.  Let $\alpha \in \Gal(L / K)$. Then $\alpha$ is uniquely determined by $\alpha(\omega)$ and $\alpha$ is a permutation of the multiplicative group generated by $\omega$. Thus $\alpha$ has the form
	$$
	\alpha_j: \omega \mapsto \omega^j,
	$$
	where $j=1,\dots,p-1$. Then $\alpha_i \alpha_j (\omega) = \alpha_j \alpha_i (\omega) = \omega^{i j}$, so $ \Gal(L / K)$ is abelian.
	
	\TODO probably rewrite using group actions
\end{proof}

\begin{theorem} \label{thm:radical-2}
	Let $K$ be a subfield of $\mathbb{C}$ which contains an $n$-th primitive root of unity. Let $\beta^n = c \in K $ where $n$ is a positive integer. Then $t^n - c$ splits in $K(\beta)$ and $\Gal(K(\beta) / K)$ is abelian.
\end{theorem}

\begin{proof}
	By Theorem \ref{thm:unity-1}, $K$ contains all $n$-th roots of unity. Let $L = K(\beta)$, and $\beta$ is a zero of $t^n-c$ in $L$. Any zero of $t^n-c$ in $L$ can be represented by $\omega \beta$, where $\omega$ is some $n$-th root of unity in $K$. Thus $t^n - c$ splits in $L$.  
	
	Let $\phi \in \Gal(L / K)$, then $\phi$ is uniquely determined by $\phi(\beta)$ and $\phi$ is a permutation of the zeros of $t^n - c$. Let $\phi_1, \phi_2 \in \Gal(L / K)$, and let $\phi_1(\beta) = \omega_1\beta$, $\phi_2(\beta) = \omega_2\beta$, where $\omega_1, \omega_2$ are $n$-th roots of unity in $K$. Then
	$$
	\phi_1 \phi_2(\beta)=\omega_1 \omega_2 \beta=\omega_2 \omega_1  \beta=\phi_2 \phi_1(\beta).
	$$
	Thus $\Gal(L / K)$ is abelian.
	
	\TODO probably rewrite using group actions
\end{proof}

\begin{theorem} \label{thm:unity-3}
	Let $K$ be a subfield of $\mathbb C$. Let $\alpha \notin K$ and $\alpha^p = c \in K$ where $p$ is a prime. Let $L / K$ be a normal extension such that $\alpha \in L$. Then $L$ contains a primitive $p$-th root of unity.
\end{theorem}

\begin{proof}
	Let $f$ be the minimal polynomial of $\alpha$ over $K$. Then $f$ splits in $L$ and has no repeated zeros. Since $\alpha \notin K$, we have $\partial f \ge 2$, and therefore there exists $\beta \in L$ such that $\beta$ is a zero of $f$ and $\beta \neq \alpha$. Since $\alpha ^ p - c = 0$, $f$ must divide $t ^ p - c$. Thus $\beta^p - c = 0$. Let $\omega=\alpha / \beta \in L $, then $\omega^p=1$ and $\omega \neq 1$.
\end{proof}

\section{Galois Extensions and Galois Correspondence}

\subsection{Galois Extension}


We are now ready to present a special kind of field extensions:

\begin{definition}
    A finite field extension is \textit{Galois} if it is both normal and separable. 
\end{definition}


\begin{theorem} \label{thm:fixed}
	Let $L/K$ be a finite field extension, and let $G = \Gal(L/K)$. Then the following statements are equivalent:
	\begin{enumerate}[label=(\roman*)]
		\item $L/K$ is a Galois extension;
		\item $[L:K] = |G|$;
		\item $K = \Fix(G)$;
		\item $L$ is a splitting field of a separable polynomial over $K$;
		%        \item There is a natural bijection between the intermediate fields $M$ such that $K \subseteq M \subseteq L$ and the subgroups $H \le G$, where $\alpha$ and $\beta$ defined by $\alpha(M) =  \Gal (L/M)$ and $\beta(H) =  \Fix(H)$ are inverses of each other.
	\end{enumerate}
	% Given a finite Galois extension $L/K$ and its Galois group $G=\Gal(L/K)$, we have $K = \Fix(G)$.
\end{theorem}

%The proof has two parts: the first part establishes the equivalence of the first four statements and the second part works on the equivalence of the fifth and the rest. 

%\begin{proof}[Proof, first part]
\begin{proof}
	
	%	We now prove the equivalence of the first four statements. We will work on the fifth statement separately shortly afterwards.
	
	
	(i) $\Rightarrow$ (ii). Consider a $K$-monomorphism $\phi: K \to \overline{K}$. We now extend $\phi$ to a $K$-monomorphism $\phi':L \to \overline{K}$ in the same pattern as in the proof of Theorem \ref{thm:galois-group-order-upper-bound}. However, at each step, we claim that the number of ways to extend is equal to $[K_i : K_{i-1}]$. Indeed, $\alpha_i$ is separable and hence $f_i$ has exactly $[K_i : K_{i-1}]$ distinct zeros, and $\overline K$ contains all these zeros. Therefore, combining all the steps gives us in total $[L:K]$ ways to extend $\phi$ to $\phi'$. Also, since $L/K$ is normal, by Theorem \ref{thm:normal-equiv-def}, each $K$-monomorphism $\phi': L \to \overline{K}$ is a $K$-automorphism of $L$, and thus each $\phi'$ corresponds to an element in $\Gal(L/K)$. Hence $[L:K] = \Gal(L/K)$. 
	
	(ii) $\Rightarrow$ (iii). Define $F:= \Fix(G)$. We have $K \subseteq F \subseteq G$. Then:
	\begin{itemize}
		\item By assumption, we have $[L:K] = |G|$;
		\item     Each $K$-automorphism of $L$ in $G$ is also a $F$-automorphism of $L$ in $\Gal(L/F)$, and thus $G \subseteq \Gal(L / F)$ and $|G| \le |\Gal(L / F)|$;
		\item  By Theorem \ref{thm:galois-group-order-upper-bound}, $|\Gal(L/F)| \le [L:F]$;
		
		\item     By Theorem \ref{thm:tower-theorem}, we see that $ [L:F] \le [L:F][F:K] = [L:K]$.
	\end{itemize}
	
	Combining them gives $$
	[L:K] = |G| \le |\Gal(L/F)| \le  [L:F] \le [L:K],
	$$    
	and thus all the $\le$ signs must take equality.  
	Thus $[L:K]=[L:F]$ and therefore by Theorem \ref{thm:tower-theorem}, we see that $[F:K]$ must equal $1$, and thus $F:=\Fix(G) = K$.
	
	(iii) $\Rightarrow$ (iv). Let $\alpha \in L$. Consider $\Orb_G(\alpha) \subseteq L$, the orbit of $\alpha$ under the $G$-action on $L$. Then since $G$ is finite, $\Orb_G(\alpha) = \{ \alpha_1 = \alpha, \alpha_2, \ldots, \alpha_n\}$. Consider the polynomial $f(t) = (t-\alpha_1) \ldots (t-\alpha_n)$ over $L$.  $f$ is separable in $L$, as $\alpha_i$ are distinct. Also, any $\sigma \in G$ acting on $\alpha_1, \dots, \alpha_n$ only permutes the elements, and thus $\sigma$ fixes $f$. Then the coefficients of $f$ must be in $\Fix(G)$, but $\Fix(G) = K$ by assumption. Hence $f$ is over $K$. Clearly, $L$ is a splitting field of $f$ over $K$. 
	
	(iv) $\Rightarrow$ (i) is trivial.
	
\end{proof}

We can now obtain a useful result.

\begin{theorem} \label{thm:galois-intermediate}
	If $L/K$ is a Galois extension, then for an intermediate field $M$ such that $K \subseteq M \subseteq L$, $L/M$ is also a Galois extension. 
\end{theorem}
\begin{proof}
	By (iv) of Theorem \ref{thm:fixed}, $L$ is the splitting of some polynomial $f$ over $K$, but $f$ is also a polynomial over $M$. Hence $L/M$ is Galois.
\end{proof}

Note that in the above theorem, $L/K$ being a Galois extension does not imply that $M/K$ is Galois. This is in general false; consider for example $K = \Q, M = \Q(\sqrt[3]{2})$ and $L = \Q(\sqrt[3]{2}, \zeta)$, where $\zeta = e^{2\pi i / 3}$. Then $L/K$ and $L/M$ are Galois but $M/K$ is not. 

\begin{definition}
   A \textit{normal closure} of a field extension $L / K$ is an extension $N$ of $L$ such that 
   \begin{enumerate}
       \item $N / K$ is normal;
       \item If $L \subseteq M \subseteq N$ and $M / K$ is normal, then $M = N$.
   \end{enumerate}
\end{definition}

\begin{theorem}
	Let $L$ be a field and let $G \le \Aut(L)$ be finite. Let $K = \Fix(G)$. Then $[L:K] =|G|$. 
\end{theorem}

\begin{proof}
	\TODO(10.5)
\end{proof}

\begin{theorem} \label{thm:fix-extension-normal}
	Let $L$ be a field and let $G \le \Aut(L)$. Let $K = \Fix(G)$. Then $L/K$ is a normal extension with $\Gal(L/K) = G$. 
\end{theorem}

\begin{proof}
	\TODO(11.14)
\end{proof}

\subsection{Examples of Galois Extensions}

\TODO

\subsection{Fundamental Theorem of Galois Theory}

%\begin{theorem}
%	Let $M$ be a field and let $G \le \Aut(M)$. Then $M / \Fix(G)$ is a Galois extension.
%\end{theorem}
%
%\begin{proof}
%	\TODO
%\end{proof}
%
%\begin{theorem}
%    Let $G$ be a finite subgroup of the group of automorphisms of a field $L$, and let $K = \Fix(G)$. Then $[L : K] = |G|$. 
%\end{theorem}
%
%\begin{proof}
%    \TODO
%\end{proof}

% \noindent We now look at a theorem which helps to characterise and identify fields, extensions and Galois groups.



Group theory tends to look at patterns and symmetries in mathematical objects, and Galois groups have an interesting symmetry. The following theorem describes how the structure of an extension of a field is the same as the subgroups of the Galois groups.

To help us prove the fundamental theorem, we first need to show a relationship between a subgroup of a Galois group and its subgroups

\begin{theorem}\label{thm:equal-subgroup-fixed-points}
    Let $L/F$ be a finite Galois extension, with $G=\Gal(L/F)$, then if we have a subgroup $H\leq G$ with $\Fix(H)=F$ then $H=G$.
\end{theorem}

\begin{proof}
    Suppose $H\leq G$ with $\Fix(H)=F$. Then, let $L = F(a)$, then $H \cdot a$ denotes the orbit under the group action of $H$ in $L$, and we can assume that $H \dot a = \{a_1,...,a_m\}.$ From this we can see that $|H \cdot a| = m$, and let $[L:F]=n$.
    Then we know that the $m$ cannot be larger than the size of $H$, and since $H\leq G \implies |H|\leq|G|$, we get $|H \cdot a| = m \leq |H| \leq |G| = [L:F] = n$ 
    
    \textcolor{red}{Put a section in polynomials regarding symmetric polynomials and the elementary symmetric polynomials and minimal polynomials.} 
    
    \noindent
    Then by evaluating the elementary symmetric polynomials at each of the elements in $H \cdot a$, we get that these are just the elements in $\Fix(H)=F$. Therefore, we see that the coefficients in the polynomial: $f(X) := (X-a_1)(X-a_2)...(X-a_m)$ lies in $F$, and we have $f(a)=0$. Then, our minimal polynomial $\mu_a(X)$ divides $f(X)$ and $\deg(f) \leq \deg(\mu_a(X) \implies m \leq n$. Hence we have $m \geq n$ and $n \geq m$, so therefore we have that $m=n$, which tells us $|H|=|G|$ and so $H=G.$
\end{proof}

\begin{theorem}[The Fundamental Theorem of Galois Theory] \label{thm:fundamental-theorem} Given a Galois extension $L/K$ and its Galois group $G = \Gal(L/K)$, there is a natural bijection between subgroups $H\leq G$ and the intermediate fields $M$ such that $K \subseteq M \subseteq L$. Define

\begin{itemize}
    \item $\alpha:M \mapsto \Gal(L/M) \leq G$ and
    \item $\beta:H \mapsto \Fix(H) \subseteq L$,
\end{itemize}
then $\alpha$ and $\beta$ are inverses of each other.
\end{theorem}
\begin{proof}
\begin{enumerate}[label=(\roman*)]

 \item ($\beta \circ \alpha = \Id$) Let $M$ be an intermediate field, and by Theorem \ref{thm:galois-intermediate} we can see that $L/M$ is a Galois Extension. Then if we consider the Galois group $\Gal(L/M)$, we can see that via Theorem \ref{thm:fixed}, $M = \Fix(\Gal(L/M))$. Also, $\alpha(M) = \Gal(L/M)$. Then we look at $(\beta \circ \alpha)(M)$, which is equal to $\beta(\Gal(L/M)) = \Fix(\Gal(L/M))$, but again as see in Theorem \ref{thm:fixed}, we have $\Fix(\Gal(L/M)) = M$. Thus $(\beta \circ \alpha)(M) = M$ and hence we see that $\beta \circ \alpha = \Id$.

 \item ($\alpha \circ \beta = \Id$) Now, let $H$ be a subgroup of $G$ and then set $M :=\Fix(H)$. Then $H$ is a subgroup of the Galois group $\Gal(L/M)$. Since $K \subseteq M \subseteq L$, by Theorem \ref{thm:galois-intermediate} we have that $L/M$ is a Galois extension. Thus, by Theorem \ref{thm:equal-subgroup-fixed-points}, since we have $H\leq \Gal(L/M)$ and we have $\Fix(H)=M$, then we can say that $H=\Gal(L/M)=\Gal(L/\Fix(H))$. Thus if we consider $(\alpha \circ \beta)(H) = \alpha(\Fix(H)) = \Gal(L/\Fix(H))= H $, we have $(\alpha \circ \beta)(H) = H$ and so $\alpha \circ \beta = \Id$.
\end{enumerate}
Hence we can see that the functions $\alpha$ and $\beta$ are inverse to each other.
\end{proof}



\begin{theorem} \label{thm:correspondence-quotient}
    Let $L / K$ is a Galois extension and let $M$ be an intermediate field such that $M /K$ is a Galois extension, then 
    $$\Gal(M / K) \cong \Gal(L / K) / \Gal(L / M). $$
\end{theorem}

\begin{proof}
	% Theoerm 58, Rotman
	Let $G = \Gal(L/K)$ and $G' = \Gal(M/K)$. Define a map $\psi: G \to G'$ as $\psi(\sigma) = \sigma | _M$ for $\sigma \in G$.
	
	We first show that indeed $\psi(\sigma) \in G'$ for all $\sigma \in G$. We only need to prove that $\sigma(M) = M$. $M/K$ is Galois, so by Theorem \ref{thm:fixed}, $M$ is a splitting field for a separable polynomial $f$ over $K$. Let $\alpha_1, \dots, \alpha_n$ be the distinct zeros of $f$, then $M = K(\alpha_1, \ldots, \alpha_n)$. But $\sigma$ permutes $\alpha_i$ by Theorem \ref{thm:galois-group-acts-on-zeros} and $\sigma(K) = K$. Therefore $\sigma(M) = \sigma(K(\alpha_i, \ldots, \alpha_n)) = K(\sigma(\alpha_1), \ldots, \sigma(\alpha_n)) = M. $
	
	
	 $\psi$ is clearly a group homomorphism. It is onto due to the normality of $L/K$ and Theorem \ref{thm:monomorphism-extend-automorphism}: each $K$-automorphism $\tau$ of $M$ is also a $K$-monomorphism from $M$ to $L$ and thus there exists a $K$-automorphism $\sigma$ of $L$ such that $\sigma | _M = \tau$. Also, $\sigma \in \operatorname{Ker}(\psi)$ if and only if $\psi(\sigma) = \sigma |_M = \Id$, which is equivalent to $\sigma \in \Gal(L/M)$. Thus $\operatorname{Ker}(\psi) = \Gal(L/M)$. 
	 
	 Thus by Theorem \ref{thm:first-iso}, 
	$$G / \operatorname{Ker}(\psi) \cong \operatorname{Im}(\psi) = G',$$
	and the required result thus follows. 
\end{proof}

\section{Galois Groups and Polynomials} \label{sec:galois-groups-and-polynomials}


\subsection{Soluble and Simple Groups}

\begin{definition} \label{def:soluble}
    A group $G$ is soluble (or solvable) if it has a finite series of subgroups 
    $$ \{ e \} = G_0 \triangleleft G_1 \triangleleft \dots \triangleleft G_n = G$$
    such that the quotient $G_{i+1} / G_{i}$ is abelian for each $i = 0, 1, ...,  n - 1$.
\end{definition}

\begin{observation}
    An abelian group is soluble. 
\end{observation}



\begin{theorem} \label{thm:soluble-main}
    Let $G$ be a group, $H \le G$ and $N \trianglelefteq G$. Then 
    \begin{enumerate}
        \item If $G$ is soluble, then $H$ is soluble;
        \item If $G$ is soluble, then $G / N$ is soluble; 
        \item If $N$ and $G / N$ are soluble, then $G$ is soluble. 
    \end{enumerate}
\end{theorem}
\begin{proof}
\begin{enumerate}
    \item If $G$ is soluble, then there exists a finite series of subgroups $G_i$ of $G$ for $i = 0, 1, \dots, n$ satisfying Definition \ref{def:soluble}. Let $H_i = G_i \cap H$. Then $H$ has a series of subgroups $H_i$ such that $\{ e \} = H_0 \triangleleft H_1 \triangleleft \dots \triangleleft H_n = H.$
    For each $i = 0, 1, \dots, n - 1$, 
    $$
    \frac{H_{i+1}}{H_i} 
        = \frac{G_{i+1} \cap H}{G_i \cap (G_{i+1} \cap H)}
        \cong \frac{G_i(G_{i+1} \cap H)} {G_i}
    $$
    by Theorem \ref{thm:second-iso}, and ${G_i(G_{i+1} \cap H)}/{G_i}$ is a subgroup of the abelian group $G_{i+1} / G_{i}$. Hence $H_{i+1} / H_{i}$ is abelian and $H$ is soluble.
    \item Take $G_i$ as before. Then $G / N$ has a series
        $N/N = G_0 N / N \triangleleft G_1 N / N \triangleleft \dots \triangleleft G_n N / N  =  G / N. $
        For each $i = 0, 1, \dots, n - 1$, 
        $(G_{i+1} N / N) / (G_{i} N / N) \cong (G_{i+1} N) / (G_i N)$
    by Theorem \ref{thm:third-iso}. Then 
    $$
    \frac{G_{i+1} N}{G_i N} =\frac{G_{i+1}\left(G_i N\right)}{G_i N} \cong \frac{G_{i+1}}{G_{i+1} \cap\left(G_i N\right)} \cong \frac{G_{i+1} / G_i}{\left(G_{i+1} \cap\left(G_i N\right)\right) / G_i},
    $$
    which is a quotient of the abelian group $G_{i+1} / G_i$, so is abelian. Hence $G / N$ is soluble.
    \item There exist two series
$$
\begin{aligned}
\{ e \} & =N_0 \triangleleft N_1 \triangleleft \ldots \triangleleft N_r=N, \\
N / N & =G_0 / N \triangleleft G_1 / N \triangleleft \ldots \triangleleft G_s / N=G / N
\end{aligned}
$$
with $N_{i+1} / N_{i}$ abelian for each $i = 0, 1, \dots, r-1$ and $(G_{i+1} / N)  / (G_{i} / N) \cong G_{i+1} / G_i $ abelian for each $i = 0,1, \dots, s-1$. Combining them gives the series of subgroups of $G$:
$$
\{ e \}=N_0 \triangleleft N_1 \triangleleft \ldots \triangleleft N_r=N=G_0 \triangleleft G_1 \triangleleft \ldots \triangleleft G_s=G .
$$
The quotients are either $N_{i+1} / N_i$  or $G_{i+1} / G_i$ and are all abelian. Therefore $G$ is soluble.
\end{enumerate}
\end{proof}

\begin{definition}
    A group $G$ is \textit{simple} if it is nontrivial and its only normal subgroups are $\{ e \}$ and $G$. 
\end{definition}

\begin{theorem} \label{thm:soluble-and-simple}
    A soluble group is simple if and only if it is a cyclic group of prime order.
\end{theorem}

\begin{proof}
    Let $G$ be a soluble group and suppose $G$ is simple. Consider the series of subgroups of $G$
$$
\{ e \}=G_0 \triangleleft G_1 \triangleleft \ldots \triangleleft G_n=G,
$$
with abelian quotients, and without loss of generality we assume $G_{i+1} \neq G_i$. Since $G$ is simple, $G_{n-1}$, which is a proper normal subgroup of $G_n = G$, must be $\{ e \}$. Solubility of $G$ gives that $G_n / G_{n -1 }$ is abelian, but $G_n / G_{n - 1} = G$, and thus $G$ is abelian. Thus for any element $g \in G$, the cyclic group $\langle g\rangle$ is a normal subgroup in $G$. Hence  $G = \langle g\rangle$ for any $g \neq e$. Hence $G$ is cyclic of prime order.

The converse is trivial.
\end{proof}


\begin{theorem} \label{thm:simple-alternating}
    The alternating group $A_n$ is simple when $n \ge 5$. 
\end{theorem}

\begin{proof}
    (Prove for the case $n=5$ \TODO) Due to the length and technical nature of the complete proof, only a concise summary is presented here. 
    Suppose that $\{ e \} \neq N \triangleleft A_n$. We can prove that $N$ must contain a $3$-cycle using case-by-case analysis. Next, we can show that if $N$ contains a $3$-cycle, then it contains all $3$-cycles. Since $A_n$ is generated by the $3$-cycles when $n \ge 3$, this means $N = A_n$.
\end{proof}

\begin{theorem} \label{thm:symmetric-not-soluble}
    The symmetric group $S_n$ is not soluble when $n \ge 5$. 
\end{theorem}

\begin{proof}
    Suppose $S_n$ is soluble for $n \ge 5$. Then since $A_n \trianglelefteq S_n$, by Theorem \ref{thm:soluble-main}, $A_n$ is soluble. By Theorem \ref{thm:simple-alternating}, $A_n$ is also simple, so $A_n$ is cyclic of prime order by Theorem \ref{thm:soluble-and-simple}. But $|A_n| = n! / 2$ which obviously is not prime. Contradiction.
\end{proof}




\subsection{Solubility by Radicals}
\begin{definition} \label{def:radical-extension}
    A field extension $L / K$ is \textit{radical} if $L=K\left(\alpha_1, \ldots, \alpha_m\right)$, where for each $i=1, \ldots, m$ there exists $n_i$ such that
$$
\alpha_i^{n_i} \in K\left(\alpha_1, \ldots, \alpha_{i-1}\right).
$$
The elements $\alpha_i$ form a radical sequence for $L / K$, and the radical degree of $\alpha_i$ is $n_i$.
\end{definition}

\begin{theorem} \label{thm:radical-single-prime}
    Let $K$ be a field. If $\alpha ^ n \in K$ for a positive integer $n$ and  $L = K (\alpha)$, then there exists a radical sequence $\beta_i$ for $L / K$ such that the radical degree for each $\beta_i$ is prime.
\end{theorem}

\begin{proof}
    Let $n = \prod_{k=1}^{s} p_{k}$ where $p_{k}$ are prime numbers (not necessarily distinct). Let $\beta_{i} = \alpha^ {p_{i + 1} \dots  p_{s}}$ for $i = 0, \dots s$. Write $L = K(\beta_1,  \dots, \beta_s = \alpha)$, and then $\beta_0 = \alpha^n \in K$ and  $\beta_i ^ {p_i} = \beta_{i-1} \in K(\beta_1, \dots, \beta_{i - 1})$ for each $i  = 1, \dots, s$. 
\end{proof}

\begin{example}
    Consider $K  = \mathbb Q$ and $\alpha = \sqrt[20]{2}$, where $\alpha ^ {20} \in \mathbb Q$. Write $n = 20 =  2 \times 2 \times 5 $. Then we can take the radical sequence as $\beta_1 = \sqrt 2, \beta_2 = \sqrt[4]{2}, \beta_3 = \sqrt[20]{2}$. Then $\mathbb Q(\sqrt[20]{2}) $ $= \mathbb Q(\sqrt{2}, \sqrt[4]{2}, \sqrt[20]{2})$ and 
    $$
    \beta_1 ^ 2 \in \mathbb Q, \quad \beta_2 ^ 2 \in \mathbb Q(\sqrt{2}), \quad \beta_3^5 \in \mathbb Q (\sqrt{2}, \sqrt[4]{2}),
    $$
    where each of the radical degrees $2, 2, 5$ is prime.
\end{example}

\begin{theorem} \label{thm:radical-all-prime}
    Let $L / K$ be a racial extension. Then there exists a radical sequence $\beta_j$ for $L / K$ where $j=1, \dots, r$ such that the radical degree of each $\beta_j$ is prime.
\end{theorem}

\begin{proof}
    For each $\alpha_i$ in any given radical sequence for $L / K$, replace it with a series $\beta_{ij}$ by Theorem \ref{thm:radical-single-prime}. Then combine all $\beta_{ij}$ and relabel the subscripts to make $\beta_{j}$. 
\end{proof}

\begin{theorem} \label{thm:radical-3}
If $K$ is a subfield of $\mathbb{C}$ and $L / K$ is normal and radical, then $\Gal(L / K)$ is soluble.
\end{theorem}

\begin{proof}
Let $L=K\left(\alpha_1, \ldots, \alpha_n\right)$ with $\alpha_j^{n_j} \in K\left(\alpha_1, \ldots, \alpha_{j-1}\right)$. By Theorem \ref{thm:radical-all-prime}, we may assume that $n_j$ is prime for all $j$. We prove the result by induction on $n$. 
 % In particular there is a prime $p$ such that $\alpha_1^p \in K$.

If $n = 0$, we have $L = K$, and the case is trivial.

If $n \ge 1$ and $\alpha_1 \in K$, then $L=K\left(\alpha_2, \ldots, \alpha_n\right)$ and $\Gal(L / K)$ is soluble by induction.

Let $n \ge 1$ and $\alpha_1 \notin K$. Let $p = n_1$, which is prime, and then $\alpha_1^p \in K$.  Then $L$ contains a primitive $p$-th root of unity $\omega$ by Theorem \ref{thm:unity-3}. Let $M = K(\omega)$ and let $N = M(\alpha_1)$. Consider the chain of subfields $K \subseteq M \subseteq N \subseteq L$. We now prove the solubility of $\Gal(L/N), \Gal(N/M), \Gal(L/M), \Gal(M/K)$, and $\Gal(L/K)$ step by step:

\begin{itemize}
    \item Since $L=N\left(\alpha_2, \ldots, \alpha_n\right)$, $L / N$ is a normal radical extension. By induction $\Gal\left(L / N\right)$ is soluble. 
    \item Since $ \omega \in M$ and $\alpha_1^p \in M$, Theorem \ref{thm:radical-2} implies that $N$ is a splitting field for $t^p-\alpha_1^p$ over $M$. Thus $N / M$ is normal. By Theorem \ref{thm:radical-2}, $\Gal\left(N / M\right)$ is abelian and hence soluble. 
    \item  $L/ K$ is finite and normal, and so is $L / M$. Apply Theorem \ref{thm:correspondence-quotient} to $L / M$ to deduce that
    $$
    \Gal\left( N / M \right) \cong \Gal(L / M) / \Gal\left(L / N\right).
    $$
    Hence by Theorem \ref{thm:soluble-main},  $ \Gal(L / M)$ is soluble.
    \item By Theorem \ref{thm:unity-1}, $M$ is the splitting field for $t^p-1$ over $K$. Then the extension $M / K$ is normal. By Theorem \ref{thm:radical-1}, $\Gal(M / K)$ is abelian and hence soluble.
    \item  Apply Theorem \ref{thm:correspondence-quotient} to $L / K$ to obtain
    $$
    \Gal(M / K) \cong \Gal(L / K) / \Gal(L / M). 
    $$
    Theorem \ref{thm:soluble-main} shows that $\Gal(L / K)$ is soluble, completing the induction step.
\end{itemize}
\end{proof}

\begin{theorem} \label{thm:radical-closure}
    If $L / K$ is a radical extension in $\mathbb{C}$ and $M$ is the normal closure of $L / K$, then $M / K$ is radical.
\end{theorem}

\begin{proof}
\TODO paraphrase

Let $L=K\left(\alpha_1, \ldots, \alpha_m\right)$ with $\alpha_i^{n_i} \in K\left(\alpha_1, \ldots, \alpha_{i-1}\right)$. Let $f_i$ be the minimal polynomial of $\alpha_i$ over $K$. Then $M \supseteq L$ is clearly the splitting field of $\prod_{i=1}^m f_i$. For every zero $\beta_{i j}$, by Theorem \ref{thm:automorphism-from-zeros}, there exists a $K$-automorphism $\tau$ of $M$ such that $\tau(\alpha_i) = \beta_{ij}$. Since $\alpha_i$ is a member of a radical sequence for a subfield of $M$, so is $\beta_{i j}$. Combining the sequences yields a radical sequence for $M$.
\end{proof}


\begin{definition}
    Let $f$ be a polynomial over a subfield $K$ of $\mathbb{C}$, with splitting field $\Sigma$ over $K$. Then $f$ is \textit{soluble by radicals} if there exists a field $M \supseteq \Sigma$ such that $M / K$ is a radical extension.  
\end{definition}


\begin{theorem} \label{thm:radical-galois-soluble}
    Let $f$ be a polynomial over a subfield $K$ of $\mathbb{C}$. If $f$ is soluble by radicals, then the Galois group of $f$ over $K$ is soluble.
\end{theorem}

\begin{proof}
\TODO paraphrase

Since $f$ is soluble by radicals, its splitting field $\Sigma$ over $K$ satisfies $K \subseteq \Sigma \subseteq M$ where $M / K$ is a radical extension. Let $K_0$ be the fixed field of $\Gal(\Sigma / K)$, and let $N / M$ be the normal closure of $M / K_0$. Then
$$
K \subseteq K_0 \subseteq \Sigma \subseteq M \subseteq N.
$$
Since $M / K_0$ is radical, Theorem \ref{thm:radical-closure} implies that $N / K_0$ is a normal radical extension. By Theorem \ref{thm:radical-3}, $\Gal\left(N / K_0\right)$ is soluble.
By Theorem \ref{thm:fix-extension-normal}, the extension $\Sigma / K_0$ is normal. By Theorem \ref{thm:correspondence-quotient}
$$
\Gal\left(\Sigma / K_0\right) \cong \Gal\left(N / K_0\right) / \Gal(N / \Sigma).
$$
Theorem \ref{thm:soluble-main} implies that $\Gal\left(\Sigma / K_0\right)$ is soluble. But $\Gal(\Sigma / K)=\Gal\left(\Sigma / K_0\right)$, so $\Gal(\Sigma / K)$ is soluble.

\end{proof}

The converse also holds, but we do not prove it here.

\subsection{An Insoluble Quintic}
We first look at how symmetric groups can be generated by its elements. 

\begin{theorem} \label{thm:symmetric-12-12n}
    The transposition $(12)$ and $n$-cycle $(12 \dots n)$ together generate $S_n$. 
\end{theorem}
\begin{proof}
    Consider the cyclic decomposition of each element in $S_n$, where each cyclic permutation can be written as 
    $
    (a_1a_2\dots a_k) = (a_1 a_k) \dots (a_1 a_3) (a_1 a_2). 
    $
    Note that $(ab) = (1a)(1b)(1a)$, therefore $(12), (13), \ldots (1n)$ together generate $S_n$. Also note that  
    $$(1k)=((k-1)k)\dots(34)(23)(12)(23)(34)\dots((k-1)k),$$
    therefore $(12), (23), \dots, ((n-1)n)$ together generate $S_n$. Finally note that 
    $$
    (k(k+1)) = (12\dots n)^{k-1} (12) (12\dots n)
    $$
    for $k = 2, 3, \dots n - 1$. Therefore $(12)$ and $(12 \dots n)$ together generate $S_n$. 
\end{proof}

\begin{theorem} \label{thm:symmetric-ab-12n}
    For $1 \le a < b \le n$ such that $(b - a, n) = 1$, the transposition $(ab)$ and $n$-cycle $(12 \dots n)$ together generate $S_n$.
\end{theorem}
\begin{proof}
   Let $\sigma=(12 \ldots n)$, so $\sigma^i(a) \equiv a+i \bmod n$. Therefore $\sigma^{b-a}(a) \equiv b \bmod n$, and since $1 \le b \le n$, we have $\sigma^{b-a}(a)=b$. Since $(b-a, n)=1,\langle\sigma\rangle=\left\langle\sigma^{b-a}\right\rangle$ and $\sigma^{b-a}$ is an $n$-cycle sending $a$ to $b$, so $\sigma^{b-a}$ is of the form $(a b \ldots)$. Then
$
\langle(a b), \sigma\rangle=\left\langle(a b), \sigma^{b-a}\right\rangle=\langle(a b),(a b \ldots)\rangle .
$
Relabel the numbers $1,2 \ldots, n$ so that $(a b)$ turns into $(12)$ and $(a b \ldots)$ into $(12 \ldots n)$. Therefore $\langle(a b), \sigma\rangle=S_n$ by Theorem \ref{thm:symmetric-12-12n}.
\end{proof}

\begin{theorem} \label{thm:symmetric-prime}
    For a prime number $p$, any transposition and $p$-cycle together generate $S_p$.
\end{theorem}
\begin{proof}
    Relabel the numbers so that the $p$-cycle turns into $(12 \dots p)$. Suppose the transposition turns into $(ab)$, where $1 \le a < b \le n$. Since $p$ is prime and $1 \le b - a < p$, we have $(b - a, p) = 1$. The result thus follows from Theorem \ref{thm:symmetric-ab-12n}.
\end{proof}

Now we establish a case where the Galois group of a polynomial over $\mathbb Q$ is isomorphic to a symmetric group.

\begin{theorem} \label{thm:galois-iso-symmetric}
    Let $p$ be a prime number and $f$ be a irreducible polynomial of degree $p$ over $\mathbb Q$. If $f$ has exactly two non-real zeros in $\mathbb C$, then the Galois group of $f$ over $\mathbb Q$ is isomorphic to the symmetric group $S_p$.
\end{theorem}

\begin{proof}
%    Theorem \ref{thm:fundamental-algebra} implies that there exists a splitting field $\Sigma $ of $f$ contained in $\mathbb C$. The Galois group $\Gal(\Sigma / \mathbb Q)$ of $f$ can be considered as a permutation group of the zeros of $f$ \textbf{(Group action??)}. By Theorem ??, $f$ has distinct zeros, so $\Gal(\Sigma / \mathbb Q)$ is isomorphic to a subgroup of $S_p$. We denote this subgroup of $S_p$ as $G$. 
	By Theorem \ref{thm:galois-group-isomorphic-symmetric-subgroup}, $\Gal(f)$ is isomorphic to a subgroup of $S_p$. We denote this subgroup as $G$. We now claim that $G = S_p$ by showing that $G$ contains a transposition and a $p$-cycle.

    Complex conjugation restricted to $\Sigma$ is a $\mathbb Q$-automorphism of $\Sigma$. It fixes all $p - 2$ real zeros of $f$ and transposes the two non-real zeros. Therefore $G$ contains a transposition. 

    Take any zero $\alpha$ of $f$. Since $f$ is irreducible, it is the minimal polynomial of $\alpha$. Then by the \textbf{Degree Theorem}, $$[\mathbb Q(\alpha) : \mathbb Q] = \partial f = p. $$ Theorem \ref{thm:tower-theorem} implies that $$[\Sigma : \mathbb Q] = [\Sigma : \mathbb Q(\alpha)] [ \mathbb Q(\alpha) : \mathbb Q]. $$ By Theorem \ref{thm:fixed}, $|G| = [\Sigma : \mathbb Q]$, and thus $p$ divides $|G|$. Theorem \ref{thm:cauchy} gives the existence of an element of order $p$ in $G$, but the only elements of order $p$ in $S_p$ are $p$-cycles. Therefore $G$ contains a $p$-cycle.

    Hence by Theorem \ref{thm:symmetric-prime}, $ G = S_p$ and therefore $\Gal(\Sigma / \mathbb Q) \cong S_p$.
\end{proof}

We now give a quintic polynomial not soluble by radicals. 

\begin{example}
    The polynomial $x^5 + 10 x^4 - 2$ over $\mathbb Q$ is not soluble by radicals.
\end{example}

\begin{proof}
    $f$ is irreducible by \textbf{Eisenstein's criterion} with $p = 2$. The degree of $f$ is $5$, which is prime. Calculus shows that $f$ has three real zeros. Thus the Galois group of $f$ is isomorphic to $S_5$ by Theorem \ref{thm:galois-iso-symmetric}. But $S_5$ is not soluble by Theorem \ref{thm:symmetric-not-soluble}. Thus by Theorem \ref{thm:radical-galois-soluble}, $f$ is not soluble by radicals.
\end{proof}

\subsection{Abel-Ruffini Theorem}

If we think of a general quadratic polynomial, i.e one of the form $aX^2+bX+c$, and try to find its roots, we know that there is a formula to find the roots of the general quadratic, which is $x = \frac{-b \pm \sqrt{b^2 - 4ac}}{2a}$. A similar formula can be found to find the roots of a general cubic and quartic polynomial. However, there is no such general formula for polynomial of degree 5 an higher.

The following theorem, the proof of which has been adapted from Mrinial \cite{Abel-Ruffini}, was posed and proved (incompletely) by Paolo Ruffini and completed by Niels Henrik Abel, shows that the general polynomial of order five and above is not solvable by radicals.

First, we need to understand a lemma in order to be able to prove the Abel-Ruffini Theorem. The proof of this lemma is adapted from a lecture by DeVille \cite{galois-lecture-polynomials}

\begin{lemma}\label{lemma:galois-symmetric}
    For an irreducible polynomial $f(x) \in F[x]$ with Galois group $K/F$, then $Gal(K/F)$ is isomorphic to the symmetric group $S_n$.
\end{lemma}

\begin{proof}
    If we first assume that $f(x)$ is irreducible, then we have that any $\sigma \in \Gal(K/F)$ maps a root of $f(x)$ to a root of $f(x)$. Thus if we let $\{\alpha_1,\alpha_2,...,\alpha_n\}$ be the roots of the polynomial $f(x)$, we can see that any $\sigma$ just permutes $\alpha_i$ to $\alpha_j$. Thus we can say that $\sigma$ is just a permutation of the set $\{1,2,...,n\}$ and thus we have an injection from $\Gal(K/F) \to S_n$. Therefore we can say that the Galois group of a polynomial is isomorphic to $S_n$
\end{proof}

\begin{theorem}[Galois' Theorem]\label{thm:galois-theorem}
     A polynomial $f(x)$ is solvable by radicals if and only if its Galois group is solvable
\end{theorem}

\begin{proof}
    "$\implies $"Given a general polynomial of the form $f(x) = a_0X^n - a_1X^{n-1}+...(-1)^na_n$, let us assume that it can be solved by radicals. Then we know that for each root of the polynomial, $x=\zeta_i$ where $0\leq i \leq n$, we have an extension $A_i/F$ which is cyclic, since $\Gal(A_i/F)$ is cyclic. If we then take the composition of all of those fields, we get another field $L$ which contains all the roots of $f(x)$ and is itself Galois. 
    \textcolor{red}{Thus proves that if a polynomial can be solved by radicals, its Galois group is solvable}

    "$\impliedby$" Instead, supposed the polynomial $f(x)$ has a Galois group, $G$, which is solvable. Then we can consider the following chain of subfields: $F=K_0\subset K_1 \subset ... \subset K_n = K$. Here we define $K_i = \Fix(G_i)$ where $G_i$ is a subgroup of $G$.
    Since $G$ is solvable, and by the Galois Correspondence, we have that $K$ is cyclic which implies that $K_{i+1}/K_i$ is also cyclic.

    Then if we attach the $k_i^{th}$ roots of unity, we can compose this with our field $F$ as well as to the chain of subfields above to give us another chain, which we will define as $F_1 = F_1K_0 \subset F_1K_1 \subset ... \subset F_1K_n = F_1K$. We therefore have a chain of extensions such that $k_i \in F_1$ $\forall i$ \hspace{0.1cm} $\in I$ where $I$ is the index for number of roots of unit. Since $K_{i+1}/K_i$ is cyclic, we also have that $F_1K_{i+1}/F_1K_i$ is cyclic. Thus, the extensions are simple radical extensions, which by \textcolor{red}{Theorem Here Lagrange} implies that $f(x)$ is solvable.
\end{proof}

\begin{theorem}[Abel-Ruffini Theorem]\label{thm:abel-ruffini-thm}
    The general polynomial of degree n is not solvable by radicals for $n \ge 5$
\end{theorem}

\begin{proof}
    By Lemma \ref{lemma:galois-symmetric}, we know that the general polynomial, written in the form $f(x)=a_0X^n - a_1X^{n-1} + ... + (-1)^na_n$ has Galois group isomorphic to $S_n$. Then by Theorem \ref{thm:symmetric-not-soluble} we can see that for $n \geq 5$, $S_n$ is not soluble, which by Theorem \ref{thm:galois-theorem} implies that the general polynomial of degree $n \geq 5$ is not solvable.
\end{proof}

\section{History of Galois Theory}

Notes:
One of the founders of modern algebra
Had to invent idea of a group-very abstract at time
Died in duel aged 20-1832
First paper was on cont. fracts
Expelled from Ecole Normale in Jan 1831
Focused on politics
Main paper left unpublished until 1846-Joseph Liouville explained things
Superseded work of Abel-Ruffini

\newpage
\appendix
\section{Group Theory Prerequisites}
Here we list some group theory results used in the text without proof. 

\begin{theorem}[First Isomorphism Theorem] \label{thm:first-iso}
	Suppose $\phi: G \to H$ is a group homomorphism. Then its kernel $\operatorname{Ker}(\phi)$ is a normal subgroup in $G$, its image $\operatorname{Im}(\phi)$ is a subgroup in $H$, and 
	$$
	G / \operatorname{Ker}(\phi) \cong \operatorname{Im}(\phi).
	$$
\end{theorem}

\begin{theorem}[Second Isomorphism Theorem] \label{thm:second-iso}
	Suppose $H \le G$ and $J \trianglelefteq G$. Then $HJ \le G$, $H \cap J \triangleleft H$ and $$
	(HJ) / J \cong H / (H \cap J). 
	$$
\end{theorem}
\begin{theorem}[Third Isomorphism Theorem] \label{thm:third-iso}
	Suppose $H, J \trianglelefteq G$ and $H \le J$. Then $J/H \trianglelefteq G/H$ and $$
	(G/H)/(J/H) \cong G / J.    $$
\end{theorem}

\begin{theorem}[Cauchy's Theorem] \label{thm:cauchy}
	Let $p$ be a prime number. Let $G$ be a finite group such that $p$ divides the order of $G$. Then $G$ contains an element of order $p$. 
\end{theorem}

\newpage
\bibliography{bibliography}
\end{document}
