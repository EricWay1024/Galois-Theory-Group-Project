\documentclass[12pt]{article}
\usepackage[utf8]{inputenc}
\usepackage{amsmath}
\usepackage{amsthm}
\usepackage{amsfonts}
\usepackage{graphicx}
\usepackage{tikz}
\usepackage{amssymb}
\usepackage{nicematrix}
\usepackage{wrapfig}
\usepackage[a4paper, total={6.5in, 9.4in}]{geometry}
\usepackage[labelsep=space]{caption}
\usepackage{enumitem}
\usepackage{natbib}
\usepackage{placeins}
\newtheorem{theorem}{Theorem}
\newtheorem{corollary}{Corollary}[theorem]
\newtheorem{definition}[theorem]{Definition}
\newtheorem{example}[theorem]{Example}
\usepackage[english]{babel}
\newtheorem{observation}[theorem]{\textbf{Observation}}
\bibliographystyle{unsrt}
\usepackage[none]{hyphenat}
\usepackage{fdsymbol}

\newcommand{\Gal}{\operatorname{Gal}}
\newcommand{\Aut}{\operatorname{Aut}}
\newcommand{\Fix}{\operatorname{Fix}}

\title{\textbf{Galois Theory}}
\author{Reece Wood, Adam Black, Yuhang Wei, Grace Anderson}
\date{January 2023}

\begin{document}

\maketitle

\tableofcontents

\newpage
\section{Introduction}
\textcolor{red}{Here we need to introduce the topics we talk about, the historical context behind Galois Theory and possibly the extensions we look at, as well as talking about the extensions}
\textcolor{blue}{The basis behind Galois Theory was an open question in mathematics until the start of the 1800s, and the theory originated and was developed due to the following question: ``Does there exist a general solution to a polynomial equation of order five?"  For equations of orders less than or equal to four, the Babylonians had already found a solution to this.
We first look at fields and extensions of fields in order to have a basis.}
\section{Definitions We Will Need}
\begin{itemize}
    \item Field - Let K be a field. This means that K is a commutative ring with 1 st every element in $K \backslash \{0\}$ has a multiplicative inverse.
    \item Tower of Fields
    \item Automorphism - A group automorphism is an isomorphism from a group to itself
    \item Monomorphism - A monomorphism is an injective homomorphism of a group $G \rightarrow{} H$
    \item Galois Group - Let $L/K$, and $G$ be the set of automorphisms of $L/K$. then $G$ is a group of transformations $L$, called the Galois Group of $L/K$ denoted $Gal(L/K)$.
    \item normal field extension
\end{itemize}
\section{Polynomials}
\section{Basics of Field Extensions}
\begin{definition}
Let \((G, *,\star )\) be a field. Let \(F\) be a subset of \(G\) such that \((F,*,\star)\) is a field. Then \((F, *,\star)\) is a \textbf{subfield}
 of \((G, *, \star)\).
\end{definition}
\begin{definition}
A field \(F\) is called a \textbf{prime field} of \(K\) if it had no proper (strictly smaller) subfield.
\end{definition}
\begin{definition}
A field \(K\) is said to be a \textbf{field extension} of \(F\), denoted \(K / F\), if \(F\) is a subfield of \(K\). \cite{Moy}
\end{definition}
\begin{example}
The field \(\mathbb{Q}[\sqrt{2}]\) is a field extension of \(\mathbb{Q}\), as \(\mathbb{Q}[\sqrt{2}] = \{a + \sqrt{2}b : a,b \in \mathbb{Q}\}\) we can clearly see that when \(b = 0\) this gives us the rational numbers so \(\mathbb{Q}\) is a subfield of \(\mathbb{Q}[\sqrt{2}]\).
\end{example}
\begin{definition}
For a real number \(\alpha\), if \(p(\alpha)=0\), for some polynomial \(p(x)\), \(\alpha\) is an \textbf{\textit{algebraic number}}.
\end{definition}
\begin{theorem}
If \(\alpha\) is an algebraic number, then \(\mathbb{Q}[\alpha]\) is a field. Similarly, \(\mathbb{Q}[\alpha,\beta]\) is a field for algebraic numbers \(\alpha\) and \(\beta\).
\end{theorem}
\begin{proof}
%proof of Qa as a field
We know, from the definition, that \(\mathbb{Q}[\alpha,\beta] = \{a + b\alpha + c\beta + d\alpha\beta : a,b,c,d \in \mathbb{Q}\} = \{(x\alpha + y)(p\beta + q) : x,y,p,q \in \mathbb{Q} \text{ where } a = yq, b = xq, c = py \text{ and } d = xp \} = \mathbb{Q}[\alpha,\beta] = K[\beta]\) where \(K = \mathbb{Q}[\alpha]\) is a field, hence we know that \(\mathbb{Q}[\alpha,\beta]\) is a field.  
\end{proof}
\begin{definition}
A \textbf{\textit{splitting field}} of a polynomial \(p(x)\) is the smallest field extension of \(\mathbb{Q}\) which contains all of the roots of \(p(x)\).
\end{definition}
\begin{example}
If we have a polynomial \(p(x) = x^4 - 12x^2 + 35\) the splitting field of \(p(x)\) is \(\mathbb{Q}[\sqrt{5},\sqrt{7}]\) as it contains all of the roots of \(p(x)\) and if it was any smaller it would not contain all of the roots or would not be a field.
\end{example}
\begin{definition}
    The degree of the field extension, $deg(L:K)=[L:K]$ is the dimension of the vector space $L$ over $K$
\end{definition}

\begin{definition}
    If $L/K$ is a field extensions, we call any field $M$ with $K \subseteq M \subseteq L$ an \textbf{intermediate field.}
\end{definition}
\begin{theorem}
(Tower Theorem): If M is an intermediate field of a finite field extension $L/K$ then:
\begin{equation*}
    [L:K] = [L:M]\cdot[M:K]
\end{equation*}
\end{theorem}
\begin{proof}
Since we have M is an intermediate field, $K \subseteq M \subseteq L$,

Suppose that we have $[L:M]=m$ and $[M:K]=n$. Then since we have that the subfields are vector spaces, we can takes a basis for each of our field extensions $L/M$ and $M/K$.
Let $\lambda = \{\lambda_1,...,\lambda_m\}$ be a basis for $L/M$, and let $\mu = \{\mu_1,...,\mu_n\}$ be a basis for $M/K$.

\noindent Then we have that $\forall x \in L$, $x = \sum^m_{i=1}x_i\lambda_i$ for some $x_i \in M$, since $\lambda$ is a basis for $L/M$.

\noindent Now let $b:=\sum^n_{j=1}\mu_j$ and $d_i:=\frac{x_i}{b}$

\noindent Then we have that $x=\sum^m_{i=1}\frac{x_i}{b}\cdot b \cdot \lambda_i = \sum^m_{i=1}\sum^n_{j=1}d_i\cdot \mu_j \cdot \lambda_i$

\noindent If we consider the set $\gamma=\{\lambda_i\mu_j : 1\leq i \leq m, 1\leq j \leq n\}$, and consider from what we have seen above that $\forall x \in L$ we can write $x$ \textcolor{red}{Find the right words here}, we can see that $\gamma$ spans $L/K$.

Now it is required to show that $\gamma$ is linearly independent, in order to show that it is a basis for $L/K$.

Suppose that for some $c_{ij} \in K$: $\sum^m_{i=1} \sum^n_{j=1} c_{ij}\lambda_i\mu_j = 0 $. Then since $\lambda$ is a linearly independent set over $M$ we have that $\forall i \in \{1,...,m\}$:\hspace{0.2cm} $\sum^n_{j=1} c_{ij}\mu_j = 0 $. By seeing that $\mu$ is also a linearly independent set over $K$ we get that $\forall i \ in \{1,...,m\}$ and $\forall j \ in \{1,...,n\}$ we have $c_{ij} = 0$. Which then implies that $\gamma$ is linearly independent and we already know it spans $L/K$, thus it is a basis for $L/K$ and since $|\gamma|=mn$ this implies that $[L:K] = mn = [L:M]\cdot[M:K]$

\end{proof}
\section{Introduction to Galois Theory}
\subsection{Galois Groups}
\begin{definition}
    Given a field extension $L/K$, its Galois Group $\Gal(L/K)$ is the group of all $K$-automorphisms $\phi$ of $L$, such that $\phi(a) = a $ for any $ a \in K$.
\end{definition}

This group becomes increasingly useful, especially when dealing with polynomials, as we will see with the following theorem.

\begin{theorem}
Let $\phi \in Gal(L/K)$ where $L/K$ is a field extension. Then for $a \in L$, a root of a polynomial $f \in K[X]$, we have $\phi(a)$ is also a root of $f(X)$.
\end{theorem}

\section{Galois Field Extensions}
\begin{definition}
A field extension is said to be a \textbf{Galois extension} if it is finite, normal and separable. For $L/K$ Galois, we have that $\Aut(L/K)$ is the \textbf{Galois group} and denoted by $\Gal(L/K).$
\end{definition}

\noindent We now look at a theorem which helps to characterise and identify fields, extensions and Galois groups.

\begin{theorem}
    Given a finite Galois extension $L/K$ and its Galois group $G=Gal(L/K)$, the following holds:
    \begin{itemize}
        \item $K = Fix(G)$
    \end{itemize}
\end{theorem}

\begin{proof}
    Define $F:=Fix(G)$. Since $L/K$ is a finite Galois extensions and G is its Galois group, we have $L/F$ must also be normal and separable, and therefore is also a Galois extension, with $G \subseteq Gal(L/F)$.

\noindent We can also see that $|Gal(L/F)| = [L:F]$ which implies that $|G|\leq|Gal(L/K)|=[L:F]$
Then by the Tower Theorem we see that $|G|=[L:K]=[L:F][F:K]\geq [L:F]$

Thus we see that $[L:F]\leq|G|\leq[L:F]$, therefore $|G|=[L:F]$ and therefore in the tower theorem, we see that $[F:K]$ must equal 1, and thus $F:=Fix(G) = K$
\end{proof}

\section{Galois Correspondence}
\subsection{Fundamental Theorem of Galois Theory}

\begin{definition}
    A \textbf{normal closure} of a field extension $L / K$ is an extension $N$ of $L$ such that 
    \begin{enumerate}
        \item $N / K$ is normal;
        \item If $L \subseteq M \subseteq N$ and $M / K$ is normal, then $M = N$.
    \end{enumerate}
\end{definition}

\begin{theorem}
    Let $G$ be a finite subgroup of the group of automorphisms of a field $K$, and let $K_0$ be the fixed field of $G$. Then $[K : K_0] = |G|$. 
\end{theorem}

\begin{proof}
    
\end{proof}

Group theory tends to look at patterns and symmetries in mathematical objects, and Galois groups have an interesting symmetry. The following theorem describes how the structure of an extension of a field is the same as the subgroups of the Galois groups.

\begin{theorem}
Given a finite Galois extension $L/K$ and its Galois group $G = \Gal(L/K)$, there is a natural bijection between subgroups $H\leq G$ and the intermediate fields $M$ \hspace{0.1cm} $(K \subseteq M \subseteq L)$:

\begin{itemize}
    \item $\alpha:F \to \Gal(L/M) \subseteq G$
    \item $\beta:H \to \Fix(H) \subseteq L$
\end{itemize}
\end{theorem}
\begin{proof}
Proof will go here
\end{proof}
\subsection{Abel-Ruffini Theorem}



\section{Galois Groups and Polynomials}

\subsection{Soluble and Simple Groups}

\begin{definition} \label{def:soluble}
    A group $G$ is soluble (or solvable) if it has a finite series of subgroups 
    $$ \{ e \} = G_0 \triangleleft G_1 \triangleleft \dots \triangleleft G_n = G$$
    such that the quotient $G_{i+1} / G_{i}$ is abelian for each $i = 0, 1, ...,  n - 1$.
\end{definition}

\begin{observation}
    An abelian group is soluble. 
\end{observation}

We now state two group isomorphism theorems without proving. 

\begin{theorem}[Second Isomorphism Theorem] \label{thm:second-iso}
    Suppose $H \le G$ and $J \trianglelefteq G$. Then $HJ \le G$, $H \cap J \triangleleft H$ and $$
    (HJ) / J \cong H / (H \cap J). 
    $$
\end{theorem}
\begin{theorem}[Third Isomorphism Theorem] \label{thm:third-iso}
    Suppose $H, J \trianglelefteq G$ and $H \le J$. Then $J/H \trianglelefteq G/H$ and $$
    (G/H)/(J/H) \cong G / J.    $$
\end{theorem}

\begin{theorem} \label{thm:soluble-main}
    Let $G$ be a group, $H \le G$ and $N \trianglelefteq G$. Then 
    \begin{enumerate}
        \item If $G$ is soluble, then $H$ is soluble;
        \item If $G$ is soluble, then $G / N$ is soluble; 
        \item If $N$ and $G / N$ are soluble, then $G$ is soluble. 
    \end{enumerate}
\end{theorem}
\begin{proof}
\begin{enumerate}
    \item If $G$ is soluble, then there exists a finite series of subgroups $G_i$ of $G$ for $i = 0, 1, \dots, n$ satisfying Definition \ref{def:soluble}. Let $H_i = G_i \cap H$. Then $H$ has a series of subgroups $H_i$ such that $\{ e \} = H_0 \triangleleft H_1 \triangleleft \dots \triangleleft H_n = H.$
    For each $i = 0, 1, \dots, n - 1$, 
    $$
    \frac{H_{i+1}}{H_i} 
        = \frac{G_{i+1} \cap H}{G_i \cap (G_{i+1} \cap H)}
        \cong \frac{G_i(G_{i+1} \cap H)} {G_i}
    $$
    by Theorem \ref{thm:second-iso}, and ${G_i(G_{i+1} \cap H)}/{G_i}$ is a subgroup of the abelian group $G_{i+1} / G_{i}$. Hence $H_{i+1} / H_{i}$ is abelian and $H$ is soluble.
    \item Take $G_i$ as before. Then $G / N$ has a series
        $N/N = G_0 N / N \triangleleft G_1 N / N \triangleleft \dots \triangleleft G_n N / N  =  G / N. $
        For each $i = 0, 1, \dots, n - 1$, 
        $(G_{i+1} N / N) / (G_{i} N / N) \cong (G_{i+1} N) / (G_i N)$
    by Theorem \ref{thm:third-iso}. Then 
    $$
    \frac{G_{i+1} N}{G_i N} =\frac{G_{i+1}\left(G_i N\right)}{G_i N} \cong \frac{G_{i+1}}{G_{i+1} \cap\left(G_i N\right)} \cong \frac{G_{i+1} / G_i}{\left(G_{i+1} \cap\left(G_i N\right)\right) / G_i},
    $$
    which is a quotient of the abelian group $G_{i+1} / G_i$, so is abelian. Hence $G / N$ is soluble.
    \item There exist two series
$$
\begin{aligned}
\{ e \} & =N_0 \triangleleft N_1 \triangleleft \ldots \triangleleft N_r=N, \\
N / N & =G_0 / N \triangleleft G_1 / N \triangleleft \ldots \triangleleft G_s / N=G / N
\end{aligned}
$$
with $N_{i+1} / N_{i}$ abelian for each $i = 0, 1, \dots, r-1$ and $(G_{i+1} / N)  / (G_{i} / N) \cong G_{i+1} / G_i $ abelian for each $i = 0,1, \dots, s-1$. Combining them gives the series of subgroups of $G$:
$$
\{ e \}=N_0 \triangleleft N_1 \triangleleft \ldots \triangleleft N_r=N=G_0 \triangleleft G_1 \triangleleft \ldots \triangleleft G_s=G .
$$
The quotients are either $N_{i+1} / N_i$  or $G_{i+1} / G_i$ and are all abelian. Therefore $G$ is soluble.
\end{enumerate}
\end{proof}

\begin{definition}
    A group $G$ is \textbf{simple} if it is nontrivial and its only normal subgroups are $\{ e \}$ and $G$. 
\end{definition}

\begin{theorem} \label{thm:soluble-and-simple}
    A soluble group is simple if and only if it is a cyclic group of prime order.
\end{theorem}

\begin{proof}
    Let $G$ be a soluble group and suppose $G$ is simple. Consider the series of subgroups of $G$
$$
\{ e \}=G_0 \triangleleft G_1 \triangleleft \ldots \triangleleft G_n=G,
$$
with abelian quotients, and without loss of generality we assume $G_{i+1} \neq G_i$. Since $G$ is simple, $G_{n-1}$, which is a proper normal subgroup of $G_n = G$, must be $\{ e \}$. Solubility of $G$ gives that $G_n / G_{n -1 }$ is abelian, but $G_n / G_{n - 1} = G$, and thus $G$ is abelian. Thus for any element $g \in G$, the cyclic group $\langle g\rangle$ is a normal subgroup in $G$. Hence  $G = \langle g\rangle$ for any $g \neq e$. Hence $G$ is cyclic of prime order.

The converse is trivial.
\end{proof}


\begin{theorem} \label{thm:simple-alternating}
    The alternating group $A_n$ is simple when $n \ge 5$. 
\end{theorem}

\begin{proof}
    (Prove for the case $n=5$ ??) Due to the length and technical nature of the complete proof, only a concise summary is presented here. 
    Suppose that $\{ e \} \neq N \triangleleft A_n$. We can prove that $N$ must contain a $3$-cycle using case-by-case analysis. Next, we can show that if $N$ contains a $3$-cycle, then it contains all $3$-cycles. Since $A_n$ is generated by the $3$-cycles when $n \ge 3$, this means $N = A_n$.
\end{proof}

\begin{theorem} \label{thm:symmetric-not-soluble}
    The symmetric group $S_n$ is not soluble when $n \ge 5$. 
\end{theorem}

\begin{proof}
    Suppose $S_n$ is soluble for $n \ge 5$. Then since $A_n \trianglelefteq S_n$, by Theorem \ref{thm:soluble-main}, $A_n$ is soluble. By Theorem \ref{thm:simple-alternating}, $A_n$ is also simple, so $A_n$ is cyclic of prime order by Theorem \ref{thm:soluble-and-simple}. But $|A_n| = n! / 2$ which obviously is not prime. Contradiction.
\end{proof}

Another important result from group theory is given below.

\begin{theorem}[Cauchy's Theorem] \label{thm:cauchy}
    Let $p$ be a prime number. Let $G$ be a finite group such that $p$ divides the order of $G$. Then $G$ contains an element of order $p$. 
\end{theorem}

\subsection{Roots of Unity}
\begin{definition}
    Let $n$ be a positive integer. An $n$-th root of unity in $\mathbb C$ is a number $\omega$ satisfying $\omega ^ n = 1$. An $n$-th root of unity $\omega$ is primitive if for any $m = 1, 2, \dots, n - 1$, $\omega ^ m \neq 1$.
\end{definition}

\begin{observation}
    Let $n$ be a positive integer. All $n$-th roots of unity form a cyclic group under multiplication, and the group is generated by any primitive $n$-th root of unity.
\end{observation}

\begin{observation}
    For a prime number $p$, let $\omega$ be a $p$-th root of unity and $\omega \neq 1$. Then $\omega$ is a primitive $p$-th root of unity and generates the multiplicative group of all $p$-th roots of unity.
\end{observation}

\begin{theorem} \label{thm:unity-1}
    Let $n$ be a positive integer and let $\omega$ be a primitive $n$-th root of unity. Let $L$ be a subfield of $\mathbb C$. If $\omega \in L$, then the polynomial $t^n - 1$ splits in $L$.
\end{theorem}
\begin{proof}
    $\omega$ has order $n$ in the multiplicative group of $L$, so the elements $1, \omega, \omega^2, \ldots, \omega^{n-1}$ are distinct $n$-th roots of unity in $L$. Therefore $t^n-1$ splits in $L$.
\end{proof}

\begin{theorem} \label{thm:unity-2}
    Let $n$ be a positive integer and let $\omega$ be a primitive $n$-th root of unity. If $L$ is the splitting field for $t^n - 1$ over a subfield $K$ of $\mathbb C$, then $L = K(\omega)$.
\end{theorem}

\begin{proof}
The derivative of $t^n-1$ is $n t^{n-1}$, which is prime to $t^n-1$, so the polynomial $t^n-1$ \textcolor{red}{has no multiple zeros} in $L$. The group of its zeros under multiplication thus has order $n$ and is cyclic. Let $\omega$ be a generator of this group, and thus $\omega$ is a primitive $n$-th root of unity. Then $L=K(\omega)$. 
\end{proof}


\begin{theorem} \label{thm:radical-1}
    Let $p$ be a prime and let $\omega$ be a primitive $p$-th root of unity. Let $K$ be a subfield of $\mathbb C$. Then $\Gal(K(\omega) / K)$ is abelian.
\end{theorem}
\begin{proof}
Let $L = K(\omega)$.  Let $\alpha \in \Gal(L / K)$. Then $\alpha$ is uniquely determined by $\alpha(\omega)$ and $\alpha$ is a permutation of the multiplicative group generated by $\omega$. Thus $\alpha$ has the form
$$
\alpha_j: \omega \mapsto \omega^j,
$$
where $j=1,\dots,p-1$. Then $\alpha_i \alpha_j (\omega) = \alpha_j \alpha_i (\omega) = \omega^{i j}$, so $ \Gal(L / K)$ is abelian.
\end{proof}

\begin{theorem} \label{thm:radical-2}
    Let $K$ be a subfield of $\mathbb{C}$ which contains an $n$-th primitive root of unity. Let $\beta^n = c \in K $ where $n$ is a positive integer. Then $t^n - c$ splits in $K(\beta)$ and $\Gal(K(\beta) / K)$ is abelian.
\end{theorem}

\begin{proof}
By Theorem \ref{thm:unity-1}, $K$ contains all $n$-th roots of unity. Let $L = K(\beta)$, and $\beta$ is a zero of $t^n-c$ in $L$. Any zero of $t^n-c$ in $L$ can be represented by $\omega \beta$, where $\omega$ is some $n$-th root of unity in $K$. Thus $t^n - c$ splits in $L$.  

Let $\phi \in \Gal(L / K)$, then $\phi$ is uniquely determined by $\phi(\beta)$ and $\phi$ is a permutation of the zeros of $t^n - c$. Let $\phi_1, \phi_2 \in \Gal(L / K)$, and let $\phi_1(\beta) = \omega_1\beta$, $\phi_2(\beta) = \omega_2\beta$, where $\omega_1, \omega_2$ are $n$-th roots of unity in $K$. Then
$$
\phi_1 \phi_2(\beta)=\omega_1 \omega_2 \beta=\omega_2 \omega_1  \beta=\phi_2 \phi_1(\beta).
$$
Thus $\Gal(L / K)$ is abelian.
\end{proof}

\begin{theorem} \label{thm:unity-3}
Let $K$ be a subfield of $\mathbb C$. Let $\alpha \notin K$ and $\alpha^p = c \in K$ where $p$ is a prime. Let $L / K$ be a normal extension such that $\alpha \in L$. Then $L$ contains a primitive $p$-th root of unity.
\end{theorem}

\begin{proof}
Let $f$ be the minimal polynomial of $\alpha$ over $K$. Then $f$ splits in $L$ and has no repeated zeros. Since $\alpha \notin K$, we have $\partial f \ge 2$, and therefore there exists $\beta \in L$ such that $\beta$ is a zero of $f$ and $\beta \neq \alpha$. Since $\alpha ^ p - c = 0$, $f$ must divide $t ^ p - c$. Thus $\beta^p - c = 0$. Let $\omega=\alpha / \beta \in L $, then $\omega^p=1$ and $\omega \neq 1$.
\end{proof}


\subsection{Solubility by Radicals}
\begin{definition} \label{def:radical-extension}
    A field extension $L / K$ is \textbf{radical} if $L=K\left(\alpha_1, \ldots, \alpha_m\right)$, where for each $i=1, \ldots, m$ there exists $n_i$ such that
$$
\alpha_i^{n_i} \in K\left(\alpha_1, \ldots, \alpha_{i-1}\right).
$$
The elements $\alpha_i$ form a radical sequence for $L / K$, and the radical degree of $\alpha_i$ is $n_i$.
\end{definition}

\begin{theorem} \label{thm:radical-single-prime}
    Let $K$ be a field. If $\alpha ^ n \in K$ for a positive integer $n$ and  $L = K (\alpha)$, then there exists a radical sequence $\beta_i$ for $L / K$ such that the radical degree for each $\beta_i$ is prime.
\end{theorem}

\begin{proof}
    Let $n = \prod_{k=1}^{s} p_{k}$ where $p_{k}$ are prime numbers (not necessarily distinctive). Let $\beta_{i} = \alpha^ {p_{i + 1} \dots  p_{s}}$ for $i = 0, \dots s$. Write $L = K(\beta_1,  \dots, \beta_s = \alpha)$, and then $\beta_0 = \alpha^n \in K$ and  $\beta_i ^ {p_i} = \beta_{i-1} \in K(\beta_1, \dots, \beta_{i - 1})$ for each $i  = 1, \dots, s$. 
\end{proof}

\begin{example}
    Consider $K  = \mathbb Q$ and $\alpha = \sqrt[20]{2}$, where $\alpha ^ {20} \in \mathbb Q$. Write $n = 20 =  2 \times 2 \times 5 $. Then we can take $\beta_1 = \sqrt 2, \beta_2 = \sqrt[4]{2}, \beta_3 = \sqrt[20]{2}$. Then $\mathbb Q(\sqrt[20]{2}) $ $= \mathbb Q(\sqrt{2}, \sqrt[4]{2}, \sqrt[20]{2})$ and 
    $$
    \beta_1 ^ 2 \in \mathbb Q, \quad \beta_2 ^ 2 \in \mathbb Q(\sqrt{2}), \quad \beta_3^5 \in \mathbb Q (\sqrt{2}, \sqrt[4]{2}),
    $$
    where each of the radical degrees $2, 2, 5$ is prime.
\end{example}

\begin{theorem} \label{thm:radical-all-prime}
    Let $L / K$ be a racial extension. Then there exists a radical sequence $\beta_j$ for $L / K$ where $j=1, \dots, r$ such that the radical degree of each $\beta_j$ is prime.
\end{theorem}

\begin{proof}
    For each $\alpha_i$ in any given radical sequence for $L / K$, replace it with a series $\beta_{ij}$ by Theorem \ref{thm:radical-single-prime}. Then combine all $\beta_{ij}$ and relabel the subscripts to make $\beta_{j}$. 
\end{proof}

\begin{theorem} \label{thm:radical-3}
If $K$ is a subfield of $\mathbb{C}$ and $L / K$ is normal and radical, then $\Gal(L / K)$ is soluble.
\end{theorem}

\begin{proof}
Let $L=K\left(\alpha_1, \ldots, \alpha_n\right)$ with $\alpha_j^{n_j} \in K\left(\alpha_1, \ldots, \alpha_{j-1}\right)$. By Theorem \ref{thm:radical-all-prime}, we may assume that $n_j$ is prime for all $j$. We prove the result by induction on $n$. 
 % In particular there is a prime $p$ such that $\alpha_1^p \in K$.

If $n = 0$, we have $L = K$, and the case is trivial.

If $n \ge 1$ and $\alpha_1 \in K$, then $L=K\left(\alpha_2, \ldots, \alpha_n\right)$ and $\Gal(L / K)$ is soluble by induction.

Let $n \ge 1$ and $\alpha_1 \notin K$. Let $p = n_1$, which is prime, and then $\alpha_1^p \in K$.  Then $L$ contains a primitive $p$-th root of unity $\omega$ by Theorem \ref{thm:unity-3}. Let $M = K(\omega)$ and let $N = M(\alpha_1)$. Consider the chain of subfields $K \subseteq M \subseteq N \subseteq L$. We now prove the solubility of $\Gal(L/N), \Gal(N/M), \Gal(L/M), \Gal(M/K)$, and $\Gal(L/K)$ step by step:

\begin{itemize}
    \item Since $L=N\left(\alpha_2, \ldots, \alpha_n\right)$, $L / N$ is a normal radical extension. By induction $\Gal\left(L / N\right)$ is soluble. 
    \item Since $ \omega \in M$ and $\alpha_1^p \in M$, Theorem \ref{thm:radical-2} implies that $N$ is a splitting field for $t^p-\alpha_1^p$ over $M$. Thus $N / M$ is normal. By Theorem \ref{thm:radical-2}, $\Gal\left(N / M\right)$ is abelian and hence soluble. 
    \item  $L/ K$ is finite and normal, and so is $L / M$. Apply \textcolor{red}{Galois Correspondence} to $L / M$ to deduce that
    $$
    \Gal\left( N / M \right) \cong \Gal(L / M) / \Gal\left(L / N\right).
    $$
    Hence by Theorem \ref{thm:soluble-main},  $ \Gal(L / M)$ is soluble.
    \item By Theorem \ref{thm:unity-1}, $M$ is the splitting field for $t^p-1$ over $K$. Then the extension $M / K$ is normal. By Theorem \ref{thm:radical-1}, $\Gal(M / K)$ is abelian and hence soluble.
    \item  \textcolor{red}{Galois Correspondence} applied to $L / K$ yields
    $$
    \Gal(M / K) \cong \Gal(L / K) / \Gal(L / M). 
    $$
    Theorem \ref{thm:soluble-main} shows that $\Gal(L / K)$ is soluble, completing the induction step.
\end{itemize}
\end{proof}

\begin{theorem} \label{thm:radical-closure}
    If $L / K$ is a radical extension in $\mathbb{C}$ and $M$ is the normal closure of $L / K$, then $M / K$ is radical.
\end{theorem}

\begin{proof}
\textbf{TODO paraphrase}

Let $L=K\left(\alpha_1, \ldots, \alpha_m\right)$ with $\alpha_i^{n_i} \in K\left(\alpha_1, \ldots, \alpha_{i-1}\right)$. Let $f_i$ be the minimal polynomial of $\alpha_i$ over $K$. Then $M \supseteq L$ is clearly the splitting field of $\prod_{i=1}^m f_i$. For every zero $\beta_{i j}$ of $f_i$ in $M$ there exists an isomorphism $\sigma : K\left(\alpha_i\right) \rightarrow K\left(\beta_{i j}\right)$ by \textbf{Corollary 5.13}. By \textbf{Proposition 11.4}, $\sigma$ extends to a $K$-automorphism $\tau: M \rightarrow M$. Since $\alpha_i$ is a member of a radical sequence for a subfield of $M$, so is $\beta_{i j}$. Combining the sequences yields a radical sequence for $M$.
\end{proof}


\begin{definition}
    Let $f$ be a polynomial over a subfield $K$ of $\mathbb{C}$, with splitting field $\Sigma$ over $K$. Then $f$ is \textbf{soluble by radicals} if there exists a field $M \supseteq \Sigma$ such that $M / K$ is a radical extension.  
\end{definition}


\begin{definition}
    Let $f$ be a polynomial over a subfield $K$ of $\mathbb{C}$, with splitting field $\Sigma$ over $K$. The \textbf{Galois group} of $f$ over $K$ is defined as $\Gal(\Sigma / K)$.
\end{definition}

\begin{theorem} \label{thm:radical-galois-soluble}
    Let $f$ be a polynomial over a subfield $K$ of $\mathbb{C}$. If $f$ is soluble by radicals, then the Galois group of $f$ over $K$ is soluble.
\end{theorem}

\begin{proof}
\textbf{TODO paraphrase}

Since $f$ is soluble by radicals, its splitting field $\Sigma$ over $K$ satisfies $K \subseteq \Sigma \subseteq M$ where $M / K$ is a radical extension. Let $K_0$ be the fixed field of $\Gal(\Sigma / K)$, and let $N / M$ be the normal closure of $M / K_0$. Then
$$
K \subseteq K_0 \subseteq \Sigma \subseteq M \subseteq N.
$$
Since $M / K_0$ is radical, Theorem \ref{thm:radical-closure} implies that $N / K_0$ is a normal radical extension. By Theorem \ref{thm:radical-3}, $\Gal\left(N / K_0\right)$ is soluble.
By \textbf{Theorem ??}, the extension $\Sigma / K_0$ is normal. By \textbf{Theorem ??}
$$
\Gal\left(\Sigma / K_0\right) \cong \Gal\left(N / K_0\right) / \Gal(N / \Sigma).
$$
Theorem \ref{thm:soluble-main} implies that $\Gal\left(\Sigma / K_0\right)$ is soluble. But $\Gal(\Sigma / K)=\Gal\left(\Sigma / K_0\right)$, so $\Gal(\Sigma / K)$ is soluble.

\end{proof}

The converse also holds, but we do not prove it here.

\subsection{An Insoluble Quintic}
We first look at how symmetric groups can be generated by its elements. 

\begin{theorem} \label{thm:symmetric-12-12n}
    The transposition $(12)$ and $n$-cycle $(12 \dots n)$ together generate $S_n$. 
\end{theorem}
\begin{proof}
    Consider the cyclic decomposition of each element in $S_n$, where each cyclic permutation can be written as 
    $
    (a_1a_2\dots a_k) = (a_1 a_k) \dots (a_1 a_3) (a_1 a_2). 
    $
    Note that $(ab) = (1a)(1b)(1a)$, therefore $(12), (13), \ldots (1n)$ together generate $S_n$. Also note that  
    $$(1k)=((k-1)k)\dots(34)(23)(12)(23)(34)\dots((k-1)k),$$
    therefore $(12), (23), \dots, ((n-1)n)$ together generate $S_n$. Finally note that 
    $$
    (k(k+1)) = (12\dots n)^{k-1} (12) (12\dots n)
    $$
    for $k = 2, 3, \dots n - 1$. Therefore $(12)$ and $(12 \dots n)$ together generate $S_n$. 
\end{proof}

\begin{theorem} \label{thm:symmetric-ab-12n}
    For $1 \le a < b \le n$ such that $(b - a, n) = 1$, the transposition $(ab)$ and $n$-cycle $(12 \dots n)$ together generate $S_n$.
\end{theorem}
\begin{proof}
   Let $\sigma=(12 \ldots n)$, so $\sigma^i(a) \equiv a+i \bmod n$. Therefore $\sigma^{b-a}(a) \equiv b \bmod n$, and since $1 \le b \le n$, we have $\sigma^{b-a}(a)=b$. Since $(b-a, n)=1,\langle\sigma\rangle=\left\langle\sigma^{b-a}\right\rangle$ and $\sigma^{b-a}$ is an $n$-cycle sending $a$ to $b$, so $\sigma^{b-a}$ is of the form $(a b \ldots)$. Then
$
\langle(a b), \sigma\rangle=\left\langle(a b), \sigma^{b-a}\right\rangle=\langle(a b),(a b \ldots)\rangle .
$
Relabel the numbers $1,2 \ldots, n$ so that $(a b)$ turns into $(12)$ and $(a b \ldots)$ into $(12 \ldots n)$. Therefore $\langle(a b), \sigma\rangle=S_n$ by Theorem \ref{thm:symmetric-12-12n}.
\end{proof}

\begin{theorem} \label{thm:symmetric-prime}
    For a prime number $p$, any transposition and $p$-cycle together generate $S_p$.
\end{theorem}
\begin{proof}
    Relabel the numbers so that the $p$-cycle turns into $(12 \dots p)$. Suppose the transposition turns into $(ab)$, where $1 \le a < b \le n$. Since $p$ is prime and $1 \le b - a < p$, we have $(b - a, p) = 1$. The result thus follows from Theorem \ref{thm:symmetric-ab-12n}.
\end{proof}

Now we are ready to establish a case where the Galois group of a polynomial over $\mathbb Q$ is isomorphic to a symmetric group.

\begin{theorem} \label{thm:galois-iso-symmetric}
    Let $p$ be a prime number and $f$ be a irreducible polynomial of degree $p$ over $\mathbb Q$. If $f$ has exactly two non-real zeros in $\mathbb C$, then the Galois group of $f$ over $\mathbb Q$ is isomorphic to the symmetric group $S_p$.
\end{theorem}

\begin{proof}
    \textcolor{red}{The Fundamental Theorem of Algebra} implies that there exists a splitting field $\Sigma $ of $f$ contained in $\mathbb C$. The Galois group $\Gal(\Sigma / \mathbb Q)$ of $f$ can be considered as a permutation group of the zeros of $f$ \textbf{(Group action??)}. By Theorem ??, $f$ has distinct zeros, so $\Gal(\Sigma / \mathbb Q)$ is isomorphic to a subgroup of $S_p$. We denote this subgroup of $S_p$ as $G$. 

    Complex conjugation restricted to $\Sigma$ is a $\mathbb Q$-automorphism of $\Sigma$. It fixes all $p - 2$ real zeros of $f$ and transposes the two non-real zeros. Therefore $G$ contains a transposition. 

    Take any zero $\alpha$ of $f$. Since $f$ is irreducible, it is the minimal polynomial of $\alpha$. Then by the \textbf{Degree Theorem}, $$[\mathbb Q(\alpha) : \mathbb Q] = \partial f = p. $$ The \textbf{tower law} implies that $$[\Sigma : \mathbb Q] = [\Sigma : \mathbb Q(\alpha)] [ \mathbb Q(\alpha) : \mathbb Q]. $$ By \textbf{Galois Correspondence}, $|G| = [\Sigma : \mathbb Q]$, and thus $p$ divides $|G|$. Theorem \ref{thm:cauchy} gives the existence of an element of order $p$ in $G$, but the only elements of order $p$ in $S_p$ are $p$-cycles. Therefore $G$ contains a $p$-cycle.

    Hence by Theorem \ref{thm:symmetric-prime}, $ G = S_p$ and therefore $\Gal(\Sigma / \mathbb Q) \cong S_p$.
\end{proof}

We now give a quintic polynomial not soluble by radicals. 

\begin{example}
    The polynomial $x^5 + 10 x^4 - 2$ over $\mathbb Q$ is not soluble by radicals.
\end{example}

\begin{proof}
    $f$ is irreducible by \textbf{Eisenstein's criterion} with $p = 2$. The degree of $f$ is $5$, which is prime. Calculus shows that $f$ has three real zeros. Thus the Galois group of $f$ is isomorphic to $S_5$ by Theorem \ref{thm:galois-iso-symmetric}. But $S_5$ is not soluble by Theorem \ref{thm:symmetric-not-soluble}. Thus by Theorem \ref{thm:radical-galois-soluble}, $f$ is not soluble by radicals.
\end{proof}

\section{History of Galios Theory}

Notes:
One of the founders of modern algebra
Had to invent idea of a group-very abstract at time
Died in duel aged 20-1832
First paper was on cont. fracts
Expelled from Ecole Normale in Jan 1831
Focused on politics
Main paper left unpublished until 1846-Joseph Liouville explained things
Superseded work of Abel-Ruffini

\newpage
\bibliography{bibliography}
\end{document}
